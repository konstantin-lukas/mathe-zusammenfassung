\documentclass[12pt]{article}

%opening
\title{Mathematik fürs Informatikstudium und Abitur:\\Eine Zusammenfassung}
\author{Konstantin Lukas}
\renewcommand{\contentsname}{Inhaltsverzeichnis}
\usepackage{xstring}
\usepackage{catchfile}

\newcommand{\monthword}[1]{\ifcase#1\or Januar\or Februar\or März\or April\or
	Mai\or Juni\or Juli\or August\or
	September\or Oktober\or November\or Dezember\fi}
\date{Fassung vom \the\day . \monthword{\month} \the\year}
\usepackage{amsmath,amsfonts,amssymb,amsthm}
\usepackage{braket}
\usepackage[a4paper, margin=2cm]{geometry}
\usepackage{bbold}
\usepackage{pgffor,listofitems}
\usepackage{tocloft}
\usepackage{etoc}
\usepackage{blindtext}
\usepackage{tikz}
\usepackage{array}
\usepackage[most]{tcolorbox}
\usepackage[hidelinks]{hyperref}
\usepackage{pgfplots}
\usepackage{tocloft}
\usepackage{polynom}
\usepackage{tkz-euclide}
\usepackage[ngerman]{babel}
\usepackage{hyphenat}
\hyphenation{Mathe-matik wieder-gewinnen}
\definecolor{gridgray}{HTML}{AAAAAA}
\renewcommand{\cftsecleader}{\cftdotfill{\cftdotsep}}
\DeclareRobustCommand{\bigfrac}[3][5pt]{%
	\frac{\hspace{#1}#2\hspace{#1}}{\hspace{#1}#3\hspace{#1}}}
\newcommand{\highlight}[2]{\textcolor{blue}{\hyperref[#1]{#2}} (S. \pageref{#1})}
\newcommand{\getcolor}[1]{\ifcase#1\or blue\or red\or teal\or violet\or
	magenta\or orange\or purple\or brown\fi}
\newcommand{\makeplot}[9]{
	\readlist\xlimits{#4}
	\def\xlower{\xlimits[1]}
	\def\xupper{\xlimits[2]}
	\readlist\ylimits{#5}
	\def\ylower{\ylimits[1]}
	\def\yupper{\ylimits[2]}
	\readlist\dimensions{#8}
	\def\width{\dimensions[1]}
	\def\height{\dimensions[2]}
	\begin{center}
		\begin{tikzpicture}
		\begin{axis}[
		domain=\xlower:\xupper,
		width=\width,
		height=\height,
		restrict y to domain=#6,
		xmin=\xlower, xmax=\xupper,
		ymin=\ylower, ymax=\yupper,
		samples=#7,
		axis y line=center,
		axis x line=middle,
		ticklabel style={fill=white},
		minor tick num=2,
		grid=both,
		grid style={line width=.1pt, draw=gridgray!10},
		major grid style={line width=.2pt,draw=gridgray!50}
		]
		\foreach \graph [count=\i] in {#1} {
			\edef\temp{\noexpand\addplot+[mark=none, color=\getcolor{\i}, solid, #9] {\graph};}
			\temp
		}
		
		
		\end{axis}
		\readlist\pos{#3}
		\foreach \label [count=\i] in {#2} {
			\node [color=\getcolor{\i}] at (\pos[\i]) {\label};
		}
		
		\end{tikzpicture}
	\end{center}
}

\begin{document}
\maketitle
\tableofcontents
\pagebreak
\etocsettocstyle{\noindent\rule{\linewidth}{.4pt}}

\section{Mengen}
	\subsection{Vereinigung}
		\begin{flalign*}
			\begin{tikzpicture}
			\fill[yellow] (3,0) circle (2cm);
			\fill[yellow] (0,0) circle (2cm);
			\draw (0,0) circle (2cm);
			\draw (3,0) circle (2cm);
			\draw (0,0) node {A};
			\draw (3,0) node {B};
			\end{tikzpicture}&&
		\end{flalign*}
		\begin{flalign*}
			A \cup B := \{x \mid x \in A \: oder \: x \in B\}&&
		\end{flalign*}
	\subsection{Durchschnitt}
		\begin{flalign*}
			\begin{tikzpicture}
			\begin{scope}
				\draw [clip](0,0) circle (2cm);
				\fill[yellow] (3,0) circle (2cm);
			\end{scope}
			\draw (0,0) circle (2cm);
			\draw (3,0) circle (2cm);
			\draw (0,0) node {A};
			\draw (3.2,0) node {B};
			\end{tikzpicture}&&
		\end{flalign*}
		\begin{flalign*}
			A \cap B := \{x \mid x \in A \: und \: x \in B\}&&
		\end{flalign*}
	\subsection{Differenz}
		\begin{flalign*}
			\begin{tikzpicture}
			\begin{scope}
				\fill[yellow] (0,0) circle (2cm);
				\clip (3,0) circle(2cm);
				\fill[white] (3,0) circle (2cm);
			\end{scope}
			\draw (0,0) circle (2cm);
			\draw (3,0) circle (2cm);
			\draw (0,0) node {A};
			\draw (3,0) node {B};
			\end{tikzpicture}&&
		\end{flalign*}
		\begin{flalign*}
			A \setminus B := \{x \mid x \in A \: und \: x \notin B\}&&
		\end{flalign*}
	\subsection{Symmetrische Differenz}
		\begin{flalign*}
			\begin{tikzpicture}
				\fill[yellow] (0,0) circle (2cm);
				\fill[yellow] (3,0) circle (2cm);
				\begin{scope}
					\draw [clip](0,0) circle (2cm);
					\fill[white] (3,0) circle (2cm);
				\end{scope}
				\draw (0,0) circle (2cm);
				\draw (3,0) circle (2cm);
				\draw (0,0) node {A};
				\draw (3,0) node {B};
				\end{tikzpicture}&&
			\end{flalign*}
		\begin{flalign*}
			A \triangle B := \{ x \mid (x \in A) \: \veebar \: ( x \in B ) \}&&
			A \triangle B := \{ x \mid (x \in A) \: \nleftrightarrow \: ( x \in B ) \}&&
		\end{flalign*}
	\subsection{Definierte Zahlenmengen}
	Natürliche Zahlen
		\begin{flalign*}
			\mathbb{N} = \{ 1; 2; 3; ... \}&&
		\end{flalign*}\newline
	Menge der Natürliche Zahlen
		\begin{flalign*}
			\mathbb{N}_0 = \{ 0; 1; 2; 3; ... \}&&
		\end{flalign*}\newline
	Ganze Zahlen
		\begin{flalign*}
			\mathbb{Z} = \{ ... ; -2; -1; 0; 1; 2; 3; ... \}&&
		\end{flalign*}\newline
	Rationale Zahlen
		\begin{flalign*}
			\mathbb{Q} = \left\{ \frac{p}{q} \mid p, q \in \mathbb{Z}, q \neq 0 \right\}&&
		\end{flalign*}\newline
		Reelle Zahlen\newline
		Die reellen Zahlen umfassen die rationalen Zahlen und die irrationalen Zahlen.\newline\newline
		Irrationale Zahlen
		\begin{flalign*}
			\mathbb{R} \setminus \mathbb{Q}&&
		\end{flalign*}
\section{Elementare Rechengesetze, -verfahren und -notationen}
\label{sec:rechengesetze}
	\subsection{Brüche dividieren}
		Um zwei Brüche zu dividieren bildet man den Kehrwert vom Divisor und multipliziert diesen mit dem Dividend.
		\begin{flalign*}
		\frac{p_1}{q_1} : \frac{p_2}{q_2} = \frac{p_1}{q_1} \cdot \frac{q_2}{p_2}&&
		\end{flalign*}
		\begin{flalign*}
		\bigfrac {\frac{p_1}{q_1}}  {\frac{p_2}{q_2}} = \frac{p_1}{q_1} \cdot \frac{q_2}{p_2}&&
		\end{flalign*}
	\subsection{Lösungsmenge}
		Beispiel 1 ($x^2=-1$):\newline
		$\mathbb{L} = \emptyset$ \newline\newline
		Beispiel 2 ($x^2 = 4$):\newline
		$\mathbb{L} = \{-2;2\}$\newline\newline
		Beispiel 3 ($sin(x) = 0$):\newline
		$\mathbb{L} = \{...;-2\pi;-\pi;0;\pi;2\pi; ...\}$\newline\newline
		Beispiel 4 ($x^2 + y = 5$):\newline
		$\mathbb{L} = \{(x_0;y_0)\in\mathbb{R}^2\mid x^2_0 + y_0 = 5\} = \{(x_0;5-x^2_0) \mid x_0\in\mathbb{R}^2\}$\newline\newline
		In diesem Fall ist die Lösungsmenge die Funktion $y=5-x^2$.\newline
		\makeplot{{5-x^2}}{{$f(x)=5-x^2$}}{{11,3.3}}{-10,10}{-10,10}{-30:30}{330}{17cm,7cm}{smooth}
	\subsection{Normalform}
		Eine Gleichung in der Form $ax^2+bx+c=0$ mit $a\neq0$ und $b,c\in \mathbb{R}$, heißt quadratisch. Spezial bezeichnet man $x^2+px+q=0$ mir $p,q\in \mathbb{R}$, als quadratische Gleichung in Normalform.\newline\newline
		Man kann eine quadratische Gleichung in die Normalform überführen, indem man durch $a$ teilt: $x^2+\frac{b}{a}x+\frac{c}{a}=0$.
		\subsubsection{p-q-Formel}
		\label{subsubsec:pqformel}
		\begin{tcolorbox}[boxsep=0pt,top=.75cm,left=1cm,right=1cm, bottom=.5cm,arc=0pt,auto outer arc,colback=white,colframe=black, enlarge top by=.25cm, enlarge bottom by=.25cm]
			Um die Nullstellen einer quadratischen Gleichung in der Normalform zu finden, kann man die p-q-Formel benutzen: $x_{\pm}=-\frac{p}{2}\pm\sqrt{\left(\frac{p}{2}\right)^2-q}$.
		\end{tcolorbox}
		\noindent$D=\left(\frac{p}{2}\right)^2-q$ ist die Diskriminante. Sie gibt Aufschluss über die Lösungsmenge.
		\begin{flalign*}
		&D>0\Rightarrow Es\;gibt\;zwei\;Lsg.\\
		&D=0\Rightarrow Es\;gibt\;eine\;Lsg.\\
		&D<0\Rightarrow Es\;gibt\;keine\;Lsg.&&
		\end{flalign*}
	\subsection{Intervalle}
		\label{subsec:intervalle}
		\subsubsection{Abgeschlossene Intervalle}
		$[a;b] := \{ x \in \mathbb{R} \mid a \le x \le b \}$
		\subsubsection{Offene Intervalle}
		$(a;b) = \: ]a;b[ \: := \{ x \in \mathbb{R} \mid a < x < b \}$
		\subsubsection{Halboffene Intervalle}
		Rechtsoffen \newline
		$[a;b) = \: [a;b[ \: :=  \{ x \in \mathbb{R} \mid a \le x < b \}$ \newline\newline
		Linksoffen \newline
		$(a;b] = \: ]a;b] \: :=  \{ x \in \mathbb{R} \mid a < x \le b \}$
	\subsection{Beträge}
		\begin{flalign*}
		\vert a \vert = \left\{ a \atop -a \right. \;\;&
		a \ge 0 \atop a < 0 &
		\end{flalign*}
		\begin{flalign*}
		\vert -a \vert = \vert a \vert&&
		\end{flalign*}
	\subsection{Binomische Formeln}
	\label{sec:binomischeformeln}
	$(a+b)^2 = a^2 + 2ab + b^2$ \newline\newline
	$(a-b)^2 = a^2 - 2ab + b^2$ \newline\newline
	$(a+b)(a-b) = a^2 - b^2$
	\subsection{Euklidischer Algorithmus}
		Der euklidische Algorithmus findet den größten gemeinsamen Teiler zweier Zahlen. Das eignet sich ausgezeichnet dazu, Brüche zu kürzen. Der vorletzte Rest bevor $R = 0$ eintritt, ist das Ergebnis.
		\begin{flalign*}
		& 2160 : 2592 = 0 \;\;\; R = 2160 \\
		& 2592 : 2160 = 1 \;\;\; R = 432 \\
		& 2160 : 432 = 5 \;\;\; R = 0 &&
		\end{flalign*}
		\begin{flalign*}
		\frac{2592}{2160} = \frac{6 \cdot 432}{5 \cdot 432} = \frac{6}{5}&&
		\end{flalign*}
	\subsection{Potenzgesetze}
	\label{subsec:potenzgesetze}
		\begin{tcolorbox}[boxsep=0pt,top=.35cm,left=1cm,right=1cm, bottom=.75cm,arc=0pt,auto outer arc,colback=white,colframe=black, enlarge top by=.25cm, enlarge bottom by=.25cm]
			\begin{flalign*}
				& a^k \cdot a^m = a^{k+m} \\\\
				& \cfrac{b^k}{b^m} = b^{k-m} \\\\
				& a^k \cdot b^k = (a \cdot b)^k \\\\
				& \cfrac{a^k}{b^k} = \left( \frac{a}{b} \right)^k \\\\
				& (a^k)^m = a^{k \cdot m} && 
			\end{flalign*}
		\end{tcolorbox}
		\noindent Für $a>0$ und jede rationale Zahl $\frac{p}{q}$ (mit $p,q \in \mathbb{Z}$ und $q>0$) ist
		\begin{flalign*}
		a^{\frac{p}{q}} = \sqrt[q]{a^p} = (\sqrt[q]{a})^p && 
		\end{flalign*}
		Beispiel: Bestimmen Sie $m$ und $n$ so, dass gilt: $(9x^7)^2 = mx^n$
		\begin{flalign*}
			(9x^7)^2 = mx^n \\
			81x^{14} = mx^n &&
		\end{flalign*}
		$m = 81$ und $n = 14$
	\subsection{Wurzelgesetze}
		\begin{tcolorbox}[boxsep=0pt,top=.75cm,left=1cm,right=1cm, bottom=.7cm,arc=0pt,auto outer arc,colback=white,colframe=black, enlarge top by=.25cm, enlarge bottom by=.25cm]
			Für $a,b,c \in \mathbb{R}$ mit $a,b \ge 0, c > 0$ und $m, n \in \mathbb{N}$ gilt
			\begin{flalign*}
				& \sqrt[n]{ab} = \sqrt[n]{a} \cdot \sqrt[n]{b} \\\\
				& \sqrt[n]{\frac{a}{c}} = \frac{\sqrt[n]{a}}{\sqrt[n]{c}} \\\\
				& \sqrt[n]{\sqrt[m]{a}} = \sqrt[n \cdot m]{a} && 
			\end{flalign*}
		\end{tcolorbox}
		\noindent Beispiel 1: Nach der dritten Binomischen Formel gilt für $a,b > 0, a \neq b$:
		\begin{flalign*}
		& \cfrac{1}{\sqrt{a}+\sqrt{b}} & \mid \cdot (\sqrt{a}-\sqrt{b}) &&&&&&&&&&&& \\
		= & \cfrac{\sqrt{a}-\sqrt{b}}{(\sqrt{a}+\sqrt{b})(\sqrt{a}-\sqrt{b})} \\ 
		= & \cfrac{\sqrt{a}-\sqrt{b}}{\sqrt{a}^2-\sqrt{b}^2} \\ 
		= & \cfrac{\sqrt{a}-\sqrt{b}}{a-b} &&
		\end{flalign*}
		\newline\newline\hrule
		Beispiel 2:
		\begin{flalign*}
		& \frac{\sqrt{(1+a^2) \cdot (a-b)^2}}{\sqrt[4]{16(1+a^2)^2}} \\
		= & \sqrt{\frac{(1+a)^2 \cdot (a-b)^2}{\sqrt{16(1+a^2)^2}}} \\ 
		= & \sqrt{\frac{(1+a)^2 \cdot (a-b)^2}{4(1+a)^2}} \\ 
		= & \frac{1}{2}\sqrt{(a-b)^2} \\ 
		= & \frac{\mid a-b \mid}{2} &&
		\end{flalign*}
		\subsubsection{Wurzeltherme vereinfachen (Beispiele)}
			\begin{flalign*}
				\sqrt{2}+\dfrac{2}{2\sqrt{2}+3} & = \dfrac{}{}\sqrt{2}+\dfrac{ 2\cdot ( 2\sqrt{2}-] ) }{\left( 2\sqrt{2}+3\right) \left( 2\sqrt{2}-3\right) } \\
				& = \sqrt{2}+\dfrac{4\sqrt{2}-6}{\left( 2\sqrt{2}\right) ^{2}-3^{2}} \\
				& = \sqrt{2}+\dfrac{4\sqrt{2}-6}{-1} \\
				& = \sqrt{2}-4\sqrt{2}+6 \\
				& = 6-3\sqrt{2} &&
			\end{flalign*}
			\hrule
			\begin{flalign*}
			\dfrac{1}{\sqrt{1+x^{2}}-1}-\dfrac{1}{\sqrt{1+x^{2}}+1} & = \dfrac{\sqrt{1+x^{2}}+1}{\left( \sqrt{1+x^{2}}-1\right) \cdot \left( \sqrt{1+x^{2}}+1\right) } - \dfrac{\sqrt{1+x^{2}}-1}{\left( \sqrt{1+x^{2}}+1\right) \cdot \left( \sqrt{1+x^{2}}-1\right) } \\
			& = \dfrac{\left( \sqrt{1+x^{2}}+1\right) -\left( \sqrt{1+x^{2}}-1\right) }{1+x^{2}-1} \\
			& = \frac{2}{x^2} &&
			\end{flalign*}
			\newline\newline\hrule
			Beispiel 3: Bestimmen Sie $x$ und $y$, sodass $\frac{x}{y}$ vollständig gekürzt ist.
			\begin{flalign*}
				\dfrac{2\cdot 2^{\frac{5}{2}}}{2^{\frac{1}{4}}}=2^{\frac{x}{y}} \\
				2 \cdot \dfrac{2^{\frac{5}{2}}}{2^{\frac{1}{4}}}=2^{\frac{x}{y}} \\
				2 \cdot 2^{\frac{5}{2}-\frac{1}{4}} = 2^{\frac{x}{y}} \\
				2 \cdot 2^{\frac{9}{4}} = 2^{\frac{x}{y}} \\
				2^{\frac{13}{4}} = 2^{\frac{x}{y}} &&
			\end{flalign*}
			Damit gilt $x = 13$ und $y = 4$.
	\subsection{Logarithmusgesetze}
	\label{subsec:logarithmusgesetze}
		\begin{tcolorbox}[boxsep=0pt,top=1cm,left=1cm,right=1cm, bottom=1cm,arc=0pt,auto outer arc,colback=white,colframe=black, enlarge top by=.25cm, enlarge bottom by=.25cm]
			Die Logarithmusrechnung dient dazu das $x$ im Term $b^x=a$ zu bestimmen. Es ist damit quasi das Gegenstück zur Potenzrechnung. Rechnen wir z.~B. $9^3$, kommen wir auf $729$. Umgekehrt können wir jetzt aber auch $log_{9}729$ rechnen und kommen auf $3$. Es gibt außerdem die speziellen Notationen $ln$ und $lg$, die jeweils \textit{natürlicher Logarithmus} und \textit{dekadischer Logarithmus} genannt werden.
			\begin{flalign*}
				ln(b):=log_e(b)&&
			\end{flalign*}
			\begin{flalign*}
				lg(b):=log_{10}(b)&&
			\end{flalign*}
			\begin{flalign*}
				log_b(b)=1&&
			\end{flalign*}
			\begin{flalign*}
				log_b(1)=0&&
			\end{flalign*}
			\begin{flalign*}
				log_b(u\cdot v)=log_b(u)+log_b(v)&&
			\end{flalign*}
			\begin{flalign*}
				log_b\left(\frac{u}{v}\right)=log_b(u)-log_b(v)&&
			\end{flalign*}
			\begin{flalign*}
				log_b(u^v)=v\cdot log_b(u)&&
			\end{flalign*}
			\begin{flalign*}
				log_b(\sqrt[u]{v})=\frac{log_b(v)}{u}&&
			\end{flalign*}
			\begin{flalign*}
				log_a(v)=\frac{log_b(v)}{log_b(a)}&&
			\end{flalign*}
		\end{tcolorbox}
\section{Vereinfachungen zum Lösen von Gleichungen}
	\label{sec:gleichungenvereinfachen}
	\subsection{Quadratische Ergänzung}
		\begin{tcolorbox}[boxsep=0pt,top=.75cm,left=1cm,right=1cm, bottom=.65cm,arc=0pt,auto outer arc,colback=white,colframe=black, enlarge top by=.25cm, enlarge bottom by=.25cm]
			Die äquivalente Umformung der quadratischen Gleichung in Normalform $x^2+px+q=0$ in $\left(x+\frac{p}{2}\right)^2=-q+\left(\frac{p}{2}\right)^2$ wird als quadratische Ergänzung bezeichnet. In anderen Worten fügt man den Term $+\left(\frac{p}{2}\right)^2-\left(\frac{p}{2}\right)^2$ hinzu.
		\end{tcolorbox}
		\noindent Beispiel:
			\begin{flalign*}
		x^2+8x+7 &= 0\\
		x^2+8x+\left(\frac{8}{2}\right)^2-\left(\frac{8}{2}\right)^2+7 &= 0\\
		x^2+8x+4^2-4^2+7 &= 0\\
		(x+4)^2-4^2+7 &= 0\\
		(x+4)^2-9 &= 0\;\;\;\;\;\;\;\;\;\mid+9\\
		(x+4)^2 &= 9\;\;\;\;\;\;\;\;\;\mid\sqrt{\ }\\
		x&=\pm\sqrt{9}-4\\
		\mathbb{L}&=\{-1;-7\}&&
		\end{flalign*}
	\subsection{Faktorisieren}
		Um die Nullstellen eines Terms zu finden, bietet es sich an, ihn als Produkt einfacher Terme zu schreiben, denn ist ein Faktor $0$, ist das Produkt ebenfalls $0$. Den Term in so eine Form zu überführen, nennt sich Faktorisieren.
		\subsubsection{Faktorisierung durch Ausklammern}
		\label{subsubsec:ausklammern}
			Beispiel:
			\begin{flalign*}
				x^4+2x^3+3x^2 &= 0\\
				x^2(x^2+2x+3) &= 0\\
				x^2 &= 0\;oder\;(x^2+2x+3) = 0\\
				\mathbb{L} &= \{0\}&&
			\end{flalign*}
			Für $x^2+2x+3 = 0$ existiert keine reelle Lösung $\Rightarrow$ \highlight{subsubsec:pqformel}{p-q-Formel}.
		\subsubsection{Faktorisierung mit binomischen Formeln}
			Beispiel:
			\begin{flalign*}
			9x^2+30x+25& = 0\\
			(3x+5)^2& = 0\\
			3x+5&=0\;\;\;\;\;\;\;\;\;\mid-5\\
			3x&=-5\;\;\;\;\;\;\;\;\;\mid:3\\
			x&=-\frac{5}{3}\\
			\mathbb{L}&=\left\{-\frac{5}{3}\right\}&&
			\end{flalign*}
		\subsubsection{Faktorisierung mit dem Satz von Viëta}
			\begin{tcolorbox}[boxsep=0pt,top=.75cm,left=1cm,right=1cm, bottom=.75cm,arc=0pt,auto outer arc,colback=white,colframe=black, enlarge top by=.25cm, enlarge bottom by=.25cm]
			Der Satz von Viëta besagt, dass $x^2+px+q=(x-x_1)\cdot(x-x_2)$ ist. $p$ und $q$ lassen sich auf die Nullstellen zurückführen: $x_1+x_2=-p$ und $x_1\cdot x_2 = q$.
			\end{tcolorbox}
			\noindent Daraus lässt sich $x_2=\frac{q}{x_1}$ ableiten. Wenn man also durch Raten eine Nullstelle findet, kann man so die andere Nullstelle auch ganz einfach finden.\newline\newline
			Beispiel (eine Nullstelle ist $1$, die andere ergibt sich als $\sqrt{2} = \frac{\sqrt{2}}{1}$):
			\begin{flalign*}
				x^2+(\sqrt{2}-1)x-\sqrt{2}&=0\\
				(x-1)\cdot (x+\sqrt{2})&=0\\
				\mathbb{L}=\{1;-\sqrt{2}\}&&
			\end{flalign*}
	\subsection{Substitution}
	\label{subsec:substitution}
		Substitution erlaubt es uns manchmal Gleichungen zu vereinfachen, um leichter mit ihnen rechnen zu können.\newline\newline
		Beispiel: $x^8-15x^4-16=0$\newline
		Hier bietet es sich an $x^4$ durch $u$ zu ersetzen.
		\begin{flalign*}
		u^2-15u-16&=0\\
		u_{\pm}&=\frac{15}{2}\pm\sqrt{\left(\frac{-15}{2}\right)^2+6}&&
		\end{flalign*}
		Diesen Term wiederum können wir ganz einfach mit der \highlight{subsubsec:pqformel}{p-q-Formel} lösen. Dabei erhalten wir $u_+=16$ und $u_-=-1$. Um unsere endgültige Lösungsmenge zu bekommen, müssen wir noch resubstituieren.
		\begin{flalign*}
		x^4&=u_+=16\\
		x_1&=2\\
		x_2&=-2&&
		\end{flalign*}
		Da es kein $x$ gibt, das $x^4=-1$ erfüllt, haben wir bereits unsere komplette Lösungsmenge:\newline\newline$\mathbb{L}=\{-2;2\}$.
\section{Ungleichungen}
	\subsection{Rechenregeln}
	\label{subsec:unglrechrgl}
		Wenn man ein Ungleichung mit einer negativen Zahl multipliziert oder durch diese teilt, muss das Vergleichszeichen umgekehrt werden.
		\begin{tcolorbox}[boxsep=0pt,top=.75cm,left=1cm,right=1cm, bottom=.65cm,arc=0pt,auto outer arc,colback=white,colframe=black, enlarge top by=.25cm, enlarge bottom by=.25cm]
			Für $c<0$ gilt:
			\begin{flalign*}
				&a<b\iff c\cdot  a>c\cdot b\\
				&a<b\iff \frac{a}{c}>\frac{b}{c} &&
			\end{flalign*}
		\end{tcolorbox}
	\subsection{Quadratische Ungleichungen}
		Um die Lösungsmenge einer quadratischen Ungleichung zu finden, formt man die Ungleichung zunächst so um, dass auf einer Seite $0$ steht. Auf der anderen Seite hat man dann optimalerweise eine quadratische Funktion. Schauen wir uns mal das Beispiel $x^2>2x+7$ an.
		\makeplot{{x^2-2*x-7}}{{$f(x)=x^2-2x-7$}}{{5,1}}{-10,10}{-10,10}{-30:30}{330}{17cm,7cm}{smooth}
		Wir stellen also zunächst um und erhalten $x^2-2x-7>0$. Daraus ergibt sich auch die Funktion oben. An der Grafik erkennt man sehr gut, was wir eigentlich suchen. Denn unsere Lösungsmenge sind alle $x$, für die $f(x)$ größer als $0$ ist. Und wie kriegen wir das raus? Indem wir die Nullstellen berechnen. Das Intervall von Unendlich bis zur linken Nullstelle ist ein Teil unserer Lösung und der andere ist das Intervall von der rechten Nullstelle bis unendlich. Dabei muss man stets verschiedene Fälle beachten. Für eine nach unten geöffnete Funktion ($-x^2$) suchen wir den Bereich zwischen den Nullstellen. Für eine Funktion oberhalb der $x$-Achse, die keine Nullstellen hat, sind alle reellen Zahlen unsere Lösungsmenge, wohingegen eine Funktion ohne Nullstellen unterhalb der $x$-Achse eine leere Lösungsmenge liefern würde. Eine Funktion mit genau einer Nullstelle liefert hingegen eine Lösungsmenge aller reellen Zahlen außer der Nullstelle. Es gibt je nach Art der Funktion und Vergleichszeichen in unserer Ungleichung viele unterschiedliche Szenarien, weshalb es immer ratsam ist eine Skizze anzufertigen. Für das Beispiel oben können wir die \highlight{subsubsec:pqformel}{p-q-Formel} verwenden, um die Nullstellen zu berechnen.
		\begin{flalign*}
		x_{\pm}&=-\frac{-2}{2}\pm \sqrt{\left(\frac{-2}{2}\right)^2+7}\\
		x_1&=1-2\sqrt{2}\\
		x_2&=1+2\sqrt{2}&&
		\end{flalign*}
		Jetzt, wo wir die Nullstellen haben, ist es nicht schwer die Lösungsmenge anzugeben. Dabei sollte man darauf achten, dass man abgeschlossene und offene \highlight{subsec:intervalle}{Intervalle} nicht verwechselt.
		\begin{flalign*}
			\mathbb{L}=\mathbb{R}\setminus \left[1-2\sqrt{2};1+2\sqrt{2}\right]=\left(-\infty;1-2\sqrt{2}\right)\cup\left(1+2\sqrt{2}\right)&&
		\end{flalign*}
	\subsection{Ungleichungen mit Beträgen}
		Das Vorgehen bei Betragsungleichungen ist im Grunde genommen dasselbe Prinzip, wie bei den quadratischen. Schauen wir uns das Beispiel $\vert x \vert -3 < 0$ an.
		\makeplot{{abs(x)-3}}{{$f(x)=\vert x \vert -3$}}{{5,1.75}}{-10,10}{-10,10}{-30:30}{330}{17cm,7cm}{sharp plot}
		Wir erkennen die Nullstellen in dem Fall sehr leicht. Das sind $-3$ und $3$. Erkennt man das nicht sofort, muss man eine \highlight{subsec:betragsfunktionen}{Fallunterscheidung} durchführen. Jetzt können wir aber erst mal unsere Lösungsmenge definieren, denn wir wissen, dass wir alle $x$ suchen für die $f(x)<0$ gilt.
		\begin{flalign*}
		\mathbb{L}=(-3;3)&&
		\end{flalign*}
		Hinweis: Wäre unsere Ausgangsungleichung $\vert x \vert -3 \le 0$, sehe unsere Lösungsmenge jetzt so aus:
		\begin{flalign*}
		\mathbb{L}=[-3;3]&&
		\end{flalign*}
	\subsection{Ungleichungen mit Variable im Nenner – Teil I}
		Aus den vorherigen Erklärungen kann man sich herleiten, wie man das macht. Deshalb ist hier nur noch mal ein erklärendes Beispiel: $2\le \frac{14}{\vert 2x+5\vert}$.
		\makeplot{{(14/abs(2*x+5))-2}}{{$f(x)=\dfrac{14}{\vert 2x+5\vert}-2$}}{{10,3.5}}{-10,10}{-10,10}{-30:30}{200}{17cm,7cm}{sharp plot}
		Wichtig ist, dass wir zunächst alle $x$ ausschließen, für die im Nenner $0$ rauskommt. In diesem Fall ist dass $-\frac{5}{2}$.
		\begin{flalign*}
			2&\le \frac{14}{\vert 2x+5\vert}\\
			2\vert 2x+5\vert&\le 14\\
			\vert 2x+5\vert&\le 7&&
		\end{flalign*}
		Fall 1: $2x+5>0$
		\begin{flalign*}
		2x+5&\le 7\\
		2x&\le 2\\
		x&\le 1&&
		\end{flalign*}
		Fall 2: $2x+5<0$
		\begin{flalign*}
		-2x-5&\le 7\\
		-2x&\le 12\\
		x&\ge -6&&
		\end{flalign*}
		\begin{flalign*}
		\mathbb{L}=\left[-6;-\frac{5}{2}\right)\cup\left(-\frac{5}{2};1\right]&&
		\end{flalign*}
	\subsection{Ungleichungen mit Variable im Nenner – Teil II}
		Wenn wir uns an die \highlight{subsec:unglrechrgl}{Rechenregeln} für Ungleichungen erinnern, könnte man sich fragen, was passiert, wenn der Nenner mit einer Variable sowohl positiv als auch negativ sein kann. Denn wenn wir mit einer negativen Zahl multiplizieren, müssten wir das Vorzeichen umkehren. Hier muss man wieder verschiedene Fälle unterscheiden.\newline\newline
		Beispiel: $\frac{1}{x-2}\le-x$
		\makeplot{{(1/(x-2))+x}}{{$f(x)=\dfrac{1}{x-2}+x$}}{{11.1,0.65}}{-10,10}{-10,10}{-20:20}{350}{17cm,7cm}{sharp plot}
		Der Fall $x=2$ ist aufgrund des $x$ im Nenner wieder auszuschließen.
		Fall 1: $x>2$
		\begin{flalign*}
			\frac{1}{x-2}&\le-x\\
			1&\le -x(x-2)\\
			1&\le -x^2+2x\\
			x^2-2x+1&\le 0\\
			(x-1)^2&\le 0&&
		\end{flalign*}
		Dieser Fall gilt für $x=1$. Das widerspricht allerdings der Bedingung $x>2$ und das Ergebnis ist entsprechend nicht Teil unserer Lösungsmenge.\newline\newline
		Fall 2: $x<2$
		\begin{flalign*}
			\frac{1}{x-2}&\le-x\\
			1&\ge -x(x-2)\\
			1&\ge -x^2+2x\\
			x^2-2x+1&\ge 0\\
			(x-1)^2&\ge 0&&
		\end{flalign*}
		Dieser Fall ist für alle $x$ erfüllt, daher gehören alle $x<2$ zur Lösungsmenge.
		\begin{flalign*}
			\mathbb{L}=(-\infty;2)&&
		\end{flalign*}\newline\newline\hrule
		Beispiel: $\frac{1}{x-9}\le8$
		\makeplot{{(1/(x-9))-8}}{{$f(x)=\dfrac{1}{x-9}-8$}}{{11,1.6}}{-2,16}{-20,5}{-30:60}{330}{17cm,7cm}{smooth}
		Fall 1: $x>9$
		\begin{flalign*}
			\frac{1}{x-9}&\le 8\\
			1&\le 8x-72\\
			8x&\ge 73\\
			x&\ge \frac{73}{8}&&
		\end{flalign*}
		Fall 2: $x<9$
		\begin{flalign*}
			\frac{1}{x-9}&\le 8\\
			1&\ge 8x-72\\
			8x&\le 73\\
			x&\le \frac{73}{8}&&
		\end{flalign*}
		\begin{flalign*}
			\mathbb{L}=(-\infty;9)\cup \left[\frac{73}{8};\infty\right)&&
		\end{flalign*}
	\section{Lineare Gleichungssysteme}
		Ein Gleichungssystem ist eine Sammlung an Gleichungen, für die man eine gemeinsame Lösung sucht. Für das Beispiel unten, ist die Lösung $x=2,y=3,z=-4$ oder anders ausgedrückt $\mathbb{L}=\{(2;3;-4)\}$. Wie man darauf kommt, wird unten erklärt.
		\begin{flalign*}
			(I)&&x+2y+z&=4\\
			(II)&&x-y+ \frac{3}{2}z&=-7\\
			(III)&&-4x+2y&=-2&&&&&&&&
		\end{flalign*}
		\subsection{Einsetzungsverfahren}
			Eine Möglichkeit hat man, wenn man eine Funktion nach einer beliebigen Variable umstellt und diese dann in einer anderen Funktion einsetzt.\newline\newline
			$(III)$
			\begin{flalign*}
				-4x+2y&=-2\\
				-4x&=-2-2y\\
				x&=\frac{1}{2}+\frac{1}{2}y&&
			\end{flalign*}
			$(I)$
			\begin{flalign*}
				x+2y+z&=4\\
				\frac{1}{2}+\frac{1}{2}y+2y+z&=4\\
				\frac{1}{2}+\frac{5}{2}y+z&=4\\
				z&=3,5-\frac{5}{2}y&&
			\end{flalign*}
			$(II)$
			\begin{flalign*}
				x-y+ \frac{3}{2}z&=-7\\
				\frac{1}{2}+\frac{1}{2}y-y+ \frac{3}{2}\left(3,5-\frac{5}{2}y\right)&=-7\\
				\frac{1}{2}+\frac{1}{2}y-y+ 5,25-\frac{15}{4}y&=-7\\
				5,75-\frac{1}{2}y-\frac{15}{4}y&=-7\\
				-\frac{17}{4}y&=-\frac{51}{4}\\
				y&=3&&
			\end{flalign*}
			$(III)$
			\begin{flalign*}
				-4x+2y&=-2\\
				-4x+2\cdot 3&=-2\\
				-4x+6&=-2\\
				-4x&=-8\\
				x&=2&&
			\end{flalign*}
			$(I)$
			\begin{flalign*}
				x+2y+z&=4\\
				2+2\cdot 3+z&=4\\
				z&=-4&&
			\end{flalign*}
		\subsection{Additionsverfahren}
			Eine andere Möglichkeit ist es, eine oder mehrere Gleichungen mit einer Zahl zu multiplizieren, sodass eine Variable entfällt, wenn man zwei Gleichungen addiert.\newline\newline
			$(I)-(III)$
			\begin{flalign*}
			5x+z&=6\\
			z&=6-5x&&
			\end{flalign*}
			$(I)+2(II)$
			\begin{flalign*}
			3x+4z&=-10\\
			3x+4(6-5x)&=-10\\
			3x+24-20x&=-10\\
			-17x&=-34\\
			x&=2&&
			\end{flalign*}
			$(III)$
			\begin{flalign*}
			-4x+2y&=-2\\
			-4\cdot 2+2y&=-2\\
			-8+2y&=-2\\
			2y&=6\\
			y&=3&&
			\end{flalign*}
			$(I)$
			\begin{flalign*}
			x+2y+z&=4\\
			2+2\cdot 3+z&=4\\
			z&=-4&&
			\end{flalign*}
		Hinweis: sind zwei Gleichungen identisch, so gibt es unendlich viele Lösungsmengen und man muss nur die entsprechende Notation für die Lösungsmenge kennen.
		\begin{flalign*}
			(I)&&-4x-2y&=-14\\
			(II)&&4x+2y&=14&&&&&&&&&&&&
		\end{flalign*}
		\begin{flalign*}
			\mathbb{L}=\{(x;7-2x)\mid x \in \mathbb{R}\}&&
		\end{flalign*}
		\subsection{Gauß-Verfahren}
			Das Gauß-Verfahren ist eine bestimmte Vorgehensweise fürs Additionsverfahrens, bei dem man die Gleichungen so umformt, dass man das LGS in die Stufenform bringt und es einfach lösen kann.
			\begin{flalign*}
			(I)&&x+2y+z&=4\\
			(II)&&x-y+ \frac{3}{2}z&=-7\;\;\;\;\;\;\mid -\frac{3}{2}(I)\\
			(III)&&-4x+2y&=-2&&&&&&&&&&&&
			\end{flalign*}
			\begin{flalign*}
			(I)&&x+2y+z&=4\\
			(II)&&-\frac{1}{2}x-4y&=-13\\
			(III)&&-4x+2y&=-2\;\;\;\;\;\;\mid +\frac{1}{2}(II)&&&&&&&&&&&&
			\end{flalign*}
			\begin{flalign*}
			(I)&&x+2y+z&=4\\
			(II)&&-\frac{1}{2}x-4y&=-13\\
			(III)&&-\frac{17}{4}x&=-\frac{17}{2}&&&&&&&&&&&&
			\end{flalign*}
			$(III)$
			\begin{flalign*}
			-\frac{17}{4}x&=-\frac{17}{2}\\
			x&=2&&
			\end{flalign*}
			$(II)$
			\begin{flalign*}
			-\frac{1}{2}x-4y&=-13\\
			-1-4y&=-13\\
			-4y&=-12\\
			y&=3&&
			\end{flalign*}
			$(I)$
			\begin{flalign*}
			x+2y+z&=4\\
			2+6+z&=4\\
			z&=-4&
			\end{flalign*}
		\subsection{LGS mit Parameter}
			Kommt in einem LGS ein Parameter vor, dann muss man eine Fallunterscheidung vornehmen und den Parameter in die Lösungsmenge mit einbeziehen.
			\begin{flalign*}
			(I)&&x-2y&=0\\
			(II)&&y+\frac{1}{3}z&=-1\\
			(III)&&(a-3)y&=1&&&&&&&&&&&&
			\end{flalign*}
			Wenn $a=3$, dann kommt bei der letzten Gleichung $0=1$ raus. Dadurch können wir schon mal sagen, was die Lösungsmenge für den Fall $a=3$ ist.
			\begin{flalign*}
				\mathbb{L}=\emptyset, falls\;a=3&&
			\end{flalign*}
			Als nächstes schauen wir uns den Fall $a\neq0$ an.\newline\newline
			$(III)$
			\begin{flalign*}
				(a-3)y&=1\\
				y&=\frac{1}{a-3}&&
			\end{flalign*}
			$(II)$
			\begin{flalign*}
				\frac{1}{a-3}+\frac{1}{3}z&=-1\\
				\frac{1}{3}z&=-\frac{1}{a-3}-1\\
				z&=3\left(-\frac{1}{a-3}-1\right)\\
				z&=-\frac{3}{a-3}-\frac{3a-9}{a-3}\\
				z&=\frac{6-3a}{a-3}&&
			\end{flalign*}
			$(I)$
			\begin{flalign*}
				x-2y&=0\\
				x-2\left(\frac{1}{a-3}\right)&=0\\
				x&=\frac{2}{a-3}&&
			\end{flalign*}
			\begin{flalign*}
			\mathbb{L}=\left\{\left(\frac{2}{a-3};\frac{1}{a-3};\frac{6-3a}{a-3}\right)\right\},\;falls\;a\neq 3&&
			\end{flalign*}
	\section{Geometrie}
	\label{sec:geometrie}
		\subsection{Rechtwinklige Dreiecke}
			\begin{center}
				\begin{tikzpicture}[scale=1.25]%,cap=round,>=latex]
					\coordinate [label=left:$B$] (B) at (-2cm,-1.cm);
					\coordinate [label=right:$A$] (A) at (2.2cm,-1.0cm);
					\coordinate [label=above:$C$] (C) at (1cm,1.0cm);
					\coordinate [label=above:$S$] (A-|B) at (0.8cm,-1.0cm);
					\draw (A) -- node[sloped,below] {\;\;\;\;\;\;\;\;\;\;q\;\;\;\;\;\;\;\;\;\;\;\;\;\;\;\;\;\;\;\;p} (B) -- node[sloped,above] {a} (C) -- node[sloped,above] {b} (A);
					\draw[dashed, opacity=.4] (C) -- (C|-A) ;
					\tikzset{/tkzmkangle/mark=none}
					\tkzMarkAngle[size=0.65cm](B,C,A)
					
					\tkzMarkAngle[size=0.8cm](C,A,B)
					\tkzLabelAngle[pos = 0.5](C,A,B){$\alpha$}
					
					\tkzMarkAngle[size=1cm](A,B,C)
					\tkzLabelAngle[pos = 0.7](A,B,C){$\beta$}
				\end{tikzpicture}
				\hspace{2cm}
				\begin{tikzpicture}[scale=1.25]%,cap=round,>=latex]
					\coordinate (B) at (-2cm,-1.cm);
					\coordinate (A) at (2.2cm,-1.0cm);
					\coordinate (C) at (1cm,1.0cm);
					\draw (A) -- node[below] {Hypotenuse} (B) -- node[sloped,above] {Gegenkathete von $\alpha$} (C) -- node[sloped,above] {Ankathete von $\alpha$} (A);
					\tikzset{/tkzmkangle/mark=none}
					\tkzMarkAngle[size=0.65cm](B,C,A)
					\tkzLabelAngle[pos = 0.4](B,C,A){$\gamma$}
					
					\tkzMarkAngle[size=0.8cm](C,A,B)
					\tkzLabelAngle[pos = 0.5](C,A,B){$\alpha$}
					
					\tkzMarkAngle[size=1cm](A,B,C)
					\tkzLabelAngle[pos = 0.7](A,B,C){$\beta$}
				\end{tikzpicture}
			\end{center}
			\begin{tcolorbox}[boxsep=0pt,top=1cm,left=1cm,right=1cm, bottom=.75cm,arc=0pt,auto outer arc,colback=white,colframe=black, enlarge top by=.25cm, enlarge bottom by=.25cm]
				\subsubsection{Kathetensatz}
				Im rechtwinkligen Dreieck ist das Quadrat über einer Kathete flächengleich zu dem Rechteck aus der Hypotenuse und dem der Kathete anliegenden Hypotenusenabschnitt.
				\begin{flalign*}
				b^2=p\cdot c\\
				a^2=q\cdot c&&
				\end{flalign*}
			\end{tcolorbox}
			\begin{tcolorbox}[boxsep=0pt,top=1cm,left=1cm,right=1cm, bottom=.75cm,arc=0pt,auto outer arc,colback=white,colframe=black, enlarge top by=.25cm, enlarge bottom by=.25cm]
				\subsubsection{Höhensatz}
				Im rechtwinkligen Dreieck ist das Quadrat über der Höhe flächengleich zu dem Rechteck aus den beiden Hypotenusenabschnitten.
				\begin{flalign*}
				h^2=p\cdot q&&
				\end{flalign*}
			\end{tcolorbox}
				\begin{tcolorbox}[boxsep=0pt,top=1cm,left=1cm,right=1cm, bottom=.75cm,arc=0pt,auto outer arc,colback=white,colframe=black, enlarge top by=.25cm, enlarge bottom by=.25cm]
				\subsubsection{Sinus, Kosinus und Tangens}
				Sinus, Kosinus und Tangens ordnen einem Winkel im rechtwinkligen Dreieck die Längenverhältnisse der Katheten und Hypotenuse zu.\newline
				\begin{flalign*}
				&\sin(\alpha)=\frac{a}{c}=\frac{Gegenkathete\;von\;\alpha}{Hypotenuse}\\\\
				&\cos(\alpha)=\frac{b}{c}=\frac{Ankathete\;von\;\alpha}{Hypothenuse}\\\\
				&\tan(\alpha)=\frac{a}{b}=\frac{Gegenkathete\;von\;\alpha}{Ankathete\;von\;\alpha}=\frac{\sin(\alpha)}{\cos(\alpha)}\\\\
				&\sin(90°-\alpha)=\cos(\alpha)\\\\
				&\cos(90°-\alpha)=\sin(\alpha)\\\\
				&(\sin(\alpha))^2+(\cos(\alpha))^2=1&&
				\end{flalign*}
			\end{tcolorbox}
		\subsection{Rechnen mit Flächen (Formeln)}
			\begin{tcolorbox}[boxsep=0pt,top=1cm,left=1cm,right=1cm, bottom=.75cm,arc=0pt,auto outer arc,colback=white,colframe=black, enlarge top by=.25cm, enlarge bottom by=.25cm]
				\subsubsection{Dreieck}
				Für ein Dreieck mit der Grundseite $c$ und der Höhe $h_c$ gilt:
				\begin{flalign*}
				&F=\frac{1}{2}\cdot c \cdot h_c&&
				\end{flalign*}
			\end{tcolorbox}
			\begin{tcolorbox}[boxsep=0pt,top=1cm,left=1cm,right=1cm, bottom=.75cm,arc=0pt,auto outer arc,colback=white,colframe=black, enlarge top by=.25cm, enlarge bottom by=.25cm]
				\subsubsection{Kreis}
				Für einen Kreis mit dem Radius $r$, dem Umfang $U$ und der Fläche $F$ gilt:
				\begin{flalign*}
				&U=2\pi r\\\\
				&F=\pi r^2&&
				\end{flalign*}
				Für einen Kreissektor mit dem Radius $r$, der Bogenlänge $b$, der Fläche $F$ und dem Winkel $\alpha$ gilt:
				\begin{flalign*}
				&F=\frac{br}{2}&&
				\end{flalign*}
				Für ein Kreissegment mit dem Radius $r$, der Bogenlänge $b$, der Fläche $F$ und dem Winkel $\alpha$ gilt:
				\begin{flalign*}
				&F=\frac{br}{2}-\frac{1}{2}r^2\cdot \sin(\alpha)&&
				\end{flalign*}
			\end{tcolorbox}
		\subsection{Rechnen mit Körpern (Formeln)}
		\begin{tcolorbox}[boxsep=0pt,top=1cm,left=1cm,right=1cm, bottom=.75cm,arc=0pt,auto outer arc,colback=white,colframe=black, enlarge top by=.25cm, enlarge bottom by=.25cm]
			\subsubsection{Prisma}
			Für ein Prisma mit der Mantelfläche $M$, der Grundfläche $A$, dem Grundflächenumfang $U$, der Oberfläche $O$, dem Volumen $V$ und der Höhe $h$ gilt:
			\begin{flalign*}
			&V=A\cdot h\\\\
			&M=U\cdot h\\\\
			&O=2\cdot A+M&&
			\end{flalign*}
		\end{tcolorbox}
		\begin{tcolorbox}[boxsep=0pt,top=1cm,left=1cm,right=1cm, bottom=.75cm,arc=0pt,auto outer arc,colback=white,colframe=black, enlarge top by=.25cm, enlarge bottom by=.25cm]
			\subsubsection{Pyramide}
			Für eine Pyramide mit der Mantelfläche $M$, der Grundfläche $A$, der Oberfläche $O$, dem Volumen $V$ und der Höhe $h$ gilt:
			\begin{flalign*}
			&V=\frac{1}{3}\cdot A \cdot h\\\\
			&O=2\cdot A+M&&
			\end{flalign*}
		\end{tcolorbox}
		\begin{tcolorbox}[boxsep=0pt,top=1cm,left=1cm,right=1cm, bottom=.75cm,arc=0pt,auto outer arc,colback=white,colframe=black, enlarge top by=.25cm, enlarge bottom by=.25cm]
			\subsubsection{Zylinder}
			Für einen Zylinder mit der Grundfläche $A$, dem Radius der Grundfläche $r$, der Oberfläche $O$, dem Volumen $V$ und der Höhe $h$ gilt:
			\begin{flalign*}
			V=\pi \cdot r^2\cdot h&&
			\end{flalign*}
			Für einen geraden Zylinder gilt außerdem:
			\begin{flalign*}
				O=2\pi r\cdot (r+h)&&
			\end{flalign*}
		\end{tcolorbox}		
		\begin{tcolorbox}[boxsep=0pt,top=1cm,left=1cm,right=1cm, bottom=.75cm,arc=0pt,auto outer arc,colback=white,colframe=black, enlarge top by=.25cm, enlarge bottom by=.25cm]
			\subsubsection{Kegel}
			Für einen Kegel mit der Grundfläche $A$, dem Radius der Grundfläche $r$, der Oberfläche $O$, dem Volumen $V$, dem Abstand der Spitze zu einem Punkt der Kreislinie $s$ und der Höhe $h$ gilt:
			\begin{flalign*}
			V=\frac{1}{3}\pi\cdot r^2 \cdot h&&
			\end{flalign*}
			Für einen geraden Kegel gilt außerdem:
			\begin{flalign*}
			&s=\sqrt{h^2+r^2}\\\\
			&O=\pi r \cdot (r+s)&&
			\end{flalign*}
		\end{tcolorbox}
	\section{Funktionen}
		\subsection{Allgemeines}
			\subsubsection{Monotonie}
			\label{subsubsec:monotonie}
				\begin{tcolorbox}[boxsep=0pt,top=1cm,left=1cm,right=1cm, bottom=.75cm,arc=0pt,auto outer arc,colback=white,colframe=black, enlarge top by=.25cm, enlarge bottom by=.25cm]
						Seien $x_1$ und $x_2$ zwei Argumente einer Funktion, so gelten folgende Definitionen:\newline\newline
						\textbf{Monoton wachsend}
						\begin{flalign*}
						wenn\;x_1\le x_2\; und\; f(x_1)\le f(x_2)&&
						\end{flalign*}
						\textbf{Streng monoton wachsend}
						\begin{flalign*}
						wenn\;x_1< x_2\; und\; f(x_1)< f(x_2)&&
						\end{flalign*}
						\textbf{Monoton fallend}
						\begin{flalign*}
						wenn\;x_1\le x_2\; und\; f(x_1)\ge f(x_2)&&
						\end{flalign*}
						\textbf{Streng monoton fallend}
						\begin{flalign*}
						wenn\;x_1< x_2\; und\; f(x_1)> f(x_2)&&
						\end{flalign*}
				\end{tcolorbox}
				\noindent Des Weiteren kann man, das Monotonieverhalten einer Funktion mithilfe ihrer \highlight{subsec:ableitung}{Ableitung} bestimmen. Ist die Ableitung $f^{\prime}$ einer Funktion größer oder gleich Null, so ist sie monoton wachsend. Ist sie größer als und ungleich Null, ist sie sogar streng monoton wachsend. Dasselbe gilt umgekehrt für monoton fallende Funktion, wenn ihre Ableitung an der untersuchten Stelle negativ ist.\newline\newline
				\textbf{Monoton wachsend}: $f^{\prime}(x)\ge 0$\newline\newline
				\textbf{Streng monoton wachsend}: $f^{\prime}(x)> 0$\newline\newline
				\textbf{Monoton fallend}: $f^{\prime}(x)\le 0$\newline\newline
				\textbf{Streng monoton fallend}: $f^{\prime}(x)< 0$\newline\newline
				\subsubsection{Besondere Stellen}
				Für manche Stellen einer Funktion werden besondere Begriffe benutzt. Hinweis: $D_f$ ist der Definitionsbereich der Funktion und $I$ ein beliebig kleiner offener Intervall, der $x_{max}$ bzw. $x_{min}$ beinhaltet.
				\begin{tcolorbox}[boxsep=0pt,top=1cm,left=1cm,right=1cm, bottom=.75cm,arc=0pt,auto outer arc,colback=white,colframe=black, enlarge top by=.25cm, enlarge bottom by=.25cm]
					\textbf{Globale Maximalstelle}
					\begin{flalign*}
						wenn\;f(x_{max})\ge f(x)\; aller\;x\in D_f&&
					\end{flalign*}
					\textbf{Lokale Maximalstelle}
					\begin{flalign*}
						wenn\;f(x_{max})\ge f(x)\; aller\;x\in D_f\cap I&&
					\end{flalign*}
					\textbf{Globale Minimalstelle}
					\begin{flalign*}
						wenn\;f(x_{max})\le f(x)\; aller\;x\in D_f&&
					\end{flalign*}
					\textbf{Lokale Minimalstelle}
					\begin{flalign*}
						wenn\;f(x_{max})\le f(x)\; aller\;x\in D_f\cap I&&
					\end{flalign*}
					\textbf{Strikte Extrema}\newline\newline
					Ersetzt man bei den obigen Definitionen das $\ge$ bzw. $\le$ durch $>$ bzw. $<$, spricht man von einem strickten Maximum oder Minimum.
				\end{tcolorbox}
				\noindent Hinweis: Maximal- und Minimalstellen werden auch als Extremalstellen bezeichnet.
				\subsubsection{Symmetrie}
				\label{subsubsec:symmetrie}
				Wenn man Funktionen untersucht, schaut man sich auch oft an, wie deren Symmetrie ist. Ist eine Funktion achsensymmetrisch zur $y$-Achse, spricht man von \textit{gerade} und wenn sie punktsymmetrisch zum Nullpunkt ist, von \textit{ungerade}.
				\begin{tcolorbox}[boxsep=0pt,top=1cm,left=1cm,right=1cm, bottom=.75cm,arc=0pt,auto outer arc,colback=white,colframe=black, enlarge top by=.25cm, enlarge bottom by=.25cm]
					\textbf{Gerade}
					\begin{flalign*}
					wenn\;f(-x)=f(x)&&
					\end{flalign*}
					\textbf{Ungerade}
					\begin{flalign*}
					wenn\;f(-x)=-f(x)&&
					\end{flalign*}
				\end{tcolorbox}
				\noindent Achtung: Eine Funktion kann nur gerade oder ungerade sein, wenn ihr Definitionsbereich symmetrisch zur Nullpunkt auf der $x$-Achse ist.
		\subsection{Potenz- und Wurzelfunktionen}
			Potenzfunktionen in der Form $f(x)=x^m$ mit $m\in\mathbb{N}_0$ und $D_f=\mathbb{R}$ heißen \textbf{Monome} (im Gegensatz zu Polynomen). Potenzfunktionen mit der Form $x^{\frac{m}{n}}$ sind \textbf{Wurzelfunktionen}, wenn $n\ge 2$ gilt und der Bruch keine ganze Zahl ist. An der Potenz kann man erkennen, ob eine Funktion gerade ($x^{2n}$) oder ungerade ($x^{2n-1}$) ist. Hier sind einige Beispiele für Graphen von Potenz- und Wurzelfunktionen:\newline
			\makeplot{{x^3},{x^2},{x^(3/2)},{x^(2/3)},{x^(-1/2)},{x^(-2)},{x^(-3)}}{{$f(x)=x^3$},{$f(x)=x^2$},{$f(x)=x^{\frac{3}{2}}$},{$f(x)=x^{\frac{2}{3}}$},{$f(x)=x^{-\frac{1}{2}}$},{$f(x)=x^{-2}$},{$f(x)=x^{-3}$}}{{1,-1},{1,-2},{4,-1},{4,-2},{7,-1},{7,-2},{10,-1}}{-6,6}{-6,6}{-10:10}{300}{17cm,7cm}{smooth}
			\subsubsection{Wurzelgleichungen}
			Bei Wurzelgleichungen wird zuerst der Definitionsbereich bestimmt werden, also die Menge an reellen Zahlen, für die der Radikand positiv oder gleich Null ist. Zur Lösung von Wurzelgleichungen wird die Wurzel auf einer Seite der Gleichung isoliert. Dann werden beide Seiten der Gleichung mit dem Wurzelexponenten (im Falle der Quadratwurzel also mit 2) so lange potenziert, bis alle Wurzeln eliminiert sind. Man bekommt also unter Umständen durch das Quadrieren (das Potenzieren mit einer geraden Zahl ist keine Äquivalenzumformung) neue Lösungen (Scheinlösungen) hinzu, die die ursprüngliche Gleichung nicht hatte. Die Probe ist folglich für Wurzelgleichungen unverzichtbar!\newline\newline
			Beispiel ($\sqrt{2x+1}=x-17$):
			\begin{flalign*}
			2x+1&\ge 0\\
			x&\ge -\frac{1}{2}&&
			\end{flalign*}
			Damit haben wir den Definitionsbereich. Jetzt kann man nach der Lösung suchen.
			\begin{flalign*}
			\sqrt{2x+1}&=x-17\\
			2x+1&=(x-17)^2\\
			2x+1&=x^2-34x+289\\
			x^2-36x+288&=0\\
			x_1&=12\\
			x_2&=24&&
			\end{flalign*}
			Jetzt MUSS man das Ergebnis noch überprüfen, indem man die Werte $x_1$ und $x_2$ in die ursprüngliche Gleichung einsetzt.
			\begin{flalign*}
			\sqrt{2x_1+1}&=x_1-17\\
			\sqrt{2\cdot 12+1}&=12-17\\
			\sqrt{25}&=-5\\
			5&=-5&&
			\end{flalign*}
			Das Einsetzen von $x_1$ liefert keine wahre Aussage und ist somit nicht Teil der Lösungsmenge.
			\begin{flalign*}
			\sqrt{2x_2+1}&=x_2-17\\
			\sqrt{2\cdot 24+1}&=24-17\\
			\sqrt{49}&=7\\
			7&=7&&
			\end{flalign*}
			Da $x_2$ im Definitionsbereich liegt und beim Einsetzen eine wahre Aussage ergibt, ist es in der Lösungsmenge enthalten.
			\begin{flalign*}
			\mathbb{L}=\{24\}&&
			\end{flalign*}
			Übrigens: Wenn man mehrere Wurzeln in der Gleichung hat, muss man den Definitionsbereich für den Radikanden jeder Wurzel bestimmen.
			\makeplot{{sqrt(abs(2*x+1))},{x-17}}{{$f(x)=\sqrt{2x+1}$},{$f(x)=x-17$}}{{6.5,3.6},{8.3,1.4}}{0,30}{0,10}{-60:60}{600}{17cm,7cm}{smooth}
			Mithilfe dieser Grafik kann man das Ergebnis wunderbar visualisieren, denn das Ergebnis ist der $x$-Wert des Schnittpunkts der beiden Funktionen, die man aus der linken und rechten Seite der Wurzelgleichung entnehmen kann.
			\subsubsection{Wurzelgleichungen mit mehreren Wurzeln (Beispiel)}
			\begin{flalign*}
			\sqrt{8x-14}+\sqrt{5x-2}&=\sqrt{27x-36}\\
			(\sqrt{8x-14}+\sqrt{5x-2})^2&=27x-36\\
			8x-14+2\sqrt{(8x-14)(5x-2)}+5x-2&=27x-36\\
			2\sqrt{(8x-14)(5x-2)}&=14x-20\\
			\sqrt{(8x-14)(5x-2)}&=7x-10\\
			40x^2-86x+28&=(7x-10)^2\\
			40x^2-86x+28&=49x^2-140x+100\\
			0&=9x^2-54x+72\\
			0&=x^2-6x+8&&
			\end{flalign*}
			Jetzt kann man die \highlight{subsubsec:pqformel}{p-q-Formel} anwenden und erhält die Lösungsmenge $\mathbb{L}=\{2;4\}$.
		\subsection{Betragsfunktionen}
			\label{subsec:betragsfunktionen}
			Um mit Betragsgleichungen oder auch Betragsfunktionen rechnen zu können muss man mehrere Fälle betrachten. Nämlich einmal den Fall, dass im Betrag ein Wert größer oder gleich $0$ entsteht und einmal den Fall, dass das Ergebnis im Betrag kleiner als Null ist. Betrachten wir einmal ein Beispiel, wo man den Schnittpunkt zwischen $f(x)=\vert x+1\vert$ und $f(x)=x+2$ finden soll.\newline
			\makeplot{{x+2},{abs(x+1)}}{{$f(x)=x+2$},{$f(x)=\vert x+1\vert$}}{{5.5,4},{9.5,3.2}}{-5,5}{-4,4}{-10:10}{100}{17cm,7cm}{sharp plot}
			Zunächst setzen wir unsere Funktionen gleich und erhalten eine Betragsgleichung. Dann betrachten wir die verschiedenen Fälle für den Betrag.
			\begin{flalign*}
			\vert x+1\vert &= \left\{x+1\;\;\;\;\;\;\;\;\;Fall\;x\ge -1  \atop -(x+1)\;\;\;\;Fall\;x<-1 \right.&&
			\end{flalign*}
			Durch die Fallunterscheidung kann man die Betragsstriche weglassen, indem man jeden Fall einzeln betrachtet. Hinterher muss man aber noch überprüfen, ob das Ergebnis der Bedingung für $x$ in dem Fall entspricht.\newline\newline
			Fall $x\ge-1$ ($x+1$ ist positiv):
			\begin{flalign*}
			x+1&=x+2 &\mid-x &&&&&&&&&&\\
			1&=2&&
			\end{flalign*}
			Für den Fall $x\ge-1$ gibt es keine Lösung, also weiter zum nächsten Fall.\newline\newline
			Fall $x<-1$ ($x+1$ ist negativ):
			\begin{flalign*}
			-x-1&=x+2 &\mid+x-2 &&&&&&&&&&\\
			2x&=-3\\
			x&=-\frac{3}{2}&&
			\end{flalign*}
			Damit haben wir unsere Lösungsmenge, denn wir bekommen für den Fall $-(x<-1)$ ein Ergebnis, welches dem Kriterium $x<-1$ entspricht.
			\begin{flalign*}
			\mathbb{L}=\left\{-\frac{3}{2}\right\}&&
			\end{flalign*}
			Durch einsetzen dieser $x$-Koordinate, finden wir auch den dazugehörigen $y$-Wert: $P\left(-\frac{3}{2}\mid\frac{1}{2}\right)$:
			\subsubsection{Betragsgleichungen mit mehreren Beträgen}
			Haben wir mehrere Beträge in unserer Gleichung, haben wir auch mehrere Fälle zu betrachten. Schon wir uns das an einem Beispiel an, indem wir die Schnittpunkte von $f(x)=\vert x+1 \vert + 5$ und $f(x)=\vert 2x-4 \vert$ suchen.
			\makeplot{{abs(x+1)+5},{abs(2*x-4)}}{{$f(x)=\vert x+1 \vert + 5$},{$f(x)=\vert 2x-4 \vert$}}{{6.9,3.6},{10,1.7}}{-4,14}{0,20}{0:20}{300}{17cm,7cm}{sharp plot}
			Zunächst setzen wir die Funktionen wieder gleich.
			\begin{flalign*}
			\vert x+1 \vert + 5=\vert 2x-4 \vert&&
			\end{flalign*}
			Die Fälle müssen wir alle einzeln betrachten. Das heißt, wir haben insgesamt 4 Fälle. Wir schauen uns zunächst die beiden Fälle eines Betrages an und dann innerhalb dieser Fälle betrachten wir die Fälle für den zweiten Betrag.\newline\newline
			1. Fall für $\vert x+1 \vert$: $x\ge-1$ ($x+1$ ist positiv)
			\begin{flalign*}
			x+1+5&=\vert 2x-4\vert\\
			x+6&=\vert 2x-4\vert&&
			\end{flalign*}
			Innerhalb dieses ersten Falles unterscheiden wir jetzt noch einmal für den übrigen Betrag.
			\begin{tcolorbox}[boxsep=0pt, left=2em, top=1em, bottom=1em,right=0cm,arc=0pt,auto outer arc,colback=white,colframe=white]
				1. Fall für $\vert2x-4\vert$: $x\ge2$ ($2x-4$ ist positiv)
				\begin{flalign*}
				x+6&=2x-4\\
				x+10&=2x\\
				10&=x&&
				\end{flalign*}
				Jetzt müssen wir überprüfen, ob $x\ge2$ und $x\ge-1$ für $x=10$ gelten. Das ist der Fall daher haben wir schon mal einen Teil unserer Lösungsmenge. Auf der Grafik kann man auch sehen, dass sich die beiden Graphen dort schneiden.\newline\newline
				2. Fall für $\vert2x-4\vert$: $x<2$ ($2x-4$ ist negativ)
				\begin{flalign*}
				x+6&=-(2x-4)\\
				x+6&=-2x+4\\
				3x+6&=4\\
				3x&=-2\\
				x&=-\frac{2}{3}&&
				\end{flalign*}
				Wir überprüfen jetzt wieder, ob $x<2$ und $x\ge-1$ für $x=-\frac{2}{3}$ gelten. Da das der Fall ist, können wir auch dieses $x$ zu unserer Lösungsmenge hinzufügen.
			\end{tcolorbox}
			\noindent 2. Fall für $\vert x+1 \vert$: $x<-1$ ($x+1$ ist negativ)
			\begin{flalign*}
			-(x+1)+5&=\vert 2x-4\vert\\
			-x+4&=\vert 2x-4\vert&&
			\end{flalign*}
			\begin{tcolorbox}[boxsep=0pt, left=2em, top=1em, bottom=1em,right=0cm,arc=0pt,auto outer arc,colback=white,colframe=white]
				1. Fall für $\vert2x-4\vert$: $x\ge2$ ($2x-4$ ist positiv)\newline\newline
				In diesem Fall müssen wir gar nicht erst versuchen $x$ auszurechnen, denn es gibt keine Zahl, die sowohl $x\ge2$, als auch $x<-1$ erfüllt.\newline\newline
				2. Fall für $\vert2x-4\vert$: $x<2$ ($2x-4$ ist negativ)
				\begin{flalign*}
				-x+4&=-(2x-4)\\
				-x+4&=-2x+4\\
				-x&=-2x\\
				x&=0&&
				\end{flalign*}
				Wir haben jetzt $x=0$ als Lösung, jedoch erfüllt dieses Ergebnis nicht die Bedingung $x<-1$ und ist daher auch nicht in der Lösungsmenge enthalten.
			\end{tcolorbox}
			\noindent Abschließend können wir feststellen, dass unsere Lösungsmenge $\mathbb{L}=\left\{10;-\frac{2}{3}\right\}$ ist. Durch Einsetzen in eine der beiden Funktionen erhalten wir dann unsere Schnittpunkte $P_1(10\mid 16)$ und $P_2\left(-\frac{2}{3}\mid\frac{16}{3}\right)$.
		\subsection{Ganzrationale Funktionen}
			Polynome sind die Summe aus den Vielfachen von Monomen. Eine ganzrationale Funktion oder auch Polynomfunktion genannt mit dem Koeffizienten $a_n$ hat folgende Form:
			\begin{flalign*}
			p(x)=a_nx^n+a_{n-1}x^{n-1}+...+a_1x^1+1_0&&
			\end{flalign*}
			Das Verhalten einer Polynomfunktion hängt für $x\to \infty$ vom Summanden mit der höchsten Potenz und für $x\to 0$ vom Summanden mit der niedrigsten Potenz ab.
			\makeplot{{x^4-x^3-2*x^2},{x^4},{-2*x^2}}{{$f(x)=x^4-x^3-2x^2$},{$f(x)=x^4$},{$f(x)=-2x^2$}}{{6.5,3.2},{7.2,5},{7.4,1}}{-7,3}{-4,4}{-10:10}{400}{17cm,7cm}{smooth}
			\begin{tcolorbox}[boxsep=0pt,top=.85cm,left=1cm,right=1cm, bottom=.85cm,arc=0pt,auto outer arc,colback=white,colframe=black, enlarge top by=.25cm, enlarge bottom by=.25cm]
				\textbf{Nullstellen}\newline\newline
				Polynome n-ten Grades haben maximal n Nullstellen.
				\begin{flalign*}
				p(x)=a_{2k-1}x^{2k-1}+...+a_1x+a_0,\;wenn\;a_{2k-1}\neq 0&&
				\end{flalign*}
				Polynome ungeraden Grades haben mindestens eine Nullstelle.
				\begin{flalign*}
				p(x)=a_{2k}x^{2k}+...+a_2x^2+a_0,\;wenn\;a_{2k}\ge 0\;und\;a_0>0&&
				\end{flalign*}
				Polynome geraden Grades besitzen keine Nullstellen.
			\end{tcolorbox}
			\begin{tcolorbox}[boxsep=0pt,top=.85cm,left=1cm,right=1cm, bottom=.85cm,arc=0pt,auto outer arc,colback=white,colframe=black, enlarge top by=.25cm, enlarge bottom by=.25cm]
				\textbf{Symmetrie}\newline\newline
				Für die Symmetrie der Funktion gilt wie bei Monomen weiterhin, dass bei geraden Potenzen eine gerade Funktion vorliegt und bei ungeraden Potenzen eine ungerade Funktion. Hat ein Polynom jedoch sowohl gerade, wie auch ungerade Exponenten, so kann man beides ausschließen.
			\end{tcolorbox}
		\subsubsection{Lösen durch Substitution}
		In diesem Beispiel werden die Nullstellen der Funktion mithilfe von \highlight{subsec:substitution}{Substitution} und anschließendem Anwenden der \highlight{subsubsec:pqformel}{p-q-Formel} ermittelt.\newline
		\makeplot{{x^4-5*x^2+2}}{{$p(x)=x^4-5x^2+2$}}{{12.5,1.3}}{-6,6}{-6,6}{-30:60}{300}{17cm,7cm}{smooth}
		\begin{flalign*}
			p(x)&=x^4-5x^2+2\\
			0&=x^4-5x^2+2\\
			0&=u^2-5u+2\\
			u_{1,2}&=\frac{5}{2}\pm\sqrt{\left(-\frac{5}{2}\right)^2-2}\\
			u_1&=\frac{5+\sqrt{17}}{2}\\
			u_2&=\frac{5-\sqrt{17}}{2}\\\\
			x_{1,2}^2&=\frac{5+\sqrt{17}}{2}\\
			x_{1,2}^2&=\pm 2,135779205\\\\
			x_{3,4}^2&=\frac{5-\sqrt{17}}{2}\\
			x_{3,4}^2&=0,6621534469\\\\
			\mathbb{L}&=\{-2,135779205;-0,6621534469;0,6621534469;2,135779205\}&&
		\end{flalign*}
		\hrule
		\subsubsection{Lösen durch Faktorisierung}
		In diesem Beispiel werden die Nullstellen der Funktion mithilfe von \highlight{subsubsec:ausklammern}{Faktorisierung durch Ausklammern} ermittelt.\newline
		\makeplot{{x^5-3*x^3}}{{$p(x)=x^5-3x^3$}}{{11.75,1.3}}{-6,6}{-6,6}{-30:60}{300}{17cm,7cm}{smooth}
		\begin{flalign*}
		p(x)&=x^5-3x^3\\
		0&=x^5-3x^3\\
		0&=x^2(x^2-3)\\
		0&=x^2(x^2-\sqrt{3})(x^2+\sqrt{3})\\
		\mathbb{L}&=\{-\sqrt{3};0;\sqrt{3}\}&&
		\end{flalign*}
		\hrule
		\subsubsection{Lösen mit binomischen Formeln}
		In diesem Beispiel wird Funktion mithilfe der \highlight{sec:binomischeformeln}{binomischen Formeln} so vereinfacht, dass man die Nullstellen ganz einfach ablesen kann.\newline
		\makeplot{{x^4-2*x^2+1}}{{$p(x)=x^4-2x^2+1$}}{{11.6,3.2}}{-6,6}{-6,6}{-30:60}{300}{17cm,7cm}{smooth}
		\begin{flalign*}
		p(x)&=x^4-4x^2+1\\
		0&=x^4-4x^2+1\\
		0&=(x^2-1)^2\\
		x&=\pm 1\\
		\mathbb{L}&=\{-1;1\}&&
		\end{flalign*}
		\hrule
		\subsubsection{Lösen durch Polynomdivision}
			Wenn alle anderen Stränge reißen, ist man leider gezwungen die Polynomdivision durchzuführen. Um damit beginnen zu können, braucht man aber mindestens eine Nullstelle, die man durch Raten findet. Für das Beispiel unten finden wir so heraus, dass eine Nullstelle $x_1=1$ ist. Jetzt stellen wir $x=1$ nach $0$ um und erhalten $0=x-1$. Anschließend teilen wir unser Polynom durch $x-1$.\newline
			\makeplot{{2*x^3-5*x^2-2*x+5}}{{$p(x)=2x^3-5x^2-2x+5$}}{{3.9,3.2}}{-6,6}{-6,6}{-30:60}{300}{17cm,7cm}{smooth}
			\begin{flalign*}
			(2x^3-5x^2-2x+5):(x-1)&&
			\end{flalign*}
			Zunächst teilt man den Term mit der höchsten Potenz $2x^3$ durch $x$ und erhält $2x^2$. Das ist der erste Teil unseres Ergebnisses.
			\begin{flalign*}
			(2x^3-5x^2-2x+5):(x-1)=2x^2...&&
			\end{flalign*}
			Jetzt muss man zurück multiplizieren, indem man den Term $2x^2$, den wir gerade bekommen haben, mit unserem ursprünglichen Divisor $x-1$ multiplizieren. Das Ergebnis ziehen wir von unserem Polynom ab und holen anschließend den nächsten Ausdruck runter. Diesen Prozess wiederholen wir jetzt so oft, wie möglich.\newline\newline
			\polylongdiv[style=C,div=:]{2x^3-5x^2-2x+5}{x-1}\newline\newline
			Mit der Funktion, die wir jetzt haben, können wir ganz einfach die restlichen Nullstellen errechnen.\newline
			\makeplot{{2*x^2-3*x-5}}{{$f(x)=2x^2-3x-5$}}{{3.9,3.2}}{-6,6}{-10,10}{-30:60}{300}{17cm,7cm}{smooth}
			\begin{flalign*}
				f(x)&=2x^2-3x-5\\
				0&=2x^2-3x-5\\
				0&=x^2-1,5x-2,5\\
				x_{1,2}&=\frac{1,5}{2}\pm\sqrt{\left(-\frac{1,5}{2}\right)^2+2,5}\\
				x_1&=2,5\\
				x_2&=-1\\
				\mathbb{L}&=\{-1;1;2,5\}&&
			\end{flalign*}
		\subsubsection{Lösen mit dem Newtonverfahren}
		\label{subsubsec:newtonverfahren}
			Bei Polynomen 3. oder höheren gerade kann es passieren, dass man keine Nullstelle findet. Manchmal nicht einmal durch Raten. Die Funktion zu vereinfachen durch Polynomdivision ist auch erst möglich, wenn man eine Nullstelle gefunden hat. In diesem hilft das Newtonverfahren sich einem Wert zu nähern. Hier ist ein Beispiel für so einen Fall.
			\makeplot{{x^3+2*x^2-x-1}}{{$f(x)=x^{3}+2x^{2}-x-1$}}{{3.9,3.5}}{-6,6}{-10,10}{-20:20}{300}{17cm,7cm}{smooth}
			Als erstes legt man eine Wertetabelle an, um die grobe Positionen der Nullstellen zu finden. Dabei will man wissen, zwischen welchen Stellen das Vorzeichen wechselt.
			\begin{center}
				\bgroup
				\def\arraystretch{1.5}
				\begin{tabular}{ | l | c | c | c | c | c | c | c | }
					\hline
					\textbf{x} & -3 & -2 & -1 & 0 & 1 & 2 & 3 \\ \hline
					\textbf{f(x)} & -7 & 1 & 1 & -1 & 1 & 13 & 41 \\
					\hline
				\end{tabular}
				\egroup
			\end{center}
			Wir untersuchen jetzt einmal die Nullstelle zwischen $-1$ und $0$. Um uns der Nullstelle anzunähern teilen wir die Funktion durch ihre Ableitung an einer der beiden Stellen, zwischen denen unsere gesuchte Nullstelle liegt. Es gilt folgende Formel:
			\begin{tcolorbox}[boxsep=0pt,top=.3cm,left=1cm,right=1cm, bottom=.75cm,arc=0pt,auto outer arc,colback=white,colframe=black, enlarge top by=.5cm, enlarge bottom by=.45cm]
				\begin{flalign*}
					x_{n+1}=x_{n}-\frac{f(x_{n})}{f^{\prime}(x_{n})}
				\end{flalign*}
			\end{tcolorbox}
			\noindent Um diese Formel anzuwenden, brauchen wir als erstes die Ableitung und dann wiederholen wir diesen Prozess solange, bis wir genügend Nachkommastellen oder die tatsächliche Nullstelle gefunden haben.
			\begin{flalign*}
				f^{\prime}(x)=3x^2+4x-1&&
			\end{flalign*}
			\begin{flalign*}
				x_1&=(-1)-\frac{(-1)^{3}+2(-1)^{2}-(-1)-1}{3(-1)^2+4(-1)-1}=-\frac{1}{2}\\
				x_2&=(-\frac{1}{2})-\frac{(-\frac{1}{2})^{3}+2(-\frac{1}{2})^{2}-(-\frac{1}{2})-1}{3(-\frac{1}{2})^2+4(-\frac{1}{2})-1}=-\frac{5}{9}\\
				x_3&=(-\frac{5}{9})-\frac{(-\frac{5}{9})^{3}+2(-\frac{5}{9})^{2}-(-\frac{5}{9})-1}{3(-\frac{5}{9})^2+4(-\frac{5}{9})-1}=-\frac{929}{1674}\\
				x_4&=(-\frac{929}{1674})-\frac{(-\frac{929}{1674})^{3}+2(-\frac{929}{1674})^{2}-(-\frac{929}{1674})-1}{3(-\frac{929}{1674})^2+4(-\frac{929}{1674})-1}=-0,5549581321\\
				x_5&=(-0,5549581321)-\frac{(-0,5549581321)^{3}+2(-0,5549581321)^{2}-(-0,5549581321)-1}{3(-0,5549581321)^2+4(-0,5549581321)-1}\\
				&=-0,5549581321&&
			\end{flalign*}
			Wenn man zwei Mal den gleich Wert bekommt, weiß man, dass man den endgültigen Wert erreicht hat. Damit haben wir jetzt eine Nullstelle bestimmt, mit der wir z.B. die Polynomdivision anwenden können.
			\subsubsection{Grenzverhalten von ganzrationalen Funktionen}
				Hat man eine Funktion wie z.~B. $f(x)=-x^5+2x^3+3x^2+x+2$ und untersucht, wie sie sich gegen (minus) Unendlich verhält, würde man intuitiv sagen, sie nähert sich (minus) Unendlich an. Hier soll es darum gehen, wie man das auch rechnerisch herausfinden kann und sicher unterscheidet, ob nun plus oder minus Unendlich richtig ist. Der Trick bei Funktionen dieser Form ist es, das $x$ mit dem höchsten Exponenten auszuklammern.
				\makeplot{{-x^5+2*x^3+3*x^2+x+2}}{{$f(x)=-x^5+2x^3+3x^2+x+2$}}{{12.2,4.6}}{-10,10}{-5,15}{-10:20}{200}{17cm,7cm}{smooth}
				\begin{flalign*}
				&\lim_{x\to\infty}-x^5+2x^3+3x^2+x+2\\
				&\lim_{x\to\infty}x^5(\frac{-x^5}{x^5}+\frac{2x^3}{x^5}+\frac{3x^2}{x^5}+\frac{x}{x^5}+\frac{2}{x^5})\\
				&\lim_{x\to\infty}x^5(-1+\frac{2}{x^2}+\frac{3}{x^3}+\frac{1}{x^4}+\frac{2}{x^5})&&
				\end{flalign*}
				Wir sehen, dass sich die Brüche in der Klammer alle Null annähern, somit bleibt dort nur noch $-1$. Währenddessen nähert sich $x^5$ Unendlich an. Multipliziert mit $-1$ ergibt das dann minus Unendlich.
				\begin{flalign*}
				&\infty^5(-1+\frac{2}{\infty^2}+\frac{3}{\infty^3}+\frac{1}{\infty^4}+\frac{2}{\infty^5})\\
				=&\infty^5(-1+0+0+0+0)\\
				=&-\infty&&
				\end{flalign*}
				Dasselbe kann man jetzt natürlich auch für $x\to -\infty$ testen.
				\begin{flalign*}
				&-\infty^5(-1-\frac{2}{\infty^2}-\frac{3}{\infty^3}-\frac{1}{\infty^4}-\frac{2}{\infty^5})\\
				=&-\infty^5(-1-0-0-0-0)\\
				=&\infty&&
				\end{flalign*}
				Hinweis: Wenn man mit $x\to(-)\infty$ arbeitet, setzt man $\infty$ normalerweise nicht in die Funktion ein. Hier habe ich es einmal gemacht, damit man das Ergebnis besser nachvollziehen kann. Wenn man ausführlicher zu arbeiten will/muss, kann man den Limes von jedem Term einzeln aufstellen, um das Endergebnis zu begründen.
		\subsection{(Gebrochen)rationale Funktionen}
			Wenn wir von (gebrochen)rationalen Funktionen reden, meinem wir eine Funktion mit einem Polynom im Nenner eines Bruches. $f(x)=\frac{2}{x^3}$ ist z.~B. eine rationale Funktion, $f(x)=\frac{x^3}{2}$ jedoch nicht.
			\makeplot{{((2*x^4-10)/(x^3-3))}}{{$f(x)=\dfrac{2x^4-10}{x^3-3}$}}{{4.5,2.8}}{-6,6}{-5,10}{-22:20}{600}{17cm,7cm}{smooth}
			Das Besondere an rationalen Funktionen der Form $f(x)=\frac{g(x)}{h(x)}$ ist, dass wir zum Bestimmen von Nullstellen und Definitionslücken den Zähler und Nenner einzeln betrachten können. Mithilfe des Zählers bestimmen wir ganz einfach Nullstellen der Funktion.
			\begin{flalign*}
				g(x)&=0\\
				0&=2x^4-10\\
				10&=2x^4\\
				5&=x^4\\
				x&=\pm\sqrt[4]{5}\\
				\mathbb{L}&=\{-\sqrt[4]{5};\sqrt[4]{5}\}&&
			\end{flalign*}
			Mithilfe des Nenners bestimmen wir Definitionslücken.
			\begin{flalign*}
				h(x)&=0\\
				0&=x^3-3\\
				3&=x^3\\
				x&=\sqrt[3]{3}\\
				\mathbb{L}&=\{\sqrt[3]{3}\}&&
			\end{flalign*}
		\subsection{Exponentialfunktionen}
			Eine Funktion der Form $f(x)=a^x$ wird als Exponentialfunktion bezeichnet, denn die Variable $x$ steht im Exponenten. Speziell wird die Funktion $f(x)=e^x$ als \textit{natürliche Exponentialfunktion} bezeichnet.
			\makeplot{{e^x},{e^(-x)}}{{$f(x)=e^x$},{$f(x)=e^{-x}$}}{{9.5,3.3},{5.6,3.3}}{-6,6}{-6,6}{-10:10}{300}{17cm,7cm}{smooth}
			\subsubsection{Lösen von Exponentialgleichungen}
				Zum Lösen von Exponentialgleichungen brauchen wir in der Regel den \highlight{subsec:logarithmusgesetze}{Logarithmus}. Wie das funktioniert, sehen wir an dem Beispiel hier drunter. Dabei ist die Nullstelle der Funktion zu bestimmen. In diesem Beispiel sollte man sich außerdem nochmal daran erinnern, dass $\sqrt[n]{x}$ dasselbe ist, wie $x^{\frac{1}{n}}$.
				\makeplot{{(7^x)^(1/4)-4}}{{$f(x)=\sqrt[4]{7^x}-4$}}{{10,3.2}}{-6,6}{-10,10}{-30:30}{300}{17cm,7cm}{smooth}
				\begin{flalign*}
					f(x)&=\sqrt[4]{7^x}-4\\
					0&=\sqrt[4]{7^x}-4 && \mid +4 &&&&&&&&&&&&& \\
					4&=7^{\frac{x}{4}} && \mid log_7() &&&&&&&&&&&&& \\
					0,7124143742&=\frac{x}{4} && \mid \cdot 4 &&&&&&&&&&&&& \\
					x&=2,849657497\\
					\mathbb{L}=\{2,849657497\}&&
				\end{flalign*}
		\subsection{Logarithmusfunktionen}
			Funktionen wie $f(x)=log_3(x^2)$ werden als Logarithmusfunktionen bezeichnet, da sie einen oder mehrere Logarithmen beinhalten. Speziell bezeichnet man $f(x)=ln(x)$ als \textit{natürliche Logarithmusfunktion} und $f(x)=lg(x)$ als \textit{dekadische Logarithmusfunktion}.
			\makeplot{{ln(x)},{log10(x)},{-ln(abs(x))}, {log2(x)}}{{$f(x)=ln(x)$},{$f(x)=lg(x)$},{$f(x)=-ln(\vert x\vert)$},{$f(x)=log_2(x)$}}{{9.5,4.3},{12.5,4.8},{3.5,1},{9.5,1}}{-6,6}{-4,4}{-200:200}{500}{17cm,7cm}{smooth}
			\subsubsection{Lösen von Logarithmusgleichungen}
				Gesucht wird hier die Nullstelle einer Logarithmusfunktion gesucht. Hinweis: Um einen Logarithmus aufzulösen musst du beide Seiten der Gleichung als Exponent zur Basis des jeweiligen Logarithmus setzen. Um dort hinzu kommen hilft es enorm, die Gleichung zunächst umzuformen. Wenn du Schwierigkeiten mit den Umformungen in diesem Beispiel hast, schaue dir noch einmal die \highlight{subsec:potenzgesetze}{Potenzgesetze} und \highlight{subsec:logarithmusgesetze}{Logarithmusgesetze} an.
				\makeplot{{3*log10(x^3)-2*log10(x^2)-4}}{{$f(x)=3\cdot lg(x^3)-2\cdot lg(x^2)-4$}}{{11,3.3}}{-10,10}{-10,10}{-20:20}{300}{17cm,7cm}{smooth}
				\begin{flalign*}
					f(x)&=3\cdot lg(x^3)-2\cdot lg(x^2)-4\\
					0&=3\cdot lg(x^3)-2\cdot lg(x^2)-4\\
					0&=lg((x^3)^3)-lg((x^2)^2)-4&&\mid +4&&&&&&&&&&&\\
					lg\left(\frac{x^9}{x^4} \right)&=4\\
					lg(x^5)&=4&&\mid 10^{()}\\
					10^{lg(x^5)}&=10^4\\
					x^5&=10^4&&\mid \sqrt[5]{\;}\\
					x&=6,309573445\\
					\mathbb{L}&=\{6,309573445\}&&
				\end{flalign*}
		\subsection{Trigonometrische Funktionen}
			Trigonometrische Funktionen oder auch Winkelfunktionen genannt, beinhalten die aus der \highlight{sec:geometrie}{Geometrie} bekannten winkelabhängigen Funktionen, wie Sinus, Kosinus und Tangens. Dabei sind diese Funktionen hier allerdings abhängig von der Variable $x$ und damit im Bogenmaß, nicht im Gradmaß. Beim Taschenrechner muss man darauf achten, dass der richtige Modus eingestellt ist, ansonsten kann es sein, dass man versehentlich im falschen Maß rechnet. Auf dem CASIO fx-86DE PLUS, drückt man Shift, dann Setup und wählt dort die 3:Deg (engl. degree) für Gradmaß oder 4:Rad (engl. radian) fürs Bogenmaß.
			\makeplot{{sin(deg(x))},{cos(deg(x))},{tan(deg(x))}}{{$f(x)=\sin(x)$},{$f(x)=\cos(x)$},{$f(x)=\tan(x)$}}{{1.5,-1},{4.5,-1},{7.5,-1}}{-10,10}{-2,2}{-10:10}{200}{17cm,7cm}{smooth}
			Anmerkung: Die Nullstellen des Sinus sind die Extremstellen des Kosinus und umgekehrt. Ebenso habe Sinus und Tanges dieselben Nullstellen.
		\subsection{Verkettete Funktionen}
			Verkettungen sind eigentliche keine eigene Funktionsart, sondern eine Möglichkeit Funktionen durch Zusammensetzung zu transformieren. Man schreibt das als $f\circ g$ ("f nach g"). Man spricht hier bei $g$ auch von der \textit{inneren Funktion}, da sie als Argument in die \textit{äußere Funktion} $f$ eingesetzt wird.
			\makeplot{{sin(deg(x))},{(x^2)/2},{sin(deg((x^2)/2))}}{{$f(x)=\sin(x)$},{$g(x)=\frac{x^2}{3}$},{$f\circ g(x)=\sin(\frac{x^2}{3})$}}{{1.5,-1},{4.5,-1},{8,-1}}{-10,10}{-2,2}{-10:10}{200}{17cm,7cm}{smooth}
	\section{Differenzialrechnung}
		\subsection{Die Ableitung}
		\label{subsec:ableitung}
				Die Ableitung einer Funktion gibt Aufschluss über ihr \highlight{subsubsec:monotonie}{Monotonieverhalten} und die Veränderung ihres Anstiegs, denn die Werte der Ableitung $f^{\prime} (x)$ einer Funktion $f(x)$ entsprechen dem Anstieg einer Tangente an derselben Stelle von $f(x)$. Das kann man auch an dem unten stehenden Beispiel erkennen. Um eine Potenzfunktion abzuleiten, nehmen wir den Exponenten von jedem $x$ und holen ihn hinunter, um ihn vor das jeweilige $x$ zu schreiben. Anschließend reduzieren wir den Exponenten um $1$. Dabei ist zu beachten, dass die Ableitung einer reinen Zahl ohne $x$ immer $0$ ist und, dass die Ableitung von $x=1$ ist, denn $x^0$ entspricht $1$.
			\makeplot{{x^2},{2*x}}{{$f(x)=x^2$},{$f^{\prime}(x)=2x$}}{{4,3.5},{3,1}}{-10,10}{-10,10}{-20:20}{200}{17cm,7cm}{smooth}
			Hinweis: Um Fehler zu vermeiden, sollte man zunächst alle Terme so umformen, dass man einfach ableiten kann. Dafür solltest du die wichtigsten \highlight{sec:rechengesetze}{Rechengesetze} beherrschen.
			\begin{tcolorbox}[boxsep=0pt,top=.75cm,left=1cm,right=1cm, bottom=.65cm,arc=0pt,auto outer arc,colback=white,colframe=black, enlarge top by=.45cm, enlarge bottom by=.25cm]
				\textbf{Ableitungsregeln}\newline\newline
				Neben der oben genannten Regel zur Ableitung von Funktionen, gibt es noch einige andere, die einem das Leben erleichtern:
				\begin{multicols}{2}
					\noindent\begin{flalign*}
					&c&\rightarrow&\;\;0\\\\
					&x^n&\rightarrow&\;\;nx^{n-1}\\\\
					&\sqrt{x}&\rightarrow&\;\;\frac{1}{2\sqrt{x}}\\\\
					&e^x&\rightarrow&\;\;e^x\\\\
					&ln(x)&\rightarrow&\;\;\frac{1}{x}\\\\
					&a^x(a>0)&\rightarrow&\;\;ln(a\cdot a^x)\\\\
					&\frac{1}{x^n}&\rightarrow&\;\;-\frac{n}{x^{n+1}}&&&&&&&&&&&
					\end{flalign*}
					\begin{flalign*}
					&\sin(x)&\rightarrow&\;\; \cos(x)\\\\
					&\cos(x)&\rightarrow&\;\;-\sin(x)\\\\
					&\tan(x)&\rightarrow&\;\;\frac{1}{\cos^2x}\\\\
					&u(x)\cdot v(x)&\rightarrow&\;\;u^{\prime}(x)\cdot v(x)+u(x)\cdot v^{\prime}(x)\\\\
					&\frac{u(x)}{v(x)}&\rightarrow&\;\;\frac{u^{\prime}(x)\cdot v(x)-u(x)\cdot v^{\prime}(x)}{(v(x))^2}\\\\
					&(u\circ v)(x)&\rightarrow&\;\;u^{\prime}(v(x))\cdot v^{\prime}(x)&&&&&&&&&&&
					\end{flalign*}
				\end{multicols}
			\end{tcolorbox}
			Tipp: Solltest du es einmal nicht schaffen, eine Funktion mit den Ableitungsregeln abzuleiten oder diese vergessen haben, kannst du die Ableitung immer noch mithilfe des \highlight{subsubsec:differentialquotient}{Differentialquotionen!} bestimmen.
			\subsubsection{Differenzenquotient}
			\label{subsubsec:differenzenquotient}
				Um die Steigung einer Sekante zwischen zwei Punkten zu berechnen, benutzen wir den Differenzenquotient.
				Dieser Quotient berechnet sich indem man $x$ und $y$ der beiden Punkte jeweils voneinander abzieht und dann $y$ durch $x$ teilt.
				\begin{flalign*}
				m=\frac{y_2-y_1}{x_2-x_1}&&
				\end{flalign*}
				\begin{center}
					\begin{tikzpicture}
					\begin{axis}[
					domain=-10:10,
					width=17cm,
					height=7cm,
					restrict y to domain=-10:20,
					xmin=-10, xmax=10,
					ymin=-5, ymax=10,
					samples=100,
					axis y line=center,
					axis x line=middle,
					ticklabel style={fill=white},
					minor tick num=2,
					grid=both,
					grid style={line width=.1pt, draw=gridgray!10},
					major grid style={line width=.2pt,draw=gridgray!50}
					]
					\addplot+[mark=none, color=blue, solid, name path=A, smooth] {x^3+2};
					\draw [color=red, name path=B] (axis cs: .2,0.75) -- (axis cs: 2.2,9);
					\fill [name intersections={of=A and B, by={a,b}}]
					(a) circle(2.5pt) node[below right] {$A(x_1\mid y_1)$}
					(b) circle (2.5pt) node[right] {$B(x_2\mid y_2)$};
					\end{axis}
					\node [color=blue] at (4.75,3) {$f(x)=x^3+2$};
					\end{tikzpicture}
				\end{center}
				Allgemeiner ausgedrückt, gilt die Formel:
				\begin{flalign*}
				m=\frac{f(x+h)-f(x)}{x+h-x}=\frac{f(x+h)-f(x)}{h}&&
				\end{flalign*}
				Dabei ist $h$ eine beliebige Zahl, mit deren Hilfe wir uns jetzt an die genaue Steigung in einem Punkt annähern können. Je kleiner wir den Abstand $h$ wählen, desto genauer kommen wir an die Tangente oder auch Steigung der Stelle $x$.
			\subsubsection{Differentialquotient}
			\label{subsubsec:differentialquotient}
				Der Differenzenquotient erlaubt es uns die Steigung einer Funktion an einer bestimmten Stelle zu bestimmen. Im Abschnitt über den \highlight{subsubsec:differenzenquotient}{Differenzenquotient} haben wir schon geklärt, das wir näher an den Anstieg an der Stelle $x$ kommen, wenn wir den Abstand $h$ verringern. Der kleinstmögliche Abstand wäre $0$. Das geht allerdings nicht, da wir nicht $h=0$ in die Formel für den Differenzenquotient einsetzen und den Divisor somit gleich $0$ setzen dürfen. Um ums zu überlegen, was also für ein minimal kleines $h$ passieren würde, brauchen wir den Limes.
				\begin{flalign*}
				\lim_{h \to 0}	m=\lim_{h \to 0}\frac{f(x+h)-f(x)}{h}=f^{\prime}(x)&&
				\end{flalign*}
				Da setzen wir jetzt unsere Funktion $f(x)=x^3+2$ ein und formen solange um, bis wir das $h$ aus dem Divisor kriegen, damit wir für $h$ die Zahl $0$ einsetzen können. Hinweis: Nachdem du für $h$ die Zahl $0$ eingesetzt hast, darfst du nicht mehr Limes davor schreiben!
				\begin{flalign*}
				&\lim_{h \to 0}\frac{(x+h)^3+2-x^3-2}{h}\\
				=&\lim_{h \to 0}\frac{x^3+3hx^2+2xh^2+h^3+2-x^3-2}{h}\\
				=&\lim_{h \to 0}\frac{3hx^2+2xh^2+h^3}{h}\\
				=&\lim_{h \to 0}(3x^2+2xh+h^2)\\
				=&3x^2&&
				\end{flalign*}
				Das, was wir jetzt haben ist die erste Ableitung $f^{\prime}(x)$. Sofern nicht anders gewünscht, kann man diese oft auch wesentlich leichter bestimmen, indem man die \highlight{subsec:ableitung}{Ableitungsregeln} kennt.
			\subsubsection{Extrem- und Wendepunkte}
				\label{subsubsec:extrema}
				Leitet man die Ableitung einer Funktion noch mal ab, erhält man die zweite Ableitung $f^{\prime\prime}(x)$. Ebenso verhält es sich mit der dritten und allen weiteren Ableitungen. In der folgenden Abbildung sieht man eine Funktion und ihre erste, zweite sowie dritte Ableitung.
				\makeplot{{x^4+2*x^3-2*x^2+x-3},{4*x^3+6*x^2-4*x+1},{12*x^2+12*x-4},{24*x+12}}{{$f(x)=x^4+2x^3-2x^2+x-3$},{$f^{\prime}(x)=4x^3+6x^2-4x+1$},{$f^{\prime\prime}(x)=12x^2+12x-4$},{$f^{\prime\prime\prime}(x)=24x+12$}}{{12,2.4},{11.7,1.75},{11.4,1.1},{10.9,0.45}}{-6,6}{-20,10}{-30:20}{200}{17cm,7cm}{smooth}
				Man spricht bei den Bedingungen zur Bestimmung besonderer Punkte von notwendigen Bedingungen. Es existieren zudem weitere Bedingungen, mit deren Hilfe man diese Punkt genauer untersuchen kann, die sogenannten hinreichenden Bedingungen. Die obige Abbildung soll helfen, diese Bedingungen nachzuvollziehen.
				\begin{tcolorbox}[boxsep=0pt,top=.75cm,left=1cm,right=1cm, bottom=.65cm,arc=0pt,auto outer arc,colback=white,colframe=black, enlarge top by=.45cm, enlarge bottom by=.25cm]
					\textbf{Notwendige Bedingungen}\newline\newline
					Wenn \colorbox{blue!30}{$f^{\prime}(x)=0$} gilt, dann ist bei $x$ ein \colorbox{blue!30}{Extremal- o. Sattelpunkt} von $f$.\newline\newline
					Wenn \colorbox{red!30}{$f^{\prime\prime}(x)=0$} gilt, dann ist bei $x$ ein \colorbox{red!30}{Wendepunkt} von $f(x)$.
				\end{tcolorbox}
				\begin{tcolorbox}[boxsep=0pt,top=.75cm,left=1cm,right=1cm, bottom=.65cm,arc=0pt,auto outer arc,colback=white,colframe=black, enlarge top by=.45cm, enlarge bottom by=.25cm]
					\textbf{Hinreichende Bedingungen}\newline\newline
					Wenn \colorbox{teal!30}{$f^{\prime}(x)=0\land f^{\prime\prime}(x)\neq 0$} gilt, besitzt $f$ eine \colorbox{teal!30}{Extremalpunkt} bei $x$.\newline\newline
					Wenn \colorbox{yellow!30}{$f^{\prime}(x)=0\land f^{\prime\prime}(x)>0$} gilt, dann ist bei $x$ ein \colorbox{yellow!30}{Minimum} von $f$.\newline\newline
					Wenn \colorbox{green!30}{$f^{\prime}(x)=0\land f^{\prime\prime}(x)<0$} gilt, dann ist bei $x$ ein \colorbox{green!30}{Maximum} von $f$.\newline\newline\newline
					Wenn \colorbox{violet!30}{$f^{\prime\prime}(x)=0\land f^{\prime}(x)=0$} gilt, besitzt $f$ einen \colorbox{violet!30}{Sattelpunkt} bei $x$.\newline\newline
					Wenn \colorbox{purple!30}{$f^{\prime\prime}(x)=0\land f^{\prime\prime\prime}(x)<0$} gilt, dann ist bei $x$ eine \colorbox{purple!30}{Links-Rechts-Krümmung.}\newline\newline
					Wenn \colorbox{orange!30}{$f^{\prime\prime}(x)=0\land f^{\prime\prime\prime}(x)>0$} gilt, dann ist bei $x$ eine \colorbox{orange!30}{Rechts-Links-Krümmung.}
				\end{tcolorbox}
				\noindent Hinweis: Das Zeichen $\land$ bedeutet »und«, während $\lor$ »oder« bedeutet.
	\subsection{Limes: Der Grenzwert}
	\label{subsec:limes}
	Für manche Funktionen ist es nicht möglich den Werte einer bestimmten Stelle zu errechnen. Manchmal will man auch das Verhalten einer Funktion wissen, wenn $x$ gegen unendlich geht. In u.a. diesen Fällen braucht man dem Limes. Um den Grenzwert einer Funktion für $x\to a$ zu ermitteln muss man den links- und den rechtsseitigen Grenzwert betrachten. Es gilt Folgendes: 
	\begin{tcolorbox}[boxsep=0pt,top=.75cm,left=1cm,right=1cm, bottom=.65cm,arc=0pt,auto outer arc,colback=white,colframe=black, enlarge top by=.45cm, enlarge bottom by=.25cm]
		Um zu überprüfen, ob für eine Funktion mit $x$ gegen $a$ ein Grenzwert existiert, schaut man sich jeweils den linksseitigen und rechtsseitigen Grenzwert an. Sind diese gleich, so existiert ein Grenzwert. Sind sie unterschiedlich, existiert kein Grenzwert.
	\end{tcolorbox}
	\noindent Beim Grenzwert setzt man Zahlen ein, die sich in die Richtung von $a$ bewegen. Nehmen wir beispielsweise $f(x)=\frac{1}{x}$. $0$ können wir ja nicht für $x$ einsetzen, da wir nicht durch Null teilen dürfen. Was wir aber machen können ist sehr kleine Zahlen einsetzen, um zu schauen, ob wir eine Tendenz feststellen können.
	\begin{flalign*}
		f(1)&=\frac{1}{1}=1\\
		f(0,1)&=\frac{1}{0,1}=10\\
		f(00,1)&=\frac{1}{00,1}=100\\
		f(000,1)&=\frac{1}{000,1}=1000\\
		\lim_{x\searrow 0}\frac{1}{x}&=\infty&&
	\end{flalign*}
	Wir sehen also, dass sich unsere Funktion für $x$ gegen Null Unendlich nähert. Das, was wir uns jetzt angeguckt haben, ist aber nur der \text{rechtsseitige Grenzwert}, da wir Werte größer als $0$ eingesetzt haben und uns somit von rechts angenähert haben. Jetzt machen wir das Ganze noch einmal von links.
	\begin{flalign*}
		f(-1)&=\frac{1}{-1}=-1\\
		f(-0,1)&=\frac{1}{-0,1}=-10\\
		f(-00,1)&=\frac{1}{-00,1}=-100\\
		f(-000,1)&=\frac{1}{-000,1}=-1000\\
		\lim_{x\nearrow 0}\frac{1}{x}&=-\infty&&
	\end{flalign*}
	Wir sehen, dass $\infty \neq -\infty$ gilt. Somit haben wir keinen Grenzwert für $x\to 0$. Übrigens gibt es für den links- und rechtsseitigen Grenzwert unterschiedliche Notationen. Ich benutze hier, die Variante mit den diagonalen Pfeilen, da ich finde, dass sie gut darstellt, was man bei der Annäherung mit dem $x$ anstellt.
	\begin{tcolorbox}[boxsep=0pt,top=.75cm,left=1cm,right=1cm, bottom=.65cm,arc=0pt,auto outer arc,colback=white,colframe=black, enlarge top by=.45cm, enlarge bottom by=.25cm]
		\begin{multicols}{2}
			\begin{flushleft}
				\textbf{Linksseitiger Grenzwert}
				\begin{flalign*}
				\lim_{x\to a^-}\;oder\;\lim_{x\uparrow a}\;oder\;\lim_{x\nearrow a}\;oder\;\lim_{\substack{x\to a\\ x<a}}&&
				\end{flalign*}
			\end{flushleft}
			\begin{flushleft}
				\textbf{Rechtsseitiger Grenzwert}
				\begin{flalign*}
				\lim_{x\to a^+}\;oder\;\lim_{x\downarrow a}\;oder\;\lim_{x\searrow a}\;oder\;\lim_{\substack{x\to a\\ x>a}}&&
				\end{flalign*}
			\end{flushleft}
		\end{multicols}
	\end{tcolorbox}
	\subsection{Differenzierbarkeit}
	\begin{center}
		\begin{tikzpicture}
		\begin{axis}[
		domain=-10:10,
		width=17cm,
		height=7cm,
		restrict y to domain=-20:20,
		xmin=-10, xmax=10,
		ymin=-5, ymax=15,
		samples=200,
		axis y line=center,
		axis x line=middle,
		ticklabel style={fill=white},
		minor tick num=2,
		grid=both,
		grid style={line width=.1pt, draw=gridgray!10},
		major grid style={line width=.2pt,draw=gridgray!50}
		]
		\addplot+[mark=none, color=blue, solid, smooth, domain=-10:2] {x^2};
		\addplot+[mark=none, color=blue, solid, smooth, domain=2:10] {0.5*x^2};
		
		
		\end{axis}
		\node [color=blue] at (12.1,2.1) {$f(x)=\left\{x^2;\;\;\;\;\;\;\;x<2\atop\frac{1}{2}x^2;\;\;\;\;x\ge2\right.$};
		
		\end{tikzpicture}
	\end{center}
	Was uns an dem obigen Beispiel jetzt speziell interessiert, ist die Stelle $x=2$. Wir wollen wissen, ob $f$ in dieser Stelle stetig bzw. differenzierbar ist. Das heißt quasi, dass wir wissen wollen, ob man die Funktion zeichnen kann, ohne den Stift abzusetzen. Aber Achtung: Das ist keine sehr akkurate Definition, denn es gibt auch Funktionen, die stetig sind, obwohl man sie nicht durchzeichnen kann. Deshalb hier die mathematischen Bedingungen, die erfüllt sein müssen.
	\begin{tcolorbox}[boxsep=0pt,top=.75cm,left=1cm,right=1cm, bottom=.65cm,arc=0pt,auto outer arc,colback=white,colframe=black, enlarge top by=.45cm, enlarge bottom by=.25cm]
		Wenn eine Funktion $f(x)$ an der Stelle $x_0$ folgende Bedingungen erfüllt, so ist sie in dieser Stelle \textit{differenzierbar} bzw. \textit{stetig}.
		\begin{flalign*}
			&1.\;x_0 \in \mathbb{D}\\\\
			&2.\;\lim_{x\to x_0}f(x)\;existiert\\\\
			&3.\;\lim_{x\to x_0}f(x) = f(x_0)&&
		\end{flalign*}
		Sind die obigen Bedingungen für alle $x$ der Definitionsmenge erfüllt, so spricht man von einer \textit{stetigen Funktion}.
	\end{tcolorbox}
	\noindent Schauen wir uns das einmal für unser Beispiel an. Die erste Bedingung ist erfüllt, denn für $x=2$ ist $x$ definiert und es gilt $f(2)=\frac{1}{2}(2)^2=2$. Als nächsten prüfen wir die zweite Bedingung.
	\begin{multicols}{2}
		\noindent\begin{flalign*}
		\lim_{x\nearrow 2}f(x)&=\lim_{x\nearrow 2}x^2=2^2=4&&
		\end{flalign*}
		\begin{flalign*}
		\lim_{x\searrow 2}f(x)&=\lim_{x\searrow 2}\frac{1}{2}x^2=\frac{1}{2}\cdot 2^2=2&&
		\end{flalign*}
	\end{multicols}
	\noindent Unsere zweite Bedingung ist somit nicht erfüllt. Der linksseitige und der rechtsseitige Grenzwert sind unterschiedlich und somit existiert an dieser Stelle auch kein Grenzwert. Daher brauchen wir die letzte Bedingung gar nicht erst überprüfen und können es sogar nicht, da uns der Grenzwert fehlt.
	\subsubsection{Stetige Erweiterung}
	Eine Funktion wie z.~B. $f(x)=\frac{1}{x}$ hat zwar keinen Grenzwert für $x\to 0$, allerdings ist sie trotzdem stetig, da ihr Definitionsbereich $\mathbb{D}=\mathbb{R}\setminus\{0\}$ die $0$ ausschließt. Das ist wichtig zu wissen bei der Bestimmung der Differenzierbarkeit.
	\makeplot{{1/x}}{{$f(x)=\frac{1}{x}$}}{{5,1.4}}{-10,10}{-10,10}{-25:25}{300}{17cm,7cm}{smooth}
	Jetzt kann es aber sein, dass wir gerne die Definitionslücken unserer Funktion definieren wollen. Das geht sogar mithilfe der stetigen Erweiterung, allerdings nicht für jede Lücke. Eine Lücke, die man bestimmen kann, nennt man (be)hebbar. Damit eine Lücke behebbar ist, müssen die Bedingungen für die Differenzierbarkeit gegeben sein, außer der, dass $x_0$ nicht im Definitionsbereich liegt. Würde $x$ im Definitionsbereich liegen hätten wir natürlich auch keine Lücke. Logisch. Das Kriterium, dass der Grenzwert dem Funktionswert an der Stelle $x_0$ entspricht entfällt dementsprechend auch.
	\begin{tcolorbox}[boxsep=0pt,top=.75cm,left=1cm,right=1cm, bottom=.65cm,arc=0pt,auto outer arc,colback=white,colframe=black, enlarge top by=.45cm, enlarge bottom by=.25cm]
		Ist eine Lücke einer Funktion \textit{(be)hebbar}, so sind folgende Bedingungen erfüllt:
		\begin{flalign*}
		&1.\;x_0 \notin \mathbb{D}\\\\
		&2.\;\lim_{x\to x_0}f(x)\;existiert&&
		\end{flalign*}
	\end{tcolorbox}
	\noindent Für $x\to 0$ bei $f(x)=\frac{1}{x}$ haben wir keinen Grenzwert, somit ist die Funktion an der Stelle $x=0$ nicht stetig erweiterbar. Es folgt noch ein Beispiel einer stetig erweiterbaren Lücke.
	\makeplot{{((x^3)/x)-5}}{{$f(x)=\frac{x^3}{x}-5$}}{{12.2,4.6}}{-10,10}{-10,10}{-25:25}{300}{17cm,7cm}{smooth}
	Diese Funktion verhält sich jetzt wie eine Normalparabel. Leider ist sie für $x=0$ nicht definiert, da wir nicht durch $0$ teilen dürfen, also schauen wir, ob wir sie an dieser Stelle stetig erweitern können. In diesem Fall können wir uns den Test für links- und rechtsseitigen Grenzwert sparen, denn ich denke, man erkennt hier, dass wir einen Grenzwert haben.
	\begin{flalign*}
		\lim_{x\to 0}\frac{x^3}{x}-5&=\lim_{x\to 0}(x^2-5)\\
		&=-5&&
	\end{flalign*}
	Mit dem Grenzwert können wir jetzt eine zusammengesetzte Funktion aufstellen.
	\begin{flalign*}
		f(x)=\left\{\frac{x^3}{x};\;\;\;\;x\neq 0\atop -5;\;\;\;\;\;x=0\right.&&
	\end{flalign*}
	\subsection{Tangentengleichung}
	\label{subsec:tangentengleichung}
		Wenn man Extremal- und Wendepunkte untersucht, kann es vorkommen, dass man eine Tangenten für diese Punkte aufstellen soll. Das ist nicht schwer, denn man muss nur zwei Konstanten bestimmen. Da eine Tangente eine lineare Funktion ist, hat sie die grundlegende Form $T(x)=mx+n$. Sagen wir, wir wollen die Tangentengleichung an der Stelle $x=2$ der Funktion $f(x)=x^2+4$ aufstellen. Um $m$ zu bestimmen brauchen wir den Anstieg an der Stelle $x$. Diesen können wir mit der ersten Ableitung an der Stelle $x$ ermitteln.
		\begin{flalign*}
			f(x)&=x^2+4\\
			f^{\prime}(x)&=2x\\
			f^{\prime}(2)&=4&&
		\end{flalign*}
		Das können wir schon mal in unsere Gleichung einsetzen und erhalten $T(x)=4x+n$. Jetzt fehlt uns noch $n$, welches wir durch Umstellen bestimmen können, nachdem wir $x$ und $T(x)$ eingesetzt haben. Gesucht ist ja die Tangente an der Stelle $x=2$. Den $x$-Wert haben wir also schon mal und $f(x)$ bzw. $T(x)$ (Punkt existiert auf beiden Funktionen) können wir ganz einfach durch Einsetzen in die Funktion berechnen.
		\begin{flalign*}
			f(2)&=8=T(2)\\
			T(2)&=4\cdot 2+n\\
			8&=4\cdot 2+n\\
			n&=0&&
		\end{flalign*}
		Da $n=0$ ist, können wir es weglassen und haben bereits unsere vollständige Tangentengleichung. Hier nochmal eine Abbildung, um zu zeigen, dass das Ergebnis auch wirklich richtig ist.
		\makeplot{{x^2+4},{4*x}}{{$f(x)=x^2+4$},{$T(x)=4x$}}{{4,3},{10.5,2}}{-10,10}{0,20}{-5:25}{200}{17cm,7cm}{smooth}
	\subsection{Kurvendiskussion}
		In einer Kurvendiskussion untersucht man verschiedene Eigenschaften einer Funktion. Wie man diese Eigenschaften jeweils untersucht wird an anderen Stellen erklärt, die auch noch mal genannt werden. Im Folgenden wird einmal eine komplette Kurvendiskussion beispielhaft durchgeführt. Da man normalerweise keine Abbildung zur Verfügung hat, weil man die Funktion selber skizzieren soll, gibt es den Graphen der Funktion erst am Ende. Die zu untersuchende Funktion ist:\newline\newline
		$f(x)=x^{5}-3x^{3}+2x$
		\subsubsection{Symmetrie}
			Für diesen Teil der Kurvendiskussion solltest du wissen, wie man die \highlight{subsubsec:symmetrie}{Symmetrie} einer Funktion bestimmt.\newline\newline
			\textbf{Aufgabe:}\newline Bestimme begründet, ob die Funktion gerade, ungerade oder weder noch ist.\newline\newline
			\textbf{Lösung:}\newline Dazu müssen wir $f(-x)$ betrachten. Ist es gleich $f(x)$ haben wir eine gerade Funktion, ist es gleich $-f(x)$ haben wir eine ungerade Funktion, ansonsten haben wir weder noch.
			\begin{flalign*}
				f(-x)&=(-x)^5-3\cdot (-x)^3+2\cdot (-x)\\
				f(-x)&=-x^5+3x^3-2x\\
				-f(x)&=-(x^5-3x^3+2x)\\
				-f(x)&=-x^5+3x^3-2x&&
			\end{flalign*}
			Da $f(-x)=-f(x)$ gilt, liegt hier eine ungerade Funktion vor.
		\subsubsection{Nullstellen}
			Für diesen Teil der Kurvendiskussion solltest du wissen, wie man \highlight{sec:gleichungenvereinfachen}{Gleichungen} lösen kann.\newline\newline
			\textbf{Aufgabe:}\newline Bestimme alle Nullstellen der Funktion.\newline\newline
			\textbf{Lösung:}
			\begin{tcolorbox}[boxsep=0pt,top=0cm,left=0cm,right=20cm, bottom=0cm,arc=0pt,auto outer arc,colback=white,colframe=white]
				\begin{flalign*}
					&&x^5-3x^3+2x&=0&&\mid \text{Faktorisierung durch Ausklammern von }x\\
					\Leftrightarrow &&x(x^4-3x^2+2)&=0\\
					\Rightarrow &&x^4-3x^2+2&=0&&\mid \text{Substitution: }u=x^2\\
					\Rightarrow &&u^2-3u+2&=0&&\mid \text{p-q-Formel}\\
					\Rightarrow &&u_{1,2}&=\frac{3}{2}\pm\sqrt{\left(-\frac{3}{2}\right)^2-2}\\
					\Rightarrow &&u_1&=2&&\mid \text{Resubstitution: }u=x^2\\
					\Rightarrow &&x^2&=2\\
					\Rightarrow &&x_{1,2}&=\pm\sqrt{2}\\
					\Rightarrow &&u_2&=1&&\mid \text{Resubstitution: }u=x^2\\
					\Rightarrow &&x^2&=1\\
					\Rightarrow &&x_{3,4}&=\pm 1&&
				\end{flalign*}
			\end{tcolorbox}
			\begin{flalign*}
				\mathbb{L}=\left\{-\sqrt{2};-1;0;1;\sqrt{2}\right\}&&
			\end{flalign*}
		\subsubsection{Schnittpunkt mit y-Achse}
			Den $y$-Achsenschnittpunkt zu bestimmen, ist ganz einfach. Man muss nur $x=0$ einsetzen und das Ergebnis ausrechnen.\newline\newline
			\textbf{Aufgabe:}\newline Bestimme den Schnittpunkt mit der $y$-Achse der Funktion.\newline\newline
			\textbf{Lösung:}\newline
			In diesem Fall müssen wir eigentlich gar nicht rechnen, da wir bereits wissen, dass bei $x=0$ eine Nullstelle vorliegt. Trotzdem ist hier noch einmal der rechnerische Nachweis.
			\begin{flalign*}
				f(x)&=x^{5}-3x^{3}+2x\\
				\Rightarrow\qquad f(0)&=0^{5}-3\cdot 0^{3}+2\cdot 0\\
				\Leftrightarrow\qquad f(0)&=0&&
			\end{flalign*}
			Damit ist der Schnittpunkt mit der $y$-Achse der Punkt $S(0\mid 0)$.
		\subsubsection{Grenzverhalten}
			Für diesen Teil der Kurvendiskussion solltest du wissen, wie man den \highlight{subsec:limes}{Limes} benutzt.\newline\newline
			\textbf{Aufgabe:}\newline Bestimme das Verhalten der Funktion für $x\to\pm\infty$.\newline\newline
			\textbf{Lösung:}\newline
			Zunächst stellen wir die Funktion so um, dass wir den Grenzwert jedes Terms leicht einzeln betrachten können.
			\begin{tcolorbox}[boxsep=0pt,top=0cm,left=0cm,right=20cm, bottom=0cm,arc=0pt,auto outer arc,colback=white,colframe=white]
				\begin{flalign*}
				&&f(x)=&x^5-3x^3+2x&&\mid x^5\text{ ausklammern}\\
				\Leftrightarrow &&=&x^5\left(\frac{x^5}{x^5}-\frac{3x^3}{x^5}+\frac{2x}{x^5}\right)\\
				\Leftrightarrow &&=&x^5\left(1-\frac{3}{x^2}+\frac{2}{x^4}\right)&&
				\end{flalign*}
			\end{tcolorbox}
			\noindent Durch das Umstellen ist der Grenzwert der Funktion wesentlich einfacher zu zeigen.
			\begin{flalign*}
			&\lim_{x\to -\infty}x^5\left(1-\frac{3}{x^2}+\frac{2}{x^4}\right)=-\infty\\
			&\lim_{x\to \infty}x^5\left(1-\frac{3}{x^2}+\frac{2}{x^4}\right)=\infty&&
			\end{flalign*}
		\subsubsection{Extrema}
			Für diesen Teil der Kurvendiskussion solltest du wissen, wie man \highlight{subsubsec:extrema}{Extrema} bestimmt.\newline\newline
			\textbf{Aufgabe:}\newline Bestimme alle Extrempunkte der Funktion und gib an, ob es sich jeweils um Maxima oder Minima handelt.\newline\newline
			\textbf{Lösung:}\newline
			In den folgenden Abschnitten werden wir die Ableitungen benötigen, daher hier einmal alle Funktionen gesammelt:
			\begin{tcolorbox}[boxsep=0pt,top=0cm,left=0cm,right=20cm, bottom=0cm,arc=0pt,auto outer arc,colback=white,colframe=white]
				\begin{flalign*}
				&&f(x)=&x^{5}-3x^{3}+2x\\
				\Rightarrow &&f^{\prime}(x)=&5x^{4}-9x^{2}+2\\
				\Rightarrow &&f^{\prime\prime}(x)=&20x^{3}-18x\\
				\Rightarrow &&f^{\prime\prime\prime}(x)=&60x^{2}-18&&
				\end{flalign*}
			\end{tcolorbox}
			\noindent Notwendige Bedingung für Extrempunkte: $f^{\prime}(x)=0$
			\begin{tcolorbox}[boxsep=0pt,top=0cm,left=0cm,right=20cm, bottom=0cm,arc=0pt,auto outer arc,colback=white,colframe=white]
				\begin{flalign*}
				&&f^{\prime}(x)=&5x^{4}-9x^{2}+2\\
				\Rightarrow &&0=&5x^{4}-9x^{2}+2&&\mid \text{Substitution: }u=x^2\\
				\Rightarrow &&0=&5u^{2}-9u+2&&\mid :5\\
				\Leftrightarrow &&0=&u^{2}-1,8u+0,4&&\mid \text{p-q-Formel}\\
				\Rightarrow &&u_{1,2}=&\frac{1,8}{2}\pm\sqrt{\left(-\frac{1,8}{2}\right)^2-0,4}\\
				\Rightarrow &&u_1=&\frac{9+\sqrt{41}}{10}&&\mid \text{Resubstitution: }u=x^2\\
				\Rightarrow &&x^2=&\frac{9+\sqrt{41}}{10}\\
				\Rightarrow &&x_{1,2}=&\pm 1,241093237\\
				\Rightarrow &&u_2=&\frac{9-\sqrt{41}}{10}&&\mid \text{Resubstitution: }u=x^2\\
				\Rightarrow &&x^2=&\frac{9-\sqrt{41}}{10}\\
				\Rightarrow &&x_{3,4}=&\pm 0,5095955026&&
				\end{flalign*}
			\end{tcolorbox}
			\noindent Durch Einsetzen der ermittelten $x$-Werte, bekommen wir auch die $y$-Werte für die Punkte.
			\begin{flalign*}
				\mathbb{L}=\{&(-1,241093237;0,3082564193),(-0,5095955026;-0,656550059),\\
				&(0,5095955026;-0,656550059),(1,241093237;0,3082564193)\}&&
			\end{flalign*}
			Hinreichende Bedingung für Maxima: $f^{\prime}(x)=0\land f^{\prime\prime}(x)<0$\newline
			Hinreichende Bedingung für Minima: $f^{\prime}(x)=0\land f^{\prime\prime}(x)>0$
			\begin{flalign*}
				&f^{\prime\prime}(-1,241093237)=20\cdot (-1,241093237)^{3}-18\cdot (-1,241093237)=-15,89374835<0\\
				&f^{\prime\prime}(-0,5095955026)=20\cdot (-0,5095955026)^{3}-18\cdot (-0,5095955026)=6,526006628>0&&
			\end{flalign*}
			Da unsere Funktion ungerade ist, können wir darauf schließen, dass die Extrema auf der anderen Seite der $y$-Achse genau gegenteilig sind. Daher gilt also:
			\begin{tcolorbox}[boxsep=0pt,top=0cm,left=0cm,right=20cm, bottom=0cm,arc=0pt,auto outer arc,colback=white,colframe=white]
				\begin{flalign*}
					&x=-1,241093237\Rightarrow&\text{Maximalstelle}\\
					&x=-0,5095955026\Rightarrow&\text{Minimalstelle}\\
					&x=0,5095955026\Rightarrow&\text{Maximalstelle}\\
					&x=1,241093237\Rightarrow&\text{Minimalstelle}&&
				\end{flalign*}
			\end{tcolorbox}
		\subsubsection{Wendepunkte}
			Für diesen Teil der Kurvendiskussion solltest du wissen, wie man \highlight{subsubsec:extrema}{Wendepunkte} bestimmt.\newline\newline
			\textbf{Aufgabe:}\newline Bestimme alle Wendepunkte der Funktion und gib an, ob gib ihr Krümmungsverhalten an.\newline\newline
			\textbf{Lösung:}\newline
			Notwendige Bedingung für Wendepunkte: $f^{\prime\prime}(x)=0$
			\begin{tcolorbox}[boxsep=0pt,top=0cm,left=0cm,right=20cm, bottom=0cm,arc=0pt,auto outer arc,colback=white,colframe=white]
				\begin{flalign*}
				&&f^{\prime\prime}(x)=&20x^{3}-18x\\
				\Rightarrow &&0=&20x^{3}-18x&&\mid \text{Faktorisierung durch Ausklammern von }x\\
				\Rightarrow &&0=&x(20x^{2}-18)&&\mid \text{Ein Wendepunkt ist }x_1=0\\
				\Rightarrow &&0=&20x^{2}-18&&\mid +18\\
				\Leftrightarrow &&18=&20x^{2}&&\mid :20\\
				\Leftrightarrow &&\frac{9}{10}=&x^{2}&&\mid \sqrt{\ }\\
				\Rightarrow &&x_{2,3}=&\pm\sqrt{\frac{9}{10}}&&
				\end{flalign*}
			\end{tcolorbox}
			\begin{flalign*}
				\mathbb{L}=\left\{\left(-\sqrt{\frac{9}{10}};-0,1043551628\right),(0;0),\left(\sqrt{\frac{9}{10}};0,1043551628\right)\right\}&&
			\end{flalign*}
			\noindent Hinreichende Bedingung für Links-Rechts-Krümmung: $f^{\prime}(x)=0\land f^{\prime\prime\prime}(x)<0$\newline
			Hinreichende Bedingung für Rechts-Links-Krümmung: $f^{\prime}(x)=0\land f^{\prime\prime\prime}(x)>0$
			\begin{flalign*}
				&f^{\prime\prime\prime}\left(-\sqrt{\frac{9}{10}}\right)=60\cdot\left(-\sqrt{\frac{9}{10}}\right)^2-18=36>0\\
				&f^{\prime\prime\prime}(0)=60\cdot(0)^2-18=-18<0\\
				&f^{\prime\prime\prime}\left(\sqrt{\frac{9}{10}}\right)=60\cdot\left(\sqrt{\frac{9}{10}}\right)^2-18=36>0&&
			\end{flalign*}
			\begin{tcolorbox}[boxsep=0pt,top=0cm,left=0cm,right=20cm, bottom=0cm,arc=0pt,auto outer arc,colback=white,colframe=white]
				\begin{flalign*}
					&x=-\sqrt{\frac{9}{10}}\Rightarrow&\text{Rechts-Links-Krümmung}\\
					&x=0\Rightarrow&\text{Links-Rechts-Krümmung}\\
					&x=\sqrt{\frac{9}{10}}\Rightarrow&\text{Rechts-Links-Krümmung}&&
				\end{flalign*}
			\end{tcolorbox}
		\subsubsection{Tangentengleichungen der Wendepunkte}
			Für diesen Teil der Kurvendiskussion solltest du wissen, wie man eine \highlight{subsec:tangentengleichung}{Tangentengleichung} aufstellt.\newline\newline
			\textbf{Aufgabe:}\newline Stelle eine Tangentengleichung für die Wendestelle mit dem kleinsten $x$-Wert auf.\newline\newline
			\textbf{Lösung:}\newline
			Die Grundgleichung für eine lineare Funktion ist $T(x)=mx+n$. Wir müssen dafür $m$ und $n$ bestimmen. Den Anstieg $m$ erhalten wir aus der ersten Ableitung an der Wendestelle.
			\begin{tcolorbox}[boxsep=0pt,top=0cm,left=0cm,right=20cm, bottom=0cm,arc=0pt,auto outer arc,colback=white,colframe=white]
				\begin{flalign*}
					&&f^{\prime}(x)=&5x^{4}-9x^{2}+2\\
					\Rightarrow &&f^{\prime}\left(-\sqrt{\frac{9}{10}}\right)=&5\left(-\sqrt{\frac{9}{10}}\right)^{4}-9\left(-\sqrt{\frac{9}{10}}\right)^{2}+2\\
					\Leftrightarrow &&=&-\frac{41}{20}&&
				\end{flalign*}
			\end{tcolorbox}
			Jetzt, wo wir $m$ bestimmt haben, können wir die Koordinaten von unserem Wendepunkt einsetzen und nach $n$ umstellen.
			\begin{tcolorbox}[boxsep=0pt,top=0cm,left=0cm,right=20cm, bottom=.3cm,arc=0pt,auto outer arc,colback=white,colframe=white]
				\begin{flalign*}
				&&-0,1043551628=&-\frac{41}{20}\cdot \left(-\sqrt{\frac{9}{10}}\right)+n\\
				\Leftrightarrow&&-0,1043551628=&1,944800761+n&&\mid -1,944800761\\
				\Leftrightarrow&&n=&-2,049155924&&
				\end{flalign*}
			\end{tcolorbox}
			\noindent Damit ist unsere Tangentengleichung $T(x)=-\frac{41}{20}x-2.049155924$. Mit den ermittelten Informationen können wir folgende Abbildung zeichnen:
			\makeplot{{x^5-3*x^3+2*x},{-(41/20)*x-2.049155924}}{{$f(x)=x^{5}-3x^{3}+2x$},{$T(x)=-\frac{41}{20}x-2.049155924$}}{{2.1,3.5},{10.5,0.6}}{-3,3}{-1,1}{-5:5}{300}{17cm,7cm}{smooth}
		\subsubsection{Flächenberechnung}
\end{document}