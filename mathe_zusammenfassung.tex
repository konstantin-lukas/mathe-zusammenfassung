\documentclass[12pt]{article}

%opening
\title{Mathematik fürs Informatikstudium und Abitur:\\Eine Zusammenfassung}
\author{Konstantin Lukas}
\renewcommand{\contentsname}{Inhaltsverzeichnis}
\usepackage{xstring}
\usepackage{catchfile}
\usepackage{makeidx}
\makeindex

\newcommand{\monthword}[1]{\ifcase#1\or Januar\or Februar\or März\or April\or
	Mai\or Juni\or Juli\or August\or
	September\or Oktober\or November\or Dezember\fi}
\date{Fassung vom \the\day . \monthword{\month} \the\year}
\usepackage{amsmath,amsfonts,amssymb,amsthm}
\usepackage{braket}
\usepackage[a4paper, margin=2cm]{geometry}
\usepackage{bbold}
\usepackage{pgffor,listofitems}
\usepackage{tocloft}
\usepackage{etoc}
\usepackage{blindtext}
\usepackage{tabularx}
\renewcommand\tabularxcolumn[1]{m{#1}}
\usepackage{tikz}
\usetikzlibrary{shapes,arrows}
\usepackage{array}
\usepackage[most]{tcolorbox}
\usepackage[hidelinks]{hyperref}
\usepackage{pgfplots}
\usepgfplotslibrary{external} 
\tikzexternalize[prefix=figures/]
%\tikzset{external/force remake}
\usepgfplotslibrary{fillbetween}
\usepackage{tocloft}
\usepackage{polynom}
\usepackage{tkz-euclide}
\usepackage[ngerman]{babel}
\usepackage{hyphenat}
\usepackage{adjustbox}
\hyphenation{Mathe-matik wieder-gewinnen}
\definecolor{gridgray}{HTML}{AAAAAA}
\definecolor{transparentblue}{HTML}{CCCCFF}
\renewcommand{\cftsecleader}{\cftdotfill{\cftdotsep}}
\DeclareRobustCommand{\bigfrac}[3][5pt]{%
	\frac{\hspace{#1}#2\hspace{#1}}{\hspace{#1}#3\hspace{#1}}}
\newcommand{\highlight}[2]{\textcolor{blue}{\hyperref[#1]{#2}} (S. \pageref{#1})}
\newcommand{\getcolor}[1]{\ifcase#1\or blue\or red\or teal\or violet\or
	magenta\or orange\or purple\or brown\fi}
\newcommand{\makeplot}[9]{
	\readlist\xlimits{#4}
	\def\xlower{\xlimits[1]}
	\def\xupper{\xlimits[2]}
	\readlist\ylimits{#5}
	\def\ylower{\ylimits[1]}
	\def\yupper{\ylimits[2]}
	\readlist\dimensions{#8}
	\def\width{\dimensions[1]}
	\def\height{\dimensions[2]}
	\begin{center}
		\begin{tikzpicture}
		\begin{axis}[
		domain=\xlower:\xupper,
		width=\width,
		height=\height,
		restrict y to domain=#6,
		xmin=\xlower, xmax=\xupper,
		ymin=\ylower, ymax=\yupper,
		samples=#7,
		axis y line=center,
		axis x line=middle,
		ticklabel style={fill=white},
		minor tick num=2,
		grid=both,
		grid style={line width=.1pt, draw=gridgray!10},
		major grid style={line width=.2pt,draw=gridgray!50}
		]
		\foreach \graph [count=\i] in {#1} {
			\edef\temp{\noexpand\addplot+[mark=none, color=\getcolor{\i}, solid, #9] {\graph};}
			\temp
		}
		
		
		\end{axis}
		\readlist\pos{#3}
		\foreach \label [count=\i] in {#2} {
			\node [color=\getcolor{\i}] at (\pos[\i]) {\label};
		}
		
		\end{tikzpicture}
	\end{center}
}
\pgfplotsset{
	integral segments/.code={\pgfmathsetmacro\integralsegments{#1}},
	integral segments=3,
	integral/.style args={#1:#2}{
		ybar interval,
		domain=#1+((#2-#1)/\integralsegments)/2:#2+((#2-#1)/\integralsegments)/2,
		samples=\integralsegments+1,
		x filter/.code=\pgfmathparse{\pgfmathresult-((#2-#1)/\integralsegments)/2}
	}
}
\pgfplotsset{compat=1.5.1}
\begin{document}
\renewcommand\indexname{Stichwortverzeichnis}
\maketitle
\tableofcontents
\pagebreak
\etocsettocstyle{\noindent\rule{\linewidth}{.4pt}}

\section{Mengen}
	\begin{multicols}{2}
		\begin{flushleft}
			\textbf{Vereinigung}
			\index{Vereinigung}
			\begin{flalign*}
			\begin{tikzpicture}
			\fill[yellow] (3,0) circle (2cm);
			\fill[yellow] (0,0) circle (2cm);
			\draw (0,0) circle (2cm);
			\draw (3,0) circle (2cm);
			\draw (0,0) node {A};
			\draw (3,0) node {B};
			\end{tikzpicture}&&
			\end{flalign*}
			\begin{flalign*}
			A \cup B := \{x \mid x \in A \: \lor \: x \in B\}&&
			\end{flalign*}
		\end{flushleft}
		\begin{flushleft}
			\textbf{Durchschnitt}
			\index{Durchschnitt}
			\begin{flalign*}
			\begin{tikzpicture}
			\begin{scope}
			\draw [clip](0,0) circle (2cm);
			\fill[yellow] (3,0) circle (2cm);
			\end{scope}
			\draw (0,0) circle (2cm);
			\draw (3,0) circle (2cm);
			\draw (0,0) node {A};
			\draw (3.2,0) node {B};
			\end{tikzpicture}&&
			\end{flalign*}
			\begin{flalign*}
			A \cap B := \{x \mid x \in A \: \land \: x \in B\}&&
			\end{flalign*}
		\end{flushleft}
	\end{multicols}
	\vspace{1cm}
	\begin{multicols}{2}
		\begin{flushleft}
			\textbf{Differenz}
			\index{Differenz}
			\begin{flalign*}
			\begin{tikzpicture}
			\begin{scope}
			\fill[yellow] (0,0) circle (2cm);
			\clip (3,0) circle(2cm);
			\fill[white] (3,0) circle (2cm);
			\end{scope}
			\draw (0,0) circle (2cm);
			\draw (3,0) circle (2cm);
			\draw (0,0) node {A};
			\draw (3,0) node {B};
			\end{tikzpicture}&&
			\end{flalign*}
			\begin{flalign*}
			A \setminus B := \{x \mid x \in A \: \land \: x \notin B\}&&
			\end{flalign*}
		\end{flushleft}
		\begin{flushleft}
			\textbf{Symmetrische Differenz}
			\index{Symmetrische Differenz}
			\begin{flalign*}
			\begin{tikzpicture}
			\fill[yellow] (0,0) circle (2cm);
			\fill[yellow] (3,0) circle (2cm);
			\begin{scope}
			\draw [clip](0,0) circle (2cm);
			\fill[white] (3,0) circle (2cm);
			\end{scope}
			\draw (0,0) circle (2cm);
			\draw (3,0) circle (2cm);
			\draw (0,0) node {A};
			\draw (3,0) node {B};
			\end{tikzpicture}&&
			\end{flalign*}
			\begin{flalign*}
			&A \triangle B := \{ x \mid (x \in A) \: \veebar \: ( x \in B ) \}\\
			&A \triangle B := \{ x \mid (x \in A) \: \nleftrightarrow \: ( x \in B ) \}&&
			\end{flalign*}
		\end{flushleft}
	\end{multicols}
	\noindent \textbf{Noch ein paar Notationsnormen:}
	\begin{tcolorbox}[boxsep=0pt,top=0cm,left=0cm,right=20cm, bottom=0cm,arc=0pt,auto outer arc,colback=white,colframe=white]
		\begin{flalign*}
			&\text{Anzahl der Elemente in einer Menge (Mächtigkeit): }&\vert M\vert = x\\
			&\text{Echte Teilmenge N von M: }&N\subsetneq M\\
			&\text{Teilmenge N von M, die gleich M sein kann: }&N\subseteq M&&
		\end{flalign*}
	\end{tcolorbox}
	 \noindent Die Potenzmenge\index{Potenzmenge} einer Menge $M=\{1;2\}$ beinhaltet alle möglichen Teilmengen von $M$: $\mathfrak{P}(M)=\{\emptyset;\{1\};\{2\};\{1;2\}\}$.\newline\newline
	 Das Komplement\index{Komplement} einer Menge $M$ in einem Universum\index{Universum} $U$ enthält alle Elemente, die nicht in $M$ sind, aber im Universum $U$ existieren. Sind das Universum z.~B. die natürlichen Zahlen, so gilt: $M^C=\bar M=\mathbb{N}\setminus M$.
	\subsection{Definierte Zahlenmengen}
	\index{Mengen}
		\begin{center}
			\bgroup
			\def\arraystretch{0}
			\def\tabcolsep{0pt}
			\begin{tabularx}{\linewidth}{X@{\hspace{0.4cm}}X}
				\begin{tcolorbox}[boxsep=0pt,top=.5cm,left=.5cm,right=.5cm, bottom=.5cm,arc=0pt,auto outer arc,colback=white,colframe=black]
					\textbf{Natürliche Zahlen}\index{Natürliche Zahl}\newline\newline
					$\mathbb{N} = \{ 1; 2; 3; ... \}$
				\end{tcolorbox}
				&
				\begin{tcolorbox}[boxsep=0pt,top=.5cm,left=.5cm,right=.5cm, bottom=.5cm,arc=0pt,auto outer arc,colback=white,colframe=black]
					\textbf{Natürliche Zahlen mit Null}\index{Menge der natürlichen Zahlen}\newline\newline
					$\mathbb{N}_0 = \{ 0; 1; 2; 3; ... \}$
				\end{tcolorbox}
			\end{tabularx}
			\egroup
		\end{center}
		\begin{center}
		\bgroup
		\def\arraystretch{0}
		\def\tabcolsep{0pt}
		\begin{tabularx}{\linewidth}{X@{\hspace{0.4cm}}X}
			\begin{tcolorbox}[boxsep=0pt,top=.5cm,left=.5cm,right=.5cm, bottom=.5cm,arc=0pt,auto outer arc,colback=white,colframe=black]
				\textbf{Ganze Zahlen}\index{Ganze Zahl}\newline\newline
				$\mathbb{Z} = \{ ... ; -2; -1; 0; 1; 2; 3; ... \}$
			\end{tcolorbox}
			&
			\begin{tcolorbox}[boxsep=0pt,top=.5cm,left=.5cm,right=.5cm, bottom=.5cm,arc=0pt,auto outer arc,colback=white,colframe=black]
				\textbf{Irrationale Zahlen}\index{Irrationale Zahl}\newline\newline
				$\mathbb{R} \setminus \mathbb{Q}$
			\end{tcolorbox}
		\end{tabularx}
		\egroup
		\end{center}
		\begin{center}
			\bgroup
			\def\arraystretch{0}
			\def\tabcolsep{0pt}
			\begin{tabularx}{\linewidth}{X@{\hspace{0.4cm}}X}
				\adjustbox{valign=t}{\begin{tcolorbox}[boxsep=0pt,top=.5cm,left=.5cm,right=.5cm, bottom=.5cm,arc=0pt,auto outer arc,colback=white,colframe=black]
					\textbf{Reelle Zahlen}\index{Reelle Zahl}\newline\newline
					Die reellen Zahlen $\mathbb{R}$ umfassen die rationalen und irrationalen Zahlen.
				\end{tcolorbox}}
				&
				\adjustbox{valign=t}{\begin{tcolorbox}[boxsep=0pt,top=.5cm,left=.5cm,right=.5cm, bottom=.5cm,arc=0pt,auto outer arc,colback=white,colframe=black]
					\textbf{Rationale Zahlen}\index{Rationale Zahl}\newline\newline
					$\mathbb{Q} = \left\{ \frac{p}{q} \mid p, q \in \mathbb{Z}, q \neq 0 \right\}$
				\end{tcolorbox}}
			\end{tabularx}
			\egroup
		\end{center}
	\subsection{Abbildungen}
		Abbildungen\index{Abbildung} ordnen jedem Wert einer Menge $A$ einen Wert aus der Menge $B$ mithilfe einer Vorschrift $T$ zu. In der Analysis kommen Abbildungen in Form von Funktionen zum Einsatz. Man findet sie aber auch in der Geometrie, Stochastik oder der linearen Algebra. Wenn man ein Element $y$ aus der Menge $B$ nimmt, dem ein $x$ aus der Menge $A$ zugeordnet wird, spricht man vom \textit{Bild}\index{Bild} von $x$. Die Notation ist wie folgt:
		\begin{flalign*}
			T:A\rightarrow B,x\mapsto T(x)&&
		\end{flalign*}
		Dabei ist $A$ der \textit{Definitionsbereich}\index{Definitionsbereich} und $B$ der \textit{Zielbereich}\index{Zielbereich} von $T$.\newline\newline
		Für die Begriffe injektiv, surjektiv und bijektiv, schaue dir das Kapitel zur \highlight{subsubsec:umkehrbarkeit}{Umkehrbarkeit} an, denn diese gelten ebenfalls ganz generell für Abbildungen und nicht nur für Funktionen.
	\subsection{Kartesisches Produkt}
		Hat man ein Zahlenpaar, wie eine 2D-Koordinate $(x;y)$, so spricht man von einem geordneten Paar, denn die Reihenfolge der Zahlen ist relevant. Wenn man zwei Mengen $A$ und $B$ hat, dann nennt man die Menge aller möglichen Paare $(x,y)$ mit $x$ aus $A$ und $y$ aus $B$ das kartesische Produkt der Mengen $A$ und $B$. Die entsprechende Notation dafür ist $A\times B$. Wenn $A$ und $B$ gleich sind, schreibt man stattdessen auch $A^2$. Obwohl hier von einem Produkt die Rede ist, ist es nicht egal, wie herum man die Faktoren schreibt. $A\times B$ ist nicht das Gleiche, wie $B\times A$.
		\begin{center}
			\bgroup
			\def\arraystretch{0}
			\def\tabcolsep{0pt}
			\begin{tabularx}{\linewidth}{X@{\hspace{0.4cm}}X}
				\adjustbox{valign=t}{\begin{tcolorbox}[boxsep=0pt,top=.5cm,left=.5cm,right=.5cm, bottom=.5cm,arc=0pt,auto outer arc,colback=white,colframe=black]
						$[4;11]\times [5;9]$
						\begin{center}
							\begin{tikzpicture}
							\begin{axis}[
							width=7cm,
							height=7cm,
							xmin=-5, xmax=15,
							ymin=-5, ymax=15,
							axis y line=center,
							axis x line=middle,
							ticklabel style={fill=white},
							minor tick num=2,
							grid=both,
							grid style={line width=.1pt, draw=gridgray!10},
							major grid style={line width=.2pt,draw=gridgray!50},
							axis equal image
							]
							\fill[blue] (axis cs: 4,5) rectangle (axis cs: 11,9);
							\end{axis}
							\end{tikzpicture}
						\end{center}
				\end{tcolorbox}}
				&
				\adjustbox{valign=t}{\begin{tcolorbox}[boxsep=0pt,top=.5cm,left=.5cm,right=.5cm, bottom=.5cm,arc=0pt,auto outer arc,colback=white]
						$[5;9]\times [4;11]$
						\begin{center}
							\begin{tikzpicture}
							\begin{axis}[
							width=7cm,
							height=7cm,
							xmin=-5, xmax=15,
							ymin=-5, ymax=15,
							axis y line=center,
							axis x line=middle,
							ticklabel style={fill=white},
							minor tick num=2,
							grid=both,
							grid style={line width=.1pt, draw=gridgray!10},
							major grid style={line width=.2pt,draw=gridgray!50},
							axis equal image
							]
							\fill[red] (axis cs: 5,4) rectangle (axis cs: 9,11);
							\end{axis}
							\end{tikzpicture}
						\end{center}
				\end{tcolorbox}}
			\end{tabularx}
			\egroup
		\end{center}
	\subsection{Komplexe Zahlen}
		Mit den reellen Zahlen war es uns bisher nicht möglich die Gleichung $x^2=-1$ zu lösen. Deshalb gibt es den Zahlenbereichen $\mathbb{C}$ der komplexen Zahlen. Darin wird $i:=\sqrt{-1}$ definiert. $i$ alleine ist aber noch nicht unbedingt eine komplexe Zahl, denn $i:=\sqrt{-1}$ alleine hilft uns immer noch nicht z.~B. $x^2=-9$ zu lösen. Um also verschiedene Gleichungen zu lösen, gibt es eine andere Notationsform: Eine komplexe Zahl\index{Komplexe Zahlen} hat die Form $z=a+b\cdot i$, wobei $a$ und $b$ reelle Zahlen sind. Dabei wird $a$ als \textit{Realteil}\index{Realteil}, $b$ als \textit{Imaginärteil}\index{Imaginärteil}, $i$ als \textit{imaginäre Einheit}\index{Imaginäre Einheit} und $b\cdot i$ als \textit{imaginäre Zahl}\index{Imaginäre Zahl} bezeichnet. Anders als die reellen Zahl werden die komplexen Zahlen nicht auf einem Zahlenstrahl dargestellt, sondern der Zahlenstrahl der reellen Zahlen wird noch um die Achse des Imaginärteils erweitert. Dieses System nennt man auch \textit{gaußsche Zahlenebene}\index{Gaußsche Zahlenebene}. Da $i$ eine definierte Einheit (also immer gleich) ist, können wir mit den reellen Zahlen $a$ und $b$ die komplexen Zahlen als \highlight{sec:vektorgeometrie}{Vektoren} darstellen. Das Praktische daran ist, dass wir die komplexen Zahl aus dieser Ebene ganz einfach ablesen können.
		\begin{center}
			\begin{tikzpicture}
			\begin{axis}[
			width=17cm,
			height=7cm,
			xmin=-30, xmax=30,
			ymin=-10, ymax=10,
			axis y line=center,
			axis x line=middle,
			ticklabel style={fill=white},
			minor tick num=2,
			grid=both,
			grid style={line width=.1pt, draw=gridgray!10},
			major grid style={line width=.2pt,draw=gridgray!50},
			axis equal image,
			xlabel={$\mathfrak{Re}(z)$},
			ylabel={$\mathfrak{Im}(z)$}
			]
			\addplot [-stealth, thick, blue] coordinates { (0,0) (10,0) };
			\addplot [-stealth, thick, red] coordinates { (10,0) (10,5) };
			\addplot [-stealth, thick, teal] coordinates { (0,0) (10,5) };
			\end{axis}
			\node at (3.75,3.25) {$z=\textcolor{blue}{10}+\textcolor{red}{5}i=\begin{pmatrix}\textcolor{blue}{10}\\\textcolor{red}{5}\end{pmatrix}$};
			\end{tikzpicture}
		\end{center}
		Für die kartesische Form\index{Kartesische Form} ($z=a+b\cdot i$) gelten alle gängigen Rechengesetze. Es gibt jedoch auch etwas neues, das sich \textit{komplexe Konjugation}\index{Komplexe Konjugation} nennt. Das klingt kompliziert als es ist, denn man dreht nur das Vorzeichen der imaginären Zahl um schreibt zur Markierung einen Strich über die Zahl: $z=a+b\cdot i\Rightarrow \bar{z}=a-b\cdot i$. Was beim Rechen mit komplexen Zahl besonders interessant ist, ist wie wir damit wieder auf reelle Zahlen kommen. Da $i=\sqrt{-1}$ festgelegt ist, gilt somit auch $i^2=-1$. Das heißt z.~B., das Produkt aus $1+i$ und $-1-i$ ist $-2i$. Etwas schwieriger wird es bei der Division, da es durch das $i$ schwer werden kann zurück auf die kartesische Form zu kommen. Dafür gibt es einen kleinen Trick, denn es gilt $z\cdot \bar{z}=a^2+b^2$. Das kann man ganz einfach durch nachrechnen allgemein prüfen:
		\begin{tcolorbox}[boxsep=0pt,top=0cm,left=0cm,right=20cm, bottom=0cm,arc=0pt,auto outer arc,colback=white,colframe=white]
			\begin{flalign*}
				&&z\cdot \bar{z}=&(a+b\cdot i)\cdot (a-b\cdot i)\\
				\Leftrightarrow&&=&a^2-abi+abi-(bi)^2\\
				\Leftrightarrow&&=&a^2-abi+abi-b^2i^2&&\mid\text{denk daran, dass }i^2=-1\text{ gilt}\\
				\Leftrightarrow&&=&a^2-abi+abi+b^2\\
				\Leftrightarrow&&=&a^2+b^2&&
			\end{flalign*}
		\end{tcolorbox}
		\noindent Das ist übrigens auch der Betrag der Zahl bzw. die Länge des Vektors ins Quadrat, denn wir erinnern uns, die Länge eines Vektors beträgt $\sqrt{a^2+b^2}$. Gut, aber wie hilft uns das jetzt zwei komplexe Zahlen zu dividieren? Nun, wenn wir in einer Gleichung mal Eins rechnen ist das ja eine Äquivalenzumformung\index{Äquivalenzumformung}, bzw. eigentlich verändern wir ja gar nichts wirklich, denn $1\cdot x$ ist $x$. Und was wollen erreichen? Dass im Nenner des Bruches kein $i$ mehr steht. Das bedeutet für die Rechnung $\frac{z_1}{z_2}$, wir multiplizieren mit $\frac{\bar{z_2}}{\bar{z_2}}$. Dazu ein praktisches Beispiel:
		\begin{tcolorbox}[boxsep=0pt,top=0cm,left=0cm,right=20cm, bottom=0cm,arc=0pt,auto outer arc,colback=white,colframe=white]
			\begin{flalign*}
				&&\frac{2+9i}{3-6i}=&\frac{2+9i}{3-6i}\cdot \frac{3+6i}{3+6i}\\
				\Leftrightarrow&&=&\frac{(2+9i)\cdot(3+6i)}{(3-6i)\cdot(3+6i)}\\
				\Leftrightarrow&&=&\frac{6+12i+27i+54i^2}{3^2+6^2}\\
				\Leftrightarrow&&=&\frac{-48+39i}{45}\\
				\Leftrightarrow&&=&-\frac{16}{15}+\frac{13}{15}i&&
			\end{flalign*}
		\end{tcolorbox}
		\subsubsection{Polarform}
			Bis hierhin haben wir die kartesische Form\index{Kartesische Form} benutzt, die uns ganz einfach die Koordinaten in der gaußschen Zahlenebene ablesen lässt. Eine komplexe Zahl lässt sich jedoch auch auf andere Weise genau festlegen, nämlich mit der \textit{Polarform}\index{Polarform}. Dabei wir die Zahl durch die Länge des Vektors und dem Winkel zwischen Abszisse und Vektor festgelegt.
			\begin{center}
				\bgroup
				\def\arraystretch{0}
				\def\tabcolsep{0pt}
				\begin{tabularx}{\linewidth}{X@{\hspace{0.4cm}}X}
					\adjustbox{valign=t}{\begin{tcolorbox}[boxsep=0pt,top=.5cm,left=.5cm,right=.5cm, bottom=.5cm,arc=0pt,auto outer arc,colback=white,colframe=black]
							\textbf{Kartesische Form}
							\begin{center}
								\begin{tikzpicture}
								\begin{axis}[
								width=7cm,
								height=7cm,
								xmin=-5, xmax=15,
								ymin=-5, ymax=15,
								axis y line=center,
								axis x line=middle,
								ticklabel style={fill=white},
								minor tick num=2,
								grid=both,
								grid style={line width=.1pt, draw=gridgray!10},
								major grid style={line width=.2pt,draw=gridgray!50},
								axis equal image,
								xlabel={$\mathfrak{Re}(z)$},
								ylabel={$\mathfrak{Im}(z)$}
								]
								\addplot [dashed, blue] coordinates { (0,10) (5,10) };
								\addplot [dashed, red] coordinates { (5,0) (5,10) };
								\addplot [-stealth, thick, teal] coordinates { (0,0) (5,10) };
								\end{axis}
								\node at (4,3.25) {$\textcolor{blue}{5}+\textcolor{red}{10}i$};
								\end{tikzpicture}
							\end{center}
							\begin{tcolorbox}[boxsep=0pt,top=0cm,left=.5cm,right=.5cm, bottom=.5cm,arc=0pt,auto outer arc,colback=white,colframe=gray]
								\begin{flalign*}
									z=a+b\cdot i
								\end{flalign*}
							\end{tcolorbox}
					\end{tcolorbox}}
					&
					\adjustbox{valign=t}{\begin{tcolorbox}[boxsep=0pt,top=.5cm,left=.5cm,right=.5cm, bottom=.5cm,arc=0pt,auto outer arc,colback=white]
							\textbf{Polarform}
							\begin{center}
								\begin{tikzpicture}
								\begin{axis}[
								width=7cm,
								height=7cm,
								xmin=-5, xmax=15,
								ymin=-5, ymax=15,
								axis y line=center,
								axis x line=middle,
								ticklabel style={fill=white},
								minor tick num=2,
								grid=both,
								grid style={line width=.1pt, draw=gridgray!10},
								major grid style={line width=.2pt,draw=gridgray!50},
								axis equal image,
								xlabel={$\mathfrak{Re}(z)$},
								ylabel={$\mathfrak{Im}(z)$}
								]
								\draw[orange] (axis cs:5,0)arc[radius=1.45cm,start angle=0,end angle=60];
								\addplot [-stealth, thick, teal] coordinates { (0,0) (5,10) };
								\node[orange] at (axis cs:2.6,1.75) {$\phi$};
								\end{axis}
								\node at (3.5,4.5) {$\textcolor{teal}{r}\cdot(\cos\textcolor{orange}{\phi}+i\cdot\sin\textcolor{orange}{\phi})$};
								\end{tikzpicture}
							\end{center}
							\begin{tcolorbox}[boxsep=0pt,top=0cm,left=.5cm,right=.5cm, bottom=.5cm,arc=0pt,auto outer arc,colback=white,colframe=gray]
								\begin{flalign*}
								z=r\cdot(\cos\phi+i\cdot\sin\phi)=r\cdot e^{i\phi}
								\end{flalign*}
							\end{tcolorbox}
					\end{tcolorbox}}
				\end{tabularx}
				\egroup
			\end{center}
			Hinweis: $r\cdot(\cos\phi+i\cdot\sin\phi)$ wird auch trigonometrische Darstellung und $r\cdot e^{i\phi}$ auch als Exponentialdarstellung der Polarform bezeichnet. $\phi$ wird auch als Argument der Zahl $z$ bezeichnet.\newline\newline
			Natürlich gibt es auch Regeln, wie man zwischen kartesischer und Polarform umwandeln kann. Größtenteils sind diese sehr einfach, da sie sich aus dem Rechnen mit dem rechtwinkligen Dreieck herleiten, jedoch ein kleiner Hinweis: $sgn(b)$ (Signum\index{Signum}) meint das Vorzeichen von $b$ und $\arccos$ ist dasselbe wie $\cos^{-1}$ auf dem Taschenrechner.
			\begin{tcolorbox}[boxsep=0pt,top=0.5cm,left=.5cm,right=.5cm, bottom=.5cm,arc=0pt,auto outer arc,colback=white,colframe=black, enlarge top by=0.25cm]
				Für die Umrechnung zwischen kartesischer Form und Polarform gelten folgende Gleichungen:
				\begin{flalign*}
					&r=\sqrt{a^2+b^2}\\
					&\phi=\displaystyle \operatorname {sgn}(b)\cdot\arccos\left(\frac{a}{r}\right)\\
					&a=r\cdot\cos(\phi)\\
					&b=r\cdot\sin(\phi)&&
				\end{flalign*}
			\end{tcolorbox}
			\noindent Noch mal der Hinweis, wie $\displaystyle \operatorname {sgn}$ definiert ist:
			\begin{flalign*}
				{\displaystyle \operatorname {sgn}(x):={\begin{cases}+1&\;{\text{falls}}\quad x>0\\\;\;\,0&\;{\text{falls}}\quad x=0\\-1&\;{\text{falls}}\quad x<0\\\end{cases}}}&&
			\end{flalign*}
			\subsubsection{Komplexe Zahlen potenzieren}
			\label{subsubsec:komplexpotenz}
				Möchte man eine komplexe Zahl mit einer sehr hohen Zahl potenzieren, ist es ratsam, sie zunächst in die Polarform zu bringen.
				Das hat den Hintergrund, dass wir dann die \highlight{subsec:potenzgesetze}{Potenzgesetze} anwenden können, allerdings nur, wenn die Potenz eine ganze Zahl ist. Dieser Zusammenhang wurde erstmals von Abraham de Moivre entdeckt und entsprechend benannt. 
				\begin{flalign*}
				r^n\cdot e^{in\phi}=r^n\cdot(\cos(n\phi)+i\cdot\sin(n\phi))&&
				\end{flalign*}
				\begin{tcolorbox}[boxsep=0pt,top=0.5cm,left=.5cm,right=.5cm, bottom=.5cm,arc=0pt,auto outer arc,colback=white,colframe=black, enlarge top by=0.25cm, enlarge bottom by=0.25cm]
					\textbf{Satz von de Moivre}
					\index{Satz von de Moivre}
					\begin{flalign*}
						(\cos(\phi)+i\cdot\sin(\phi))^n=\cos(n\cdot\phi)+i\cdot\sin(n\cdot\phi)=e^{in\phi}\text{ wenn }n\in\mathbb{Z}&&
					\end{flalign*}
				\end{tcolorbox}
				\noindent Zur Anwendung dieser Formel, hier ein Beispiel:
				\begin{tcolorbox}[boxsep=0pt,top=0cm,left=0cm,right=20cm, bottom=0cm,arc=0pt,auto outer arc,colback=white,colframe=white]
					\begin{flalign*}
						&&z=&\left(\frac{\sqrt{2}}{2}-\frac{\sqrt{2}}{2}i\right)^{20}\\
						&&r=&\sqrt{\left(\frac{\sqrt{2}}{2}\right)^2+\left(-\frac{\sqrt{2}}{2}\right)^2}\\
						\Leftrightarrow&&=&1\\
						&&\phi=&\displaystyle \operatorname {sgn}\left(-\frac{\sqrt{2}}{2}\right)\cdot\arccos\left(\frac{\sqrt{2}}{2}\right)&&\mid\text{Achtung: Bogenmaß verwenden!}\\
						\Leftrightarrow&&=&-\frac{1}{4}\pi\\
						&&z=&(1\cdot e^{-\frac{1}{4}\pi i})^{20}\\
						\Leftrightarrow&&=&e^{-5\pi i}=e^{\pi i}=\cos(\pi)+i\cdot\sin(\pi)=-1&&
					\end{flalign*}
				\end{tcolorbox}
				\noindent Das Ergebnis einer solchen Rechnung wird i.~d.~R. so gekürzt, dass der Winkel zwischen $0$ und $2\pi$ bzw. $0^{\circ}$ und $360^{\circ}$ liegt, indem man jeweils eine Umdrehung dazu rechnet oder abzieht ($\pm2\pi$ oder $\pm360^{\circ}$). Das hat den Hintergrund, dass eine Drehung des Vektors um mehr als $360^{\circ}$ nur unnötig kompliziert wäre, denn man braucht in der Ebene nur maximal $360^{\circ}$ und kann seinen Vektor in jede Richtung zeigen lassen.
			\subsubsection{Komplexe Zahlen radizieren}
				Wie wir wurzeln aus einer komplexen Zahl ziehen, können wir uns mit den \highlight{subsec:potenzgesetze}{Potenzgesetzen} herleiten. Der große Unterschied zum Potenzieren mit einer ganzen Zahl ist, dass wir mehrere Ergebnisse erhalten. Beim Rechnen der $n$-ten Wurzel, gibt es $n$ Ergebnisse um genau zu sein. Welchen Hintergrund das hat, schauen wir uns hier an. Anschließend folgt ein Beispiel. Hinweis: für dieses Kapitel solltest du das Kapitel zum \highlight{subsubsec:komplexpotenz}{Potenzieren} von komplexen Zahlen gelesen haben.\newline\newline
				Zunächst ist es wieder sehr hilfreich, die komplexe Zahl in der Polarform vorliegen zu haben. Außerdem wissen wir, dass wir beliebig oft um $2\pi$ drehen können, weshalb wir zum Winkel $\phi$ Umdrehungen der Anzahl $k$ hinzufügen: $2\pi k$.
				\begin{flalign*}
					r\cdot e^{i\phi}=r\cdot e^{i(\phi+2\pi k)}&&
				\end{flalign*}
				Wenn wir jetzt die $n$-te Wurzel ziehen sollen, erhalten wir:
				\begin{flalign*}
					\sqrt[n]{r\cdot e^{i(\phi+2\pi k)}}&&
				\end{flalign*}
				Laut Potenzgesetze\index{Potenzgesetz} können wir das auch schreiben als:
				\begin{flalign*}
					&(r\cdot e^{i(\phi+2\pi k)})^{\frac{1}{n}}\\
					=&r^{\frac{1}{n}}\cdot e^{i\frac{1}{n}(\phi+2\pi k)}\\
					=&r^{\frac{1}{n}}\cdot e^{i\left(\frac{\phi}{n}+\frac{2\pi}{n}k\right)}&&
				\end{flalign*}
				Von Interesse ist die letzte Zeile. Das $r$ lassen wir mal außen vor, da es für alle Ergebnisse gleich ist und die folgende Erklärung dem Winkel gilt. Wir haben ja in der Polarform in der Exponentialschreibweise normalerweise zu stehen $i\phi$. Jetzt wo wir unseren Term so umgeformt haben, ist unser Winkel $\phi$ gleich $\frac{\phi}{n}+\frac{2\pi}{n}k$.
				\begin{center}
					\begin{tikzpicture}
					\begin{axis}[
					width=17cm,
					height=10cm,
					xmin=-16, xmax=16,
					ymin=-10, ymax=10,
					axis y line=center,
					axis x line=middle,
					minor tick num=2,
					grid=both,
					grid style={line width=.1pt, draw=gridgray!10},
					major grid style={line width=.2pt,draw=gridgray!50},
					axis equal image,
					xlabel={$\mathfrak{Re}(z)$},
					ylabel={$\mathfrak{Im}(z)$},
					yticklabels={,,},
					xticklabels={,,}
					]
					\draw (axis cs:0,0) circle (3cm);
					\node[anchor=west] at (axis cs:-14,8) {$\phi=\frac{1}{4}\pi$};
					\node[anchor=west] at (axis cs:-14,6) {$n=6$};
					\node[anchor=east] at (axis cs:14,6) {$\textcolor{red}{\frac{\phi}{n}}+\textcolor{blue}{\frac{2\pi}{n}k}$};
					\node[anchor=west, red] at (axis cs:4,2) {$\frac{\phi}{n}$};
					\fill[blue] (axis cs:0,0) ++(405:3cm) circle[radius=2pt] node[above right] {$k=0$};
					\fill[blue] (axis cs:0,0) ++(105:3cm) circle[radius=2pt] node[above left] {$k=1$};
					\fill[blue] (axis cs:0,0) ++(165:3cm) circle[radius=2pt] node[left] {$k=2$};
					\fill[blue] (axis cs:0,0) ++(225:3cm) circle[radius=2pt] node[below left] {$k=3$};
					\fill[blue] (axis cs:0,0) ++(285:3cm) circle[radius=2pt] node[below right] {$k=4$};
					\fill[blue] (axis cs:0,0) ++(345:3cm) circle[radius=2pt] node[right] {$k=5$};
					\addplot [-stealth, red] coordinates { (0,0) (5,5) };
					\draw[stealth-,red] (axis cs:4.5,4.5) arc (45:0:2.7cm);
					\draw[-stealth,blue] (axis cs:4.5,4.5) arc (45:105:2.7cm);
					\draw[-stealth,blue] (axis cs:4,4) arc (45:165:2.4cm);
					\draw[-stealth,blue] (axis cs:3.5,3.5) arc (45:225:2.1cm);
					\draw[-stealth,blue] (axis cs:3,3) arc (45:285:1.8cm);
					\draw[-stealth,blue] (axis cs:2.5,2.5) arc (45:345:1.5cm);
					\end{axis}
					\end{tikzpicture}
				\end{center}
				Was die obige Abbildung bedeutet ist folgendes: Der Term $\textcolor{red}{\frac{\phi}{n}}$ bleibt konstant; er ist unser Startwinkel. In diesem Fall sind das 45 Grad. Jetzt addieren wir noch $\textcolor{blue}{\frac{2\pi}{n}k}$ dazu und das für jedes $k$ zwischen $0$ und $n-1$. In diesem Fall addieren wir jedes Mal 60 Grad drauf. Für $k=1$ hätten wir z.~B. einen Winkel von 105 Grad. So kriegen wir am Ende 6 verschiedene Lösungen, deren Beträge zwar alle gleich sind, aber deren Argumente (Winkel) sich unterscheiden.\newline\newline
				Wie versprochen dazu noch ein Beispiel:
				\begin{tcolorbox}[boxsep=0pt,top=0cm,left=0cm,right=20cm, bottom=0cm,arc=0pt,auto outer arc,colback=white,colframe=white]
					\begin{flalign*}
					\text{Beispiel folgt}
					\end{flalign*}
				\end{tcolorbox}
\pagebreak
\section{Elementare Rechengesetze, -verfahren und -notationen}
\label{sec:rechengesetze}
	\subsection{Brüche dividieren}
		Um zwei Brüche zu dividieren bildet man den Kehrwert \index{Kehrwert}vom Divisor\index{Divisor} und multipliziert diesen mit dem Dividend.\index{Dividend}
		\begin{flalign*}
		\frac{p_1}{q_1} : \frac{p_2}{q_2} = \frac{p_1}{q_1} \cdot \frac{q_2}{p_2}&&
		\end{flalign*}
		\begin{flalign*}
		\bigfrac {\frac{p_1}{q_1}}  {\frac{p_2}{q_2}} = \frac{p_1}{q_1} \cdot \frac{q_2}{p_2}&&
		\end{flalign*}
	\subsection{Lösungsmenge}\index{Lösungsmenge}
		Beispiel 1 ($x^2=-1$):\newline
		$\mathbb{L} = \emptyset$ \newline\newline
		Beispiel 2 ($x^2 = 4$):\newline
		$\mathbb{L} = \{-2;2\}$\newline\newline
		Beispiel 3 ($sin(x) = 0$):\newline
		$\mathbb{L} = \{...;-2\pi;-\pi;0;\pi;2\pi; ...\}$\newline\newline
		Beispiel 4 ($x^2 + y = 5$):\newline
		$\mathbb{L} = \{(x_0;y_0)\in\mathbb{R}^2\mid x^2_0 + y_0 = 5\} = \{(x_0;5-x^2_0) \mid x_0\in\mathbb{R}^2\}$\newline\newline
		In diesem Fall ist die Lösungsmenge die Funktion $y=5-x^2$.\newline
		\makeplot{{5-x^2}}{{$f(x)=5-x^2$}}{{11,3.3}}{-10,10}{-10,10}{-30:30}{330}{17cm,7cm}{smooth}
	\subsection{Normalform}
		Eine Gleichung in der Form $ax^2+bx+c=0$ mit $a\neq0$ und $b,c\in \mathbb{R}$, heißt quadratisch\index{Quadratisch}. Spezial bezeichnet man $x^2+px+q=0$ mir $p,q\in \mathbb{R}$, als quadratische Gleichung in Normalform. \index{Normalform}\newline\newline
		Man kann eine quadratische Gleichung in die Normalform überführen, indem man durch $a$ teilt: $x^2+\frac{b}{a}x+\frac{c}{a}=0$.
		\subsubsection{p-q-Formel}
		\label{subsubsec:pqformel}
		\begin{tcolorbox}[boxsep=0pt,top=.75cm,left=1cm,right=1cm, bottom=.5cm,arc=0pt,auto outer arc,colback=white,colframe=black, enlarge top by=.25cm, enlarge bottom by=.25cm]
			Um die Nullstellen einer quadratischen Gleichung in der Normalform zu finden, kann man die p-q-Formel\index{p-q-Formel} benutzen: $x_{\pm}=-\frac{p}{2}\pm\sqrt{\left(\frac{p}{2}\right)^2-q}$.
		\end{tcolorbox}
		\noindent$D=\left(\frac{p}{2}\right)^2-q$ ist die Diskriminante\index{Diskriminante}. Sie gibt Aufschluss über die Lösungsmenge.
		\begin{flalign*}
		&D>0\Rightarrow \text{Es gibt zwei Lösungen}\\
		&D=0\Rightarrow \text{Es gibt eine Lösung}\\
		&D<0\Rightarrow \text{Es gibt keine Lösungen}&&
		\end{flalign*}
	\subsection{Intervalle}
		\label{subsec:intervalle}
		\textbf{Abgeschlossene Intervalle}\index{Abgeschlossenes Intervall}\newline
		\noindent $[a;b] := \{ x \in \mathbb{R} \mid a \le x \le b \}$\newline\newline
		\textbf{Offene Intervalle}\index{Offenes Intervall}\newline
		\noindent $(a;b) = \: ]a;b[ \: := \{ x \in \mathbb{R} \mid a < x < b \}$\newline\newline
		\textbf{Halboffene Intervalle}\index{Halboffenes Intervall}\newline
		\noindent Rechtsoffen\index{Rechtsoffen}\newline
		$[a;b) = \: [a;b[ \: :=  \{ x \in \mathbb{R} \mid a \le x < b \}$ \newline\newline
		Linksoffen\index{Linksoffen}\newline
		$(a;b] = \: ]a;b] \: :=  \{ x \in \mathbb{R} \mid a < x \le b \}$
	\subsection{Beträge}
		\begin{flalign*}
		\vert a \vert = \genfrac{\{}{.}{0pt}{}{a}{-a}\;\;\genfrac{}{}{0pt}{}{a \ge 0}{a < 0}&&
		\end{flalign*}
		\begin{flalign*}
		\vert -a \vert = \vert a \vert&&
		\end{flalign*}
	\subsection{Binomische Formeln}
	\label{sec:binomischeformeln}\index{Binomische Formel}
	$(a+b)^2 = a^2 + 2ab + b^2$ \newline\newline
	$(a-b)^2 = a^2 - 2ab + b^2$ \newline\newline
	$(a+b)(a-b) = a^2 - b^2$
	\subsection{Euklidischer Algorithmus}
		Der euklidische Algorithmus\index{Euklidischer Algorithmus} findet den größten gemeinsamen Teiler\index{Größter gemeinsamer Teiler} zweier Zahlen. Das eignet sich ausgezeichnet dazu, Brüche zu kürzen\index{Kürzen}. Der vorletzte Rest bevor $R = 0$ eintritt, ist das Ergebnis.
		\begin{flalign*}
		& 2160 : 2592 = 0 \;\;\; R = 2160 \\
		& 2592 : 2160 = 1 \;\;\; R = 432 \\
		& 2160 : 432 = 5 \;\;\; R = 0 &&
		\end{flalign*}
		\begin{flalign*}
		\frac{2592}{2160} = \frac{6 \cdot 432}{5 \cdot 432} = \frac{6}{5}&&
		\end{flalign*}
	\subsection{Potenzgesetze}
	\label{subsec:potenzgesetze}\index{Potenzgesetz}
		\begin{tcolorbox}[boxsep=0pt,top=.35cm,left=1cm,right=1cm, bottom=.75cm,arc=0pt,auto outer arc,colback=white,colframe=black, enlarge top by=.25cm, enlarge bottom by=.25cm]
			\begin{flalign*}
				& a^k \cdot a^m = a^{k+m} \\\\
				& \cfrac{b^k}{b^m} = b^{k-m} \\\\
				& a^k \cdot b^k = (a \cdot b)^k \\\\
				& \cfrac{a^k}{b^k} = \left( \frac{a}{b} \right)^k \\\\
				& (a^k)^m = a^{k \cdot m} && 
			\end{flalign*}
		\end{tcolorbox}
		\noindent Für $a>0$ und jede rationale Zahl\index{Rationale Zahl} $\frac{p}{q}$ (mit $p,q \in \mathbb{Z}$ und $q>0$) ist
		\begin{flalign*}
		a^{\frac{p}{q}} = \sqrt[q]{a^p} = (\sqrt[q]{a})^p && 
		\end{flalign*}
		Beispiel: Bestimmen Sie $m$ und $n$ so, dass gilt: $(9x^7)^2 = mx^n$
		\begin{flalign*}
			(9x^7)^2 = mx^n \\
			81x^{14} = mx^n &&
		\end{flalign*}
		$m = 81$ und $n = 14$
	\subsection{Wurzelgesetze}\index{Wurzelgesetz}
		\begin{tcolorbox}[boxsep=0pt,top=.75cm,left=1cm,right=1cm, bottom=.7cm,arc=0pt,auto outer arc,colback=white,colframe=black, enlarge top by=.25cm, enlarge bottom by=.25cm]
			Für $a,b,c \in \mathbb{R}$ mit $a,b \ge 0, c > 0$ und $m, n \in \mathbb{N}$ gilt
			\begin{flalign*}
				& \sqrt[n]{ab} = \sqrt[n]{a} \cdot \sqrt[n]{b} \\\\
				& \sqrt[n]{\frac{a}{c}} = \frac{\sqrt[n]{a}}{\sqrt[n]{c}} \\\\
				& \sqrt[n]{\sqrt[m]{a}} = \sqrt[n \cdot m]{a} && 
			\end{flalign*}
		\end{tcolorbox}
		\noindent Beispiel 1: Nach der dritten Binomischen Formel\index{Binomische Formel} gilt für $a,b > 0, a \neq b$:
		\begin{flalign*}
		& \cfrac{1}{\sqrt{a}+\sqrt{b}} & \mid \cdot (\sqrt{a}-\sqrt{b}) &&&&&&&&&&&& \\
		= & \cfrac{\sqrt{a}-\sqrt{b}}{(\sqrt{a}+\sqrt{b})(\sqrt{a}-\sqrt{b})} \\ 
		= & \cfrac{\sqrt{a}-\sqrt{b}}{\sqrt{a}^2-\sqrt{b}^2} \\ 
		= & \cfrac{\sqrt{a}-\sqrt{b}}{a-b} &&
		\end{flalign*}
		\hrule
		\noindent\newline\newline Beispiel 2:
		\begin{flalign*}
		& \frac{\sqrt{(1+a^2) \cdot (a-b)^2}}{\sqrt[4]{16(1+a^2)^2}} \\
		= & \sqrt{\frac{(1+a)^2 \cdot (a-b)^2}{\sqrt{16(1+a^2)^2}}} \\ 
		= & \sqrt{\frac{(1+a)^2 \cdot (a-b)^2}{4(1+a)^2}} \\ 
		= & \frac{1}{2}\sqrt{(a-b)^2} \\ 
		= & \frac{\mid a-b \mid}{2} &&
		\end{flalign*}
		\subsubsection{Wurzeltherme vereinfachen (Beispiele)}\index{Vereinfachen}
			\begin{flalign*}
				\sqrt{2}+\dfrac{2}{2\sqrt{2}+3} & = \dfrac{}{}\sqrt{2}+\dfrac{ 2\cdot ( 2\sqrt{2}-] ) }{\left( 2\sqrt{2}+3\right) \left( 2\sqrt{2}-3\right) } \\
				& = \sqrt{2}+\dfrac{4\sqrt{2}-6}{\left( 2\sqrt{2}\right) ^{2}-3^{2}} \\
				& = \sqrt{2}+\dfrac{4\sqrt{2}-6}{-1} \\
				& = \sqrt{2}-4\sqrt{2}+6 \\
				& = 6-3\sqrt{2} &&
			\end{flalign*}
			\hrule
			\begin{flalign*}
			\dfrac{1}{\sqrt{1+x^{2}}-1}-\dfrac{1}{\sqrt{1+x^{2}}+1} & = \dfrac{\sqrt{1+x^{2}}+1}{\left( \sqrt{1+x^{2}}-1\right) \cdot \left( \sqrt{1+x^{2}}+1\right) } - \dfrac{\sqrt{1+x^{2}}-1}{\left( \sqrt{1+x^{2}}+1\right) \cdot \left( \sqrt{1+x^{2}}-1\right) } \\
			& = \dfrac{\left( \sqrt{1+x^{2}}+1\right) -\left( \sqrt{1+x^{2}}-1\right) }{1+x^{2}-1} \\
			& = \frac{2}{x^2} &&
			\end{flalign*}
			\hrule
			\noindent\newline\newline Beispiel 3: Bestimmen Sie $x$ und $y$, sodass $\frac{x}{y}$ vollständig gekürzt\index{Gekürzt} ist.
			\begin{flalign*}
				\dfrac{2\cdot 2^{\frac{5}{2}}}{2^{\frac{1}{4}}}=2^{\frac{x}{y}} \\
				2 \cdot \dfrac{2^{\frac{5}{2}}}{2^{\frac{1}{4}}}=2^{\frac{x}{y}} \\
				2 \cdot 2^{\frac{5}{2}-\frac{1}{4}} = 2^{\frac{x}{y}} \\
				2 \cdot 2^{\frac{9}{4}} = 2^{\frac{x}{y}} \\
				2^{\frac{13}{4}} = 2^{\frac{x}{y}} &&
			\end{flalign*}
			Damit gilt $x = 13$ und $y = 4$.
	\subsection{Logarithmusgesetze}
	\label{subsec:logarithmusgesetze}
		\begin{tcolorbox}[boxsep=0pt,top=1cm,left=1cm,right=1cm, bottom=1cm,arc=0pt,auto outer arc,colback=white,colframe=black, enlarge top by=.25cm, enlarge bottom by=.25cm]
			Die Logarithmusrechnung\index{Logarithmus} dient dazu das $x$ im Term\index{Term} $b^x=a$ zu bestimmen. Es ist damit quasi das Gegenstück zur Potenzrechnung\index{Potenzrechnung}. Rechnen wir z.~B. $9^3$, kommen wir auf $729$. Umgekehrt können wir jetzt aber auch $log_{9}729$ rechnen und kommen auf $3$. Es gibt außerdem die speziellen Notationen\index{Notation} $ln$ und $lg$, die jeweils \textit{natürlicher Logarithmus}\index{Natürlicher Logarithmus} und \textit{dekadischer Logarithmus}\index{Dekadischer Logarithmus} genannt werden.
			\begin{flalign*}
				ln(b):=log_e(b)&&
			\end{flalign*}
			\begin{flalign*}
				lg(b):=log_{10}(b)&&
			\end{flalign*}
			\begin{flalign*}
				log_b(b)=1&&
			\end{flalign*}
			\begin{flalign*}
				log_b(1)=0&&
			\end{flalign*}
			\begin{flalign*}
				log_b(u\cdot v)=log_b(u)+log_b(v)&&
			\end{flalign*}
			\begin{flalign*}
				log_b\left(\frac{u}{v}\right)=log_b(u)-log_b(v)&&
			\end{flalign*}
			\begin{flalign*}
				log_b(u^v)=v\cdot log_b(u)&&
			\end{flalign*}
			\begin{flalign*}
				log_b(\sqrt[u]{v})=\frac{log_b(v)}{u}&&
			\end{flalign*}
			\begin{flalign*}
				log_a(v)=\frac{log_b(v)}{log_b(a)}&&
			\end{flalign*}
		\end{tcolorbox}
\pagebreak
\section{Vereinfachungen zum Lösen von Gleichungen}
	\label{sec:gleichungenvereinfachen}
	\subsection{Quadratische Ergänzung}
		\begin{tcolorbox}[boxsep=0pt,top=.75cm,left=1cm,right=1cm, bottom=.65cm,arc=0pt,auto outer arc,colback=white,colframe=black, enlarge top by=.25cm, enlarge bottom by=.25cm]
			Die äquivalente Umformung\index{Äquivalenzumformung} der quadratischen Gleichung\index{Quadratische Gleichung} in Normalform\index{Normalform} $x^2+px+q=0$ in $\left(x+\frac{p}{2}\right)^2=-q+\left(\frac{p}{2}\right)^2$ wird als quadratische Ergänzung\index{Quadratische Ergänzung} bezeichnet. In anderen Worten fügt man den Term $+\left(\frac{p}{2}\right)^2-\left(\frac{p}{2}\right)^2$ hinzu. Das darf man, da dieser Wert an sich Null ergibt und die Gleichung\index{Gleichung} nicht verändert.
		\end{tcolorbox}
		\noindent Beispiel:
			\begin{flalign*}
		x^2+8x+7 &= 0\\
		x^2+8x+\left(\frac{8}{2}\right)^2-\left(\frac{8}{2}\right)^2+7 &= 0\\
		x^2+8x+4^2-4^2+7 &= 0\\
		(x+4)^2-4^2+7 &= 0\\
		(x+4)^2-9 &= 0\;\;\;\;\;\;\;\;\;\mid+9\\
		(x+4)^2 &= 9\;\;\;\;\;\;\;\;\;\mid\sqrt{\ }\\
		x&=\pm\sqrt{9}-4\\
		\mathbb{L}&=\{-1;-7\}&&
		\end{flalign*}
	\subsection{Faktorisieren}
		Um die Nullstellen\index{Nullstelle} eines Terms\index{Term} zu finden, bietet es sich an, ihn als Produkt\index{Produkt} einfacher Terme zu schreiben, denn ist ein Faktor\index{Faktor} $0$, ist das Produkt ebenfalls $0$. Den Term in so eine Form zu überführen, nennt sich Faktorisieren\index{Faktorisieren}.
		\subsubsection{Faktorisierung durch Ausklammern}
		\label{subsubsec:ausklammern}
			Beispiel:
			\begin{flalign*}
				x^4+2x^3+3x^2 &= 0\\
				x^2(x^2+2x+3) &= 0\\
				x^2 &= 0\;oder\;(x^2+2x+3) = 0\\
				\mathbb{L} &= \{0\}&&
			\end{flalign*}
			Für $x^2+2x+3 = 0$ existiert keine reelle Lösung $\Rightarrow$ \highlight{subsubsec:pqformel}{p-q-Formel}.
		\subsubsection{Faktorisierung mit binomischen Formeln}
			Beispiel:
			\begin{flalign*}
			9x^2+30x+25& = 0\\
			(3x+5)^2& = 0\\
			3x+5&=0\;\;\;\;\;\;\;\;\;\mid-5\\
			3x&=-5\;\;\;\;\;\;\;\;\;\mid:3\\
			x&=-\frac{5}{3}\\
			\mathbb{L}&=\left\{-\frac{5}{3}\right\}&&
			\end{flalign*}
		\subsubsection{Faktorisierung mit dem Satz von Viëta}
			\begin{tcolorbox}[boxsep=0pt,top=.75cm,left=1cm,right=1cm, bottom=.75cm,arc=0pt,auto outer arc,colback=white,colframe=black, enlarge top by=.25cm, enlarge bottom by=.25cm]
			Der Satz von Viëta\index{Satz von Viëta} besagt, dass $x^2+px+q=(x-x_1)\cdot(x-x_2)$ ist. $p$ und $q$ lassen sich auf die Nullstellen zurückführen: $x_1+x_2=-p$ und $x_1\cdot x_2 = q$.
			\end{tcolorbox}
			\noindent Daraus lässt sich $x_2=\frac{q}{x_1}$ ableiten. Wenn man also durch Raten eine Nullstelle\index{Nullstelle} findet, kann man so die andere Nullstelle\index{Nullstelle} auch ganz einfach finden.\newline\newline
			Beispiel (eine Nullstelle\index{Nullstelle} ist $1$, die andere ergibt sich als $\sqrt{2} = \frac{\sqrt{2}}{1}$):
			\begin{flalign*}
				x^2+(\sqrt{2}-1)x-\sqrt{2}&=0\\
				(x-1)\cdot (x+\sqrt{2})&=0\\
				\mathbb{L}=\{1;-\sqrt{2}\}&&
			\end{flalign*}
	\subsection{Substitution}
	\label{subsec:substitution}
		Substitution\index{Substitution} erlaubt es uns manchmal Gleichungen\index{Gleichung} zu vereinfachen\index{Vereinfachen}, um leichter mit ihnen rechnen zu können.\newline\newline
		Beispiel: $x^8-15x^4-16=0$\newline
		Hier bietet es sich an $x^4$ durch $u$ zu ersetzen.
		\begin{flalign*}
		u^2-15u-16&=0\\
		u_{\pm}&=\frac{15}{2}\pm\sqrt{\left(\frac{-15}{2}\right)^2+6}&&
		\end{flalign*}
		Diesen Term\index{Term} wiederum können wir ganz einfach mit der \highlight{subsubsec:pqformel}{p-q-Formel}\index{p-q-Formel} lösen. Dabei erhalten wir $u_+=16$ und $u_-=-1$. Um unsere endgültige Lösungsmenge\index{Lösungsmenge} zu bekommen, müssen wir noch die Resubstitution durchführen\index{Resubstitution}.
		\begin{flalign*}
		x^4&=u_+=16\\
		x_1&=2\\
		x_2&=-2&&
		\end{flalign*}
		Da es kein $x$ gibt, das $x^4=-1$ erfüllt, haben wir bereits unsere komplette Lösungsmenge\index{Lösungsmenge}:\newline\newline$\mathbb{L}=\{-2;2\}$.
\pagebreak
\section{Ungleichungen}
	\subsection{Rechenregeln}
	\label{subsec:unglrechrgl}
		Wenn man eine Ungleichung\index{Ungleichung} mit einer negativen Zahl multipliziert oder durch diese teilt, muss das Vergleichszeichen\index{Vergleichszeichen} umgekehrt werden.
		\begin{tcolorbox}[boxsep=0pt,top=.75cm,left=1cm,right=1cm, bottom=.65cm,arc=0pt,auto outer arc,colback=white,colframe=black, enlarge top by=.25cm, enlarge bottom by=.25cm]
			Für $c<0$ gilt:
			\begin{flalign*}
				&a<b\iff c\cdot  a>c\cdot b\\
				&a<b\iff \frac{a}{c}>\frac{b}{c} &&
			\end{flalign*}
		\end{tcolorbox}
	\subsection{Quadratische Ungleichungen}
		Um die Lösungsmenge\index{Lösungsmenge} einer quadratischen Ungleichung\index{Quadratische Ungleichung} zu finden, formt man die Ungleichung\index{Ungleichung} zunächst so um, dass auf einer Seite $0$ steht. Auf der anderen Seite hat man dann optimalerweise eine quadratische Funktion\index{Quadratische Funktion}. Schauen wir uns mal das Beispiel $x^2>2x+7$ an.
		\makeplot{{x^2-2*x-7}}{{$f(x)=x^2-2x-7$}}{{5,1}}{-10,10}{-10,10}{-30:30}{330}{17cm,7cm}{smooth}
		Wir stellen also zunächst um\index{Umstellen} und erhalten $x^2-2x-7>0$. Daraus ergibt sich auch die Funktion oben. An der Grafik erkennt man sehr gut, was wir eigentlich suchen. Denn unsere Lösungsmenge\index{Lösungsmenge} sind alle $x$, für die $f(x)$ größer als $0$ ist. Und wie kriegen wir das raus? Indem wir die Nullstellen\index{Nullstelle} berechnen. Das Intervall\index{Intervall} von Unendlich\index{Unendlich} bis zur linken Nullstelle ist ein Teil unserer Lösung und der andere ist das Intervall von der rechten Nullstelle bis unendlich. Dabei muss man stets verschiedene Fälle beachten. Für eine nach unten geöffnete Funktion ($-x^2$) suchen wir den Bereich zwischen den Nullstellen. Für eine Funktion oberhalb der $x$-Achse, die keine Nullstellen hat, sind alle reellen Zahlen unsere Lösungsmenge, wohingegen eine Funktion ohne Nullstellen unterhalb der $x$-Achse eine leere Lösungsmenge liefern würde. Eine Funktion mit genau einer Nullstelle liefert hingegen eine Lösungsmenge aller reellen Zahlen\index{Reelle Zahl} außer der Nullstelle. Es gibt je nach Art der Funktion und Vergleichszeichen in unserer Ungleichung viele unterschiedliche Szenarien, weshalb es immer ratsam ist eine Skizze anzufertigen. Für das Beispiel oben können wir die \highlight{subsubsec:pqformel}{p-q-Formel}\index{p-q-Formel} verwenden, um die Nullstellen zu berechnen.
		\begin{flalign*}
		x_{\pm}&=-\frac{-2}{2}\pm \sqrt{\left(\frac{-2}{2}\right)^2+7}\\
		x_1&=1-2\sqrt{2}\\
		x_2&=1+2\sqrt{2}&&
		\end{flalign*}
		Jetzt, wo wir die Nullstellen\index{Nullstelle} haben, ist es nicht schwer die Lösungsmenge\index{Lösungsmenge} anzugeben. Dabei sollte man darauf achten, dass man abgeschlossene\index{Abgeschlossenes Intervall} und offene \highlight{subsec:intervalle}{Intervalle}\index{Offenes Intervall} nicht verwechselt.
		\begin{flalign*}
			\mathbb{L}=\mathbb{R}\setminus \left[1-2\sqrt{2};1+2\sqrt{2}\right]=\left(-\infty;1-2\sqrt{2}\right)\cup\left(1+2\sqrt{2}\right)&&
		\end{flalign*}
	\subsection{Ungleichungen mit Beträgen}
		Das Vorgehen bei Betragsungleichungen\index{Betragsungleichung} ist im Grunde genommen dasselbe Prinzip, wie bei den quadratischen. Schauen wir uns das Beispiel $\vert x \vert -3 < 0$ an.
		\makeplot{{abs(x)-3}}{{$f(x)=\vert x \vert -3$}}{{5,1.75}}{-10,10}{-10,10}{-30:30}{330}{17cm,7cm}{sharp plot}
		Wir erkennen die Nullstellen\index{Nullstelle} in dem Fall sehr leicht. Das sind $-3$ und $3$. Erkennt man das nicht sofort, muss man eine \highlight{subsec:betragsfunktionen}{Fallunterscheidung}\index{Fallunterscheidung} durchführen. Jetzt können wir aber erst mal unsere Lösungsmenge definieren, denn wir wissen, dass wir alle $x$ suchen für die $f(x)<0$ gilt.
		\begin{flalign*}
		\mathbb{L}=(-3;3)&&
		\end{flalign*}
		Hinweis: Wäre unsere Ausgangsungleichung $\vert x \vert -3 \le 0$, sehe unsere Lösungsmenge jetzt so aus:
		\begin{flalign*}
		\mathbb{L}=[-3;3]&&
		\end{flalign*}
	\subsection{Ungleichungen mit Variable im Nenner – Teil I}
		Aus den vorherigen Erklärungen kann man sich herleiten, wie man das macht. Deshalb ist hier nur noch mal ein erklärendes Beispiel: $2\le \frac{14}{\vert 2x+5\vert}$.
		\makeplot{{(14/abs(2*x+5))-2}}{{$f(x)=\dfrac{14}{\vert 2x+5\vert}-2$}}{{10,3.5}}{-10,10}{-10,10}{-30:30}{200}{17cm,7cm}{sharp plot}
		Wichtig ist, dass wir zunächst alle $x$ ausschließen, für die im Nenner\index{Nenner} $0$ rauskommt. In diesem Fall ist dass $-\frac{5}{2}$.
		\begin{flalign*}
			2&\le \frac{14}{\vert 2x+5\vert}\\
			2\vert 2x+5\vert&\le 14\\
			\vert 2x+5\vert&\le 7&&
		\end{flalign*}
		Fall 1: $2x+5>0$
		\begin{flalign*}
		2x+5&\le 7\\
		2x&\le 2\\
		x&\le 1&&
		\end{flalign*}
		Fall 2: $2x+5<0$
		\begin{flalign*}
		-2x-5&\le 7\\
		-2x&\le 12\\
		x&\ge -6&&
		\end{flalign*}
		\begin{flalign*}
		\mathbb{L}=\left[-6;-\frac{5}{2}\right)\cup\left(-\frac{5}{2};1\right]&&
		\end{flalign*}
	\subsection{Ungleichungen mit Variable im Nenner – Teil II}
		Wenn wir uns an die \highlight{subsec:unglrechrgl}{Rechenregeln}\index{Rechenregel} für Ungleichungen\index{Ungleichung} erinnern, könnte man sich fragen, was passiert, wenn der Nenner mit einer Variable sowohl positiv\index{Positiv} als auch negativ\index{Negativ} sein kann. Denn wenn wir mit einer negativen Zahl multiplizieren, müssten wir das Vorzeichen\index{Vorzeichen} umkehren. Hier muss man wieder verschiedene Fälle unterscheiden.\newline\newline
		Beispiel: $\frac{1}{x-2}\le-x$
		\makeplot{{(1/(x-2))+x}}{{$f(x)=\dfrac{1}{x-2}+x$}}{{11.1,0.65}}{-10,10}{-10,10}{-20:20}{350}{17cm,7cm}{sharp plot}
		Der Fall $x=2$ ist aufgrund des $x$ im Nenner wieder auszuschließen.
		Fall 1: $x>2$
		\begin{flalign*}
			\frac{1}{x-2}&\le-x\\
			1&\le -x(x-2)\\
			1&\le -x^2+2x\\
			x^2-2x+1&\le 0\\
			(x-1)^2&\le 0&&
		\end{flalign*}
		Dieser Fall gilt für $x=1$. Das widerspricht allerdings der Bedingung $x>2$ und das Ergebnis ist entsprechend nicht Teil unserer Lösungsmenge.\newline\newline
		Fall 2: $x<2$
		\begin{flalign*}
			\frac{1}{x-2}&\le-x\\
			1&\ge -x(x-2)\\
			1&\ge -x^2+2x\\
			x^2-2x+1&\ge 0\\
			(x-1)^2&\ge 0&&
		\end{flalign*}
		Dieser Fall\index{Fall} ist für alle $x$ erfüllt, daher gehören alle $x<2$ zur Lösungsmenge.
		\begin{flalign*}
			\mathbb{L}=(-\infty;2)&&
		\end{flalign*}\hrule
		\noindent\newline\newline Beispiel: $\frac{1}{x-9}\le8$
		\makeplot{{(1/(x-9))-8}}{{$f(x)=\dfrac{1}{x-9}-8$}}{{11,1.6}}{-2,16}{-20,5}{-30:60}{330}{17cm,7cm}{smooth}
		Fall 1: $x>9$
		\begin{flalign*}
			\frac{1}{x-9}&\le 8\\
			1&\le 8x-72\\
			8x&\ge 73\\
			x&\ge \frac{73}{8}&&
		\end{flalign*}
		Fall 2: $x<9$
		\begin{flalign*}
			\frac{1}{x-9}&\le 8\\
			1&\ge 8x-72\\
			8x&\le 73\\
			x&\le \frac{73}{8}&&
		\end{flalign*}
		\begin{flalign*}
			\mathbb{L}=(-\infty;9)\cup \left[\frac{73}{8};\infty\right)&&
		\end{flalign*}
\pagebreak
	\section{Lineare Gleichungssysteme}
	\label{sec:gleichungssysteme}
		Ein Gleichungssystem\index{Gleichungssystem} ist eine Sammlung an Gleichungen, für die man eine gemeinsame Lösung sucht. Für das Beispiel unten, ist die Lösung $x=2,y=3,z=-4$ oder anders ausgedrückt $\mathbb{L}=\{(2;3;-4)\}$. Wie man darauf kommt, wird unten erklärt.
		\begin{flalign*}
			(I)&&x+2y+z&=4\\
			(II)&&x-y+ \frac{3}{2}z&=-7\\
			(III)&&-4x+2y&=-2&&&&&&&&
		\end{flalign*}
		\subsection{Einsetzungsverfahren}
			Eine Möglichkeit hat man, wenn man eine Funktion nach einer beliebigen Variable umstellt\index{Umstellen} und diese dann in einer anderen Funktion einsetzt.\newline\newline
			$(III)$
			\begin{flalign*}
				-4x+2y&=-2\\
				-4x&=-2-2y\\
				x&=\frac{1}{2}+\frac{1}{2}y&&
			\end{flalign*}
			$(I)$
			\begin{flalign*}
				x+2y+z&=4\\
				\frac{1}{2}+\frac{1}{2}y+2y+z&=4\\
				\frac{1}{2}+\frac{5}{2}y+z&=4\\
				z&=3,5-\frac{5}{2}y&&
			\end{flalign*}
			$(II)$
			\begin{flalign*}
				x-y+ \frac{3}{2}z&=-7\\
				\frac{1}{2}+\frac{1}{2}y-y+ \frac{3}{2}\left(3,5-\frac{5}{2}y\right)&=-7\\
				\frac{1}{2}+\frac{1}{2}y-y+ 5,25-\frac{15}{4}y&=-7\\
				5,75-\frac{1}{2}y-\frac{15}{4}y&=-7\\
				-\frac{17}{4}y&=-\frac{51}{4}\\
				y&=3&&
			\end{flalign*}
			$(III)$
			\begin{flalign*}
				-4x+2y&=-2\\
				-4x+2\cdot 3&=-2\\
				-4x+6&=-2\\
				-4x&=-8\\
				x&=2&&
			\end{flalign*}
			$(I)$
			\begin{flalign*}
				x+2y+z&=4\\
				2+2\cdot 3+z&=4\\
				z&=-4&&
			\end{flalign*}
		\subsection{Additionsverfahren}
			Eine andere Möglichkeit ist es, eine oder mehrere Gleichungen mit einer Zahl zu multiplizieren, sodass eine Variable entfällt, wenn man zwei Gleichungen addiert.\newline\newline
			$(I)-(III)$
			\begin{flalign*}
			5x+z&=6\\
			z&=6-5x&&
			\end{flalign*}
			$(I)+2(II)$
			\begin{flalign*}
			3x+4z&=-10\\
			3x+4(6-5x)&=-10\\
			3x+24-20x&=-10\\
			-17x&=-34\\
			x&=2&&
			\end{flalign*}
			$(III)$
			\begin{flalign*}
			-4x+2y&=-2\\
			-4\cdot 2+2y&=-2\\
			-8+2y&=-2\\
			2y&=6\\
			y&=3&&
			\end{flalign*}
			$(I)$
			\begin{flalign*}
			x+2y+z&=4\\
			2+2\cdot 3+z&=4\\
			z&=-4&&
			\end{flalign*}
		Hinweis: sind zwei Gleichungen identisch, so gibt es unendlich viele Lösungen und man muss nur die entsprechende Notation\index{Notation} für die Lösungsmenge kennen.
		\begin{flalign*}
			(I)&&-4x-2y&=-14\\
			(II)&&4x+2y&=14&&&&&&&&&&&&
		\end{flalign*}
		\begin{flalign*}
			\mathbb{L}=\{(x;7-2x)\mid x \in \mathbb{R}\}&&
		\end{flalign*}
		\subsection{Gauß-Verfahren}
			Das Gauß-Verfahren\index{Gauß-Verfahren} ist eine bestimmte Vorgehensweise fürs Additionsverfahrens\index{Additionsverfahren}, bei dem man die Gleichungen\index{Gleichung} so umformt, dass man das LGS\index{Lineares Gleichungssystem} in die Stufenform\index{Stufenform} bringt und es einfach lösen kann.
			\begin{flalign*}
			(I)&&x+2y+z&=4\\
			(II)&&x-y+ \frac{3}{2}z&=-7\;\;\;\;\;\;\mid -\frac{3}{2}(I)\\
			(III)&&-4x+2y&=-2&&&&&&&&&&&&
			\end{flalign*}
			\begin{flalign*}
			(I)&&x+2y+z&=4\\
			(II)&&-\frac{1}{2}x-4y&=-13\\
			(III)&&-4x+2y&=-2\;\;\;\;\;\;\mid +\frac{1}{2}(II)&&&&&&&&&&&&
			\end{flalign*}
			\begin{flalign*}
			(I)&&x+2y+z&=4\\
			(II)&&-\frac{1}{2}x-4y&=-13\\
			(III)&&-\frac{17}{4}x&=-\frac{17}{2}&&&&&&&&&&&&
			\end{flalign*}
			$(III)$
			\begin{flalign*}
			-\frac{17}{4}x&=-\frac{17}{2}\\
			x&=2&&
			\end{flalign*}
			$(II)$
			\begin{flalign*}
			-\frac{1}{2}x-4y&=-13\\
			-1-4y&=-13\\
			-4y&=-12\\
			y&=3&&
			\end{flalign*}
			$(I)$
			\begin{flalign*}
			x+2y+z&=4\\
			2+6+z&=4\\
			z&=-4&
			\end{flalign*}
		\subsection{LGS mit Parameter}
			Kommt in einem LGS\index{Lineares Gleichungssystem} ein Parameter\index{Parameter} vor, dann muss man eine Fallunterscheidung\index{Fallunterscheidung} vornehmen und den Parameter in die Lösungsmenge\index{Lösungsmenge} mit einbeziehen.
			\begin{flalign*}
			(I)&&x-2y&=0\\
			(II)&&y+\frac{1}{3}z&=-1\\
			(III)&&(a-3)y&=1&&&&&&&&&&&&
			\end{flalign*}
			Wenn $a=3$, dann kommt bei der letzten Gleichung\index{Gleichung} $0=1$ raus. Dadurch können wir schon mal sagen, was die Lösungsmenge\index{Lösungsmenge} für den Fall $a=3$ ist.
			\begin{flalign*}
				\mathbb{L}=\emptyset, falls\;a=3&&
			\end{flalign*}
			Als nächstes schauen wir uns den Fall\index{Fall} $a\neq0$ an.\newline\newline
			$(III)$
			\begin{flalign*}
				(a-3)y&=1\\
				y&=\frac{1}{a-3}&&
			\end{flalign*}
			$(II)$
			\begin{flalign*}
				\frac{1}{a-3}+\frac{1}{3}z&=-1\\
				\frac{1}{3}z&=-\frac{1}{a-3}-1\\
				z&=3\left(-\frac{1}{a-3}-1\right)\\
				z&=-\frac{3}{a-3}-\frac{3a-9}{a-3}\\
				z&=\frac{6-3a}{a-3}&&
			\end{flalign*}
			$(I)$
			\begin{flalign*}
				x-2y&=0\\
				x-2\left(\frac{1}{a-3}\right)&=0\\
				x&=\frac{2}{a-3}&&
			\end{flalign*}
			\begin{flalign*}
			\mathbb{L}=\left\{\left(\frac{2}{a-3};\frac{1}{a-3};\frac{6-3a}{a-3}\right)\right\},\;falls\;a\neq 3&&
			\end{flalign*}
\pagebreak
	\section{Geometrie}
	\label{sec:geometrie}
		\subsection{Rechtwinklige Dreiecke}
			\begin{center}
				\begin{tikzpicture}[scale=1.25]%,cap=round,>=latex]
					\coordinate [label=left:$B$] (B) at (-2cm,-1.cm);
					\coordinate [label=right:$A$] (A) at (2.2cm,-1.0cm);
					\coordinate [label=above:$C$] (C) at (1cm,1.0cm);
					\coordinate [label=above:$S$] (A-|B) at (0.8cm,-1.0cm);
					\draw (A) -- node[sloped,below] {\;\;\;\;\;\;\;\;\;\;q\;\;\;\;\;\;\;\;\;\;\;\;\;\;\;\;\;\;\;\;p} (B) -- node[sloped,above] {a} (C) -- node[sloped,above] {b} (A);
					\draw[dashed, opacity=.4] (C) -- (C|-A) ;
					\tikzset{/tkzmkangle/mark=none}
					\tkzMarkAngle[size=0.65cm](B,C,A)
					
					\tkzMarkAngle[size=0.8cm](C,A,B)
					\tkzLabelAngle[pos = 0.5](C,A,B){$\alpha$}
					
					\tkzMarkAngle[size=1cm](A,B,C)
					\tkzLabelAngle[pos = 0.7](A,B,C){$\beta$}
				\end{tikzpicture}
				\hspace{2cm}
				\begin{tikzpicture}[scale=1.25]%,cap=round,>=latex]
					\coordinate (B) at (-2cm,-1.cm);
					\coordinate (A) at (2.2cm,-1.0cm);
					\coordinate (C) at (1cm,1.0cm);
					\draw (A) -- node[below] {Hypotenuse} (B) -- node[sloped,above] {Gegenkathete von $\alpha$} (C) -- node[sloped,above] {Ankathete von $\alpha$} (A);
					\tikzset{/tkzmkangle/mark=none}
					\tkzMarkAngle[size=0.65cm](B,C,A)
					\tkzLabelAngle[pos = 0.4](B,C,A){$\gamma$}
					
					\tkzMarkAngle[size=0.8cm](C,A,B)
					\tkzLabelAngle[pos = 0.5](C,A,B){$\alpha$}
					
					\tkzMarkAngle[size=1cm](A,B,C)
					\tkzLabelAngle[pos = 0.7](A,B,C){$\beta$}
				\end{tikzpicture}
			\end{center}
			\begin{tcolorbox}[boxsep=0pt,top=1cm,left=1cm,right=1cm, bottom=.75cm,arc=0pt,auto outer arc,colback=white,colframe=black, enlarge top by=.25cm, enlarge bottom by=.25cm]
				\subsubsection{Kathetensatz}
				Im rechtwinkligen Dreieck\index{Rechtwinkliges Dreieck} ist das Quadrat\index{Quadrat} über einer Kathete\index{Kathete} flächengleich zu dem Rechteck\index{Rechteck} aus der Hypotenuse\index{Hypotenuse} und dem der Kathete anliegenden Hypotenusenabschnitt\index{Hypotenusenabschnitt}.
				\begin{flalign*}
				b^2=p\cdot c\\
				a^2=q\cdot c&&
				\end{flalign*}
			\end{tcolorbox}
			\begin{tcolorbox}[boxsep=0pt,top=1cm,left=1cm,right=1cm, bottom=.75cm,arc=0pt,auto outer arc,colback=white,colframe=black, enlarge top by=.25cm, enlarge bottom by=.25cm]
				\subsubsection{Höhensatz}
				Im rechtwinkligen Dreieck\index{Rechtwinkliges Dreieck} ist das Quadrat\index{Quadrat} über der Höhe flächengleich zu dem Rechteck aus den beiden Hypotenusenabschnitten\index{Hypotenusenabschnitt}.
				\begin{flalign*}
				h^2=p\cdot q&&
				\end{flalign*}
			\end{tcolorbox}
				\begin{tcolorbox}[boxsep=0pt,top=1cm,left=1cm,right=1cm, bottom=.75cm,arc=0pt,auto outer arc,colback=white,colframe=black, enlarge top by=.25cm, enlarge bottom by=.25cm]
				\subsubsection{Sinus, Kosinus und Tangens}
				Sinus\index{Sinus}, Kosinus\index{Kosinus} und Tangens\index{Tangens} ordnen einem Winkel im rechtwinkligen Dreieck\index{Rechtwinkliges Dreieck} die Längenverhältnisse der Katheten\index{Kathete} und Hypotenuse\index{Hypotenuse} zu.\newline
				\begin{flalign*}
				&\sin(\alpha)=\frac{a}{c}=\frac{Gegenkathete\;von\;\alpha}{Hypotenuse}\\\\
				&\cos(\alpha)=\frac{b}{c}=\frac{Ankathete\;von\;\alpha}{Hypothenuse}\\\\
				&\tan(\alpha)=\frac{a}{b}=\frac{Gegenkathete\;von\;\alpha}{Ankathete\;von\;\alpha}=\frac{\sin(\alpha)}{\cos(\alpha)}&&
				\end{flalign*}
			\end{tcolorbox}
			\noindent Um mit \highlight{subsec:trigonometrisch}{trigonometrischen Funktionen}\index{Trigonometrische Funktion} umgehen zu können, hilft es außerdem einige Regeln zu kennen, die beim Vereinfachen\index{Vereinfachen} und Umformen von trigonometrischen Termen helfen.
			\begin{tcolorbox}[boxsep=0pt,top=1cm,left=1cm,right=1cm, bottom=.75cm,arc=0pt,auto outer arc,colback=white,colframe=black, enlarge top by=.25cm, enlarge bottom by=.25cm]
				\textbf{Sinus, Kosinus und Tangens Umformen und Vereinfachen}
				\begin{multicols}{2}
					\noindent\begin{flalign*}
					&\cos(-x)=\cos(-x)\\
					&\sin(-x)=-\sin(x)\\
					&\tan(-x)=-\tan(x)\\
					&\cos\left(x+\frac{\pi}{2}\right)=-\sin(x)\\
					&\sin\left(x+\frac{\pi}{2}\right)=\cos(x)&&
					\end{flalign*}
					\begin{flalign*}
					&\tan\left(x+\frac{\pi}{2}\right)=\frac{1}{\tan(x)}\\
					&\cos(x+\pi)=-\cos(\pi)\\
					&\sin(x+\pi)=-\sin(x)\\
					&\tan(x+\pi)=\tan(x)&&
					\end{flalign*}
				\end{multicols}
			\end{tcolorbox}
		\subsection{Rechnen mit Flächen (Formeln)}
			\begin{tcolorbox}[boxsep=0pt,top=1cm,left=1cm,right=1cm, bottom=.75cm,arc=0pt,auto outer arc,colback=white,colframe=black, enlarge top by=.25cm, enlarge bottom by=.25cm]
				\subsubsection{Dreieck}
				Für ein Dreieck\index{Dreieck} mit der Grundseite\index{Grundseite} $c$ und der Höhe\index{Höhe} $h_c$ gilt:
				\begin{flalign*}
				&F=\frac{1}{2}\cdot c \cdot h_c&&
				\end{flalign*}
			\end{tcolorbox}
			\begin{tcolorbox}[boxsep=0pt,top=1cm,left=1cm,right=1cm, bottom=.75cm,arc=0pt,auto outer arc,colback=white,colframe=black, enlarge top by=.25cm, enlarge bottom by=.25cm]
				\subsubsection{Kreis}
				Für einen Kreis\index{Kreis} mit dem Radius\index{Radius} $r$, dem Umfang\index{Umfang} $U$ und der Fläche\index{Fläche} $F$ gilt:
				\begin{flalign*}
				&U=2\pi r\\\\
				&F=\pi r^2&&
				\end{flalign*}
				Für einen Kreissektor\index{Kreissektor} mit dem Radius\index{Radius} $r$, der Bogenlänge\index{Bogenlänge} $b$, der Fläche\index{Fläche} $F$ und dem Winkel\index{Winkel} $\alpha$ gilt:
				\begin{flalign*}
				&F=\frac{br}{2}&&
				\end{flalign*}
				Für ein Kreissegment\index{Kreissegment} mit dem Radius\index{Radius} $r$, der Bogenlänge\index{Bogenlänge} $b$, der Fläche\index{Fläche} $F$ und dem Winkel\index{Winkel} $\alpha$ gilt:
				\begin{flalign*}
				&F=\frac{br}{2}-\frac{1}{2}r^2\cdot \sin(\alpha)&&
				\end{flalign*}
			\end{tcolorbox}
		\subsection{Rechnen mit Körpern (Formeln)}
		\begin{tcolorbox}[boxsep=0pt,top=1cm,left=1cm,right=1cm, bottom=.75cm,arc=0pt,auto outer arc,colback=white,colframe=black, enlarge top by=.25cm, enlarge bottom by=.25cm]
			\subsubsection{Prisma}
			Für ein Prisma\index{Prisma} mit der Mantelfläche\index{Mantelfläche} $M$, der Grundfläche\index{Grundfläche} $A$, dem Grundflächenumfang\index{Grundflächenumfang} $U$, der Oberfläche\index{Oberfläche} $O$, dem Volumen\index{Volumen} $V$ und der Höhe\index{Höhe} $h$ gilt:
			\begin{flalign*}
			&V=A\cdot h\\\\
			&M=U\cdot h\\\\
			&O=2\cdot A+M&&
			\end{flalign*}
		\end{tcolorbox}
		\begin{tcolorbox}[boxsep=0pt,top=1cm,left=1cm,right=1cm, bottom=.75cm,arc=0pt,auto outer arc,colback=white,colframe=black, enlarge top by=.25cm, enlarge bottom by=.25cm]
			\subsubsection{Pyramide}
			Für eine Pyramide\index{Pyramide} mit der Mantelfläche\index{Mantelfläche} $M$, der Grundfläche\index{Grundfläche} $A$, der Oberfläche\index{Oberfläche} $O$, dem Volumen\index{Volumen} $V$ und der Höhe\index{Höhe} $h$ gilt:
			\begin{flalign*}
			&V=\frac{1}{3}\cdot A \cdot h\\\\
			&O=2\cdot A+M&&
			\end{flalign*}
		\end{tcolorbox}
		\begin{tcolorbox}[boxsep=0pt,top=1cm,left=1cm,right=1cm, bottom=.75cm,arc=0pt,auto outer arc,colback=white,colframe=black, enlarge top by=.25cm, enlarge bottom by=.25cm]
			\subsubsection{Zylinder}
			Für einen Zylinder\index{Zylinder} mit der Grundfläche\index{Grundfläche} $A$, dem Radius\index{Radius} der Grundfläche $r$, der Oberfläche\index{Oberfläche} $O$, dem Volumen\index{Volumen} $V$ und der Höhe\index{Höhe} $h$ gilt:
			\begin{flalign*}
			V=\pi \cdot r^2\cdot h&&
			\end{flalign*}
			Für einen geraden Zylinder\index{Gerader Zylinder} gilt außerdem:
			\begin{flalign*}
				O=2\pi r\cdot (r+h)&&
			\end{flalign*}
		\end{tcolorbox}		
		\begin{tcolorbox}[boxsep=0pt,top=1cm,left=1cm,right=1cm, bottom=.75cm,arc=0pt,auto outer arc,colback=white,colframe=black, enlarge top by=.25cm, enlarge bottom by=.25cm]
			\subsubsection{Kegel}
			Für einen Kegel\index{Kegel} mit der Grundfläche\index{Grundfläche} $A$, dem Radius\index{Radius} der Grundfläche $r$, der Oberfläche\index{Oberfläche} $O$, dem Volumen\index{Volumen} $V$, dem Abstand\index{Abstand} der Spitze zu einem Punkt der Kreislinie\index{Kreislinie} $s$ und der Höhe\index{Höhe} $h$ gilt:
			\begin{flalign*}
			V=\frac{1}{3}\pi\cdot r^2 \cdot h&&
			\end{flalign*}
			Für einen geraden Kegel\index{Gerader Kegel} gilt außerdem:
			\begin{flalign*}
			&s=\sqrt{h^2+r^2}\\\\
			&O=\pi r \cdot (r+s)&&
			\end{flalign*}
		\end{tcolorbox}
	\pagebreak
	\section{Vektorgeometrie}
	\label{sec:vektorgeometrie}
		Vektoren kommen häufig in der Physik zum Einsatz, denn oftmals interessieren wir uns nicht nur für eine Größe an sich, sondern auch für ihre Richtung\index{Richtung}. Vektoren werden durch Pfeile gekennzeichnet $\vec v$, ihr Betrag $\vert\vec v\vert$ gibt uns ihre Länge. Wenn zwei gleichlange Vektoren in genau die entgegengesetzte Richtung zeigen ($\vec v$ und $-\vec v$), sprechen wir von \textit{Gegenvektoren}\index{Gegenvektor}. Ein Vektor mit der Länge Null, wird auch als \textit{Nullvektor}\index{Nullvektor} bezeichnet. Neben Vektoren, sind Pfeile ein wichtiger Begriff. Diese werden definiert durch ihren Anfangspunkt sowie, ihrem Endpunkt bzw. ihrer Länge und Richtung. Alle Pfeile der selben Länge und Richtung können in einer \textit{Pfeilklasse}\index{Pfeilklasse} zusammengefasst werden. Vektoren können als Pfeilklassen interpretiert werden, denn sie werden nicht durch ihren Anfangspunkt definiert, sondern nur durch Richtung und Länge. Vektoren können jedoch einen festen Anfangspunkt besitzen. In diesem Fall spricht man von \textit{gebundenen Vektoren}\index{Gebundener Vektor}, andernfalls von \textit{freien Vektoren}\index{Freier Vektor}. Außerdem existiert eine besondere Art von gebundenen Vektoren, die sogenannten \textit{Ortsvektoren}\index{Ortsvektor}, die ihren Anfangspunkt im Koordinatenursprung haben. Der Koordinatenursprung\index{Koordinatenursprung} wird mit $O=(0;0)$ bezeichnet.
		\begin{tcolorbox}[boxsep=0pt,top=.75cm,left=1cm,right=1cm, bottom=.75cm,arc=0pt,auto outer arc,colback=white,colframe=black, enlarge top by=.25cm, enlarge bottom by=.25cm]
			Vektoren werden folgendermaßen notiert:
			\begin{flalign*}
				&\text{2D: }\overrightarrow{PQ}=\begin{pmatrix}Q_x-P_x\\Q_y-P_y\end{pmatrix}=\begin{pmatrix}x\\y\end{pmatrix}\\\\
				&\text{3D: }\overrightarrow{PQ}=\begin{pmatrix}Q_x-P_x\\Q_y-P_y\\Q_z-P_z\end{pmatrix}=\begin{pmatrix}x\\y\\z\end{pmatrix}&&
			\end{flalign*}
			Diese Notation dient dazu sie von Punkten zu unterscheiden. Man spricht hierbei von \textit{Spaltenvektoren}\index{Spaltenvektor}.
		\end{tcolorbox}
		\noindent Die Länge eines Vektors können wir mithilfe des Satz des Pythagoras berechnen, denn wir können einer Vektor auch als Hypotenuse\index{Hypotenuse} eines rechtwinkligen Dreiecks sehen, bei dem die Katheten die $x$- und $y$-Werte sind.
		\begin{tcolorbox}[boxsep=0pt,top=.75cm,left=1cm,right=1cm, bottom=.75cm,arc=0pt,auto outer arc,colback=white,colframe=black, enlarge top by=.25cm, enlarge bottom by=.25cm]
			\begin{multicols}{2}
				Für die Berechnung der Länge eines Vektors gelten folgende Formeln:
				\vspace{.3cm}
				\begin{flalign*}
				&\text{2D: }\vert\vec v\vert = \sqrt{v_x^2+v_y^2}\\\\
				&\text{3D: }\vert\vec v\vert = \sqrt{v_x^2+v_y^2+v_z^2}&&
				\end{flalign*}
				\newline
				\begin{tikzpicture}
				\begin{axis}[
				width=8cm,
				height=6cm,
				xmin=0, xmax=16,
				ymin=0, ymax=10,
				axis y line=center,
				axis x line=middle,
				ticklabel style={fill=white},
				minor tick num=2,
				grid=both,
				grid style={line width=.1pt, draw=gridgray!10},
				major grid style={line width=.2pt,draw=gridgray!50},
				axis equal image,
				yticklabels={,,},
				xticklabels={,,}
				]
				\addplot [-stealth, thick, teal] coordinates { (6,2) (14,2)};
				\addplot [-stealth, thick, red] coordinates { (14,2) (14,8)};
				\addplot [-stealth, thick, blue] coordinates { (6,2) (14,8)};
				\end{axis}
				\node at (2.5,2.75) {$\textcolor{blue}{\left\vert\begin{pmatrix}\textcolor{teal}{8}\\\textcolor{red}{6}\end{pmatrix}\right\vert}=\sqrt{\textcolor{teal}{8}^2+\textcolor{red}{6}^2}=\textcolor{blue}{10}$};
				\node at (4.5,1.2) {$\textcolor{teal}{8}$};
				\node at (6,2) {$\textcolor{red}{6}$};
				\end{tikzpicture}
			\end{multicols}
		\end{tcolorbox}
		\subsection{Rechnen mit Vektoren}
			Wenn wir Vektoren addieren wollen, können wir das ganz einfach tun, indem wir ihre jeweiligen Werte miteinander addieren. Die untere Abbildung zeigt, dass es egal ist, in welcher Reihenfolge wir das tun und auch, dass die Position von Vektoren keine Rolle spielt.
			\begin{center}
				\begin{tikzpicture}
				\begin{axis}[
				width=17cm,
				height=7cm,
				xmin=0, xmax=28,
				ymin=0, ymax=10,
				axis y line=center,
				axis x line=middle,
				ticklabel style={fill=white},
				minor tick num=2,
				grid=both,
				grid style={line width=.1pt, draw=gridgray!10},
				major grid style={line width=.2pt,draw=gridgray!50},
				axis equal image,
				yticklabels={,,},
				xticklabels={,,}
				]
				\addplot [-stealth, thick, blue] coordinates { (11,2) (15,2)};
				\addplot [-stealth, thick, blue] coordinates { (13,8) (17,8)};
				\addplot [-stealth, thick, red] coordinates { (15,2) (17,8)};
				\addplot [-stealth, thick, red] coordinates { (11,2) (13,8)};
				\addplot [-stealth, thick, teal] coordinates { (11,2) (17,8)};
				\end{axis}
				\node at (3.75,3.75) {$\textcolor{blue}{\begin{pmatrix}4\\0\end{pmatrix}}+\textcolor{red}{\begin{pmatrix}2\\6\end{pmatrix}}=\textcolor{teal}{\begin{pmatrix}6\\6\end{pmatrix}}$};
				\end{tikzpicture}
			\end{center}
			Ebenso können wir Vektoren ganz einfach mit einer Zahl (Skalar\index{Skalar}) multiplizieren. Dabei spricht man von \textit{skalarer Multiplikation}\index{skalare Multiplikation}. Dabei können sagen, dass zwei Vektoren parallel sind, wenn es ein $\lambda$ gibt, welches $\lambda\cdot\vec c=\vec w$ erfüllt. Das gilt jedoch nicht für den Nullvektor\index{Nullvektor}.
			\begin{center}
				\begin{tikzpicture}
				\begin{axis}[
				width=17cm,
				height=7cm,
				xmin=0, xmax=28,
				ymin=0, ymax=10,
				axis y line=center,
				axis x line=middle,
				ticklabel style={fill=white},
				minor tick num=2,
				grid=both,
				grid style={line width=.1pt, draw=gridgray!10},
				major grid style={line width=.2pt,draw=gridgray!50},
				axis equal image,
				yticklabels={,,},
				xticklabels={,,}
				]
				\addplot [|-stealth, thick, red] coordinates { (3,2) (9,6.5)};
				\addplot [-stealth, thick, blue] coordinates { (3,2) (7,5)};
				\addplot [-stealth, thick, teal] coordinates { (20,5) (12,8)};
				\addplot [|-stealth, thick, violet] coordinates { (20,5) (24,3.5)};
				\end{axis}
				\node at (2.5,3.75) {$\dfrac{3}{2}\cdot\textcolor{blue}{\begin{pmatrix}4\\3\end{pmatrix}}=\textcolor{red}{\begin{pmatrix}6\\4,5\end{pmatrix}}$};
				\node at (9,1.6) {$-\dfrac{1}{2}\cdot\textcolor{teal}{\begin{pmatrix}-8\\3\end{pmatrix}}=\textcolor{violet}{\begin{pmatrix}4\\-1,5\end{pmatrix}}$};
				\end{tikzpicture}
			\end{center}
			Neben der Multiplikation mit einer Zahl, können wir auch zwei Vektoren miteinander multiplizieren. Das ist das Skalarprodukt\index{Skalarprodukt}, welches man nicht mit skalarer Multiplikation verwechseln sollte. Das Skalarprodukt kann man benutzen, um die Länge von Vektoren sowie, den Winkel zwischen ihnen zu bestimmen. Insbesondere gilt: Wenn das Skalarprodukt gleich Null ist, dann haben wir einen rechten Winkel und die Vektoren sind orthogonal\index{Orthogonal} bzw. liegen senkrecht aufeinander. Das Skalarprodukt heißt übrigens so, weil unser Ergebnis ein Skalar\index{Skalar} ist.
			\begin{center}
				\begin{tikzpicture}
				\begin{axis}[
				width=17cm,
				height=7cm,
				xmin=0, xmax=28,
				ymin=0, ymax=10,
				axis y line=center,
				axis x line=middle,
				ticklabel style={fill=white},
				minor tick num=2,
				grid=both,
				grid style={line width=.1pt, draw=gridgray!10},
				major grid style={line width=.2pt,draw=gridgray!50},
				axis equal image,
				yticklabels={,,},
				xticklabels={,,}
				]
				\draw (axis cs:2.9,2)arc[radius=.5cm,start angle=0,end angle=90];
				\addplot [-stealth, thick, red] coordinates { (2,2) (2,4)};
				\addplot [-stealth, thick, blue] coordinates { (2,2) (7,2)};
				\addplot [-stealth, thick, teal] coordinates { (20,5) (12,8)};
				\addplot [-stealth, thick, violet] coordinates { (20,5) (24,7)};
				\end{axis}
				\node at (2.5,3.75) {$\textcolor{blue}{\begin{pmatrix}5\\0\end{pmatrix}}\cdot\textcolor{red}{\begin{pmatrix}0\\2\end{pmatrix}}=0$};
				\node at (9,1.6) {$\textcolor{teal}{\begin{pmatrix}-8\\3\end{pmatrix}}\cdot\textcolor{violet}{\begin{pmatrix}4\\1,5\end{pmatrix}}=-27,5$};
				\node at (1.25,1.25) {\Huge $\cdot$};
				\end{tikzpicture}
			\end{center}
			Wenn das Skalarprodukt nicht Null ist, können wir trotzdem den Winkel bestimmen, es ist nur etwas komplizierter. Dafür rechnet man das Skalarprodukt durch das Produkt der beiden Vektorlängen und erhält damit den Kosinuswert.
			\begin{tcolorbox}[boxsep=0pt,top=1cm,left=1cm,right=1cm, bottom=.75cm,arc=0pt,auto outer arc,colback=white,colframe=black, enlarge top by=.25cm, enlarge bottom by=.25cm]
				\textbf{Winkelberechnung zwischen Vektoren}
				\begin{flalign*}
					\cos(\alpha)=\frac{\vec v\cdot \vec w}{\vert\vec v\vert\cdot\vert\vec w\vert}&&
				\end{flalign*}
			\end{tcolorbox}
			\noindent Nehmen wir dazu das Beispiel $\textcolor{teal}{\begin{pmatrix}-8\\3\end{pmatrix}}\cdot\textcolor{violet}{\begin{pmatrix}4\\1,5\end{pmatrix}}$.
			\begin{flalign*}
				\cos(\alpha)&=\frac{
					\begin{pmatrix}-8\\3\end{pmatrix}\cdot\begin{pmatrix}4\\1,5\end{pmatrix}}{\left\vert\begin{pmatrix}-8\\3\end{pmatrix}\right\vert\cdot\left\vert\begin{pmatrix}4\\1,5\end{pmatrix}\right\vert}\\
				&=\frac{-27,5}{\sqrt{73}\cdot\frac{\sqrt{73}}{2}}\\
				&=\dfrac{-27,5}{\frac{73}{2}}\\
				&=-\frac{55}{73}\\
				\Rightarrow\alpha&\approx 138,9^{\circ}&&
			\end{flalign*}
			Achtung: Achte darauf, dass du im Taschenrechner das Gradmaß und nicht das Bogenmaß eingestellt hast!\newline\newline
			Mit dem Skalarprodukt finden wir alternativ zum Satz des Pythagoras\index{Satz des Pythagoras} auch die Länge eines Vektors, denn es gilt $\vec v \cdot \vec v=\vert \vec v\vert^2$. Wenn wir also das Skalarprodukt eines Vektors mit sich selber bilden und anschließend die Wurzel vom Betrag des Skalarproduktes ziehen, erhalten wir die Länge des Vektors.\newline\newline
			Beispiel:
			\begin{flalign*}
				\left\vert\begin{pmatrix}4\\-1\end{pmatrix}\cdot\begin{pmatrix}4\\-1\end{pmatrix}\right\vert^2&=17\\
				\Rightarrow\sqrt{17}&\approx 4,123&&
			\end{flalign*}
			Hinweis: All die oben genannten Rechengesetze, wie z.~B. das Skalarprodukt lassen sich im dreidimensionalen Raum genauso anwenden wie im zweidimensionalen.
			\vspace{.5cm}\hrule\vspace{.7cm}
			\noindent Neben dem Skalarprodukt gibt es noch eine Möglichkeit das Produkt aus zwei Vektoren zu bilden, nämlich das sogenannte \textit{Kreuzprodukt}\index{Kreuzprodukt}. Mit dem Kreuzprodukt kriegen wir den Normalenvektor heraus. Das ist ein Vektor, der Orthogonal zu deinen beiden ursprünglichen Vektoren ist. Um Das Kreuzprodukt zu berechnen schreibt man nebeneinander die Werte der Vektoren jeweils zwei mal übereinander und multipliziert wie unten angeben über Kreuz: Erst die blauen Linien und dann Minus die roten.
			\begin{center}
				\begin{tikzpicture}
				\draw [blue] (1.15,-1) rectangle (2.35,1);
				\draw [red] (2.75,-1) rectangle (3.95,1);
				\node[align=left,violet] at (8,0) {$x_1$\\$y_1$\\$z_1$\\$x_1$\\$y_1$\\$z_1$};
				\node[align=left,orange] at (10,0) {$x_1$\\$y_1$\\$z_1$\\$x_1$\\$y_1$\\$z_1$};
				\node at (0,0) {$\textcolor{violet}{\vec{v_1}}\times\textcolor{orange}{\vec{v_2}}=\begin{pmatrix}\textcolor{violet}{x_1}\\\textcolor{violet}{y_1}\\\textcolor{violet}{z_1}\end{pmatrix}\times\begin{pmatrix}\textcolor{orange}{x_2}\\\textcolor{orange}{y_2}\\\textcolor{orange}{z_2}\end{pmatrix}=\begin{pmatrix}\textcolor{violet}{y_1}\textcolor{orange}{z_2}-\textcolor{violet}{z_1}\textcolor{orange}{y_2}\\\textcolor{violet}{z_1}\textcolor{orange}{x_2}-\textcolor{violet}{x_1}\textcolor{orange}{z_2}\\\textcolor{violet}{x_1}\textcolor{orange}{y_2}-\textcolor{violet}{y_1}\textcolor{orange}{x_2}\end{pmatrix}$};
				\draw[solid, red] (8.3,0.35) -- (9.6,0.75);
				\draw[solid, red] (8.3,-0.15) -- (9.6,0.25);
				\draw[solid, red] (8.3,-0.65) -- (9.6,-0.25);
				\draw[solid, blue] (8.3,0.75) -- (9.6,0.35);
				\draw[solid, blue] (8.3,0.25) -- (9.6,-0.15);
				\draw[solid, blue] (8.3,-0.25) -- (9.6,-0.65);
				\end{tikzpicture}
			\end{center}
			Anmerkung: Um den Normalenvektor zu ermitteln, kann man alternativ auch ein Gleichungssystem aufstellen, indem man sagt, dass das Skalarprodukt aus dem Normalenvektor und jeweils einem der beiden Richtungsvektoren Null ergibt. Man hätte dann zwei Gleichungen der Form $\vec{n}\cdot\vec{v}=0$.
			\subsection{Geraden}
				Um eine Gerade aufzustellen, braucht man zwei von einander verschiedene Punkte.
				\begin{center}
					\bgroup
					\def\arraystretch{0}
					\def\tabcolsep{0pt}
					\begin{tabularx}{\linewidth}{XX}
						\textbf{Koordinatenform}\newline\newline
						\index{Koordinatenform}Diese Notation ist wahrscheinlich am einfachsten zu verstehen, denn man hat zwei \textcolor{red}{Punkte}, zwischen denen man eine \textcolor{teal}{Gerade} zeichnen kann.
						\begin{tcolorbox}[boxsep=0pt,top=0cm,left=.5cm,right=.5cm, bottom=.5cm,arc=0pt,auto outer arc,colback=white,colframe=black,enlarge top by=0.5cm]
							\begin{flalign*}
							g:ax+by=c
							\end{flalign*}
						\end{tcolorbox}
						&
						\begin{flushright}
							\begin{tikzpicture}[x=0.5cm,y=0.5cm,z=0.3cm,>=stealth]
							\draw[->] (xyz cs:x=-7) -- (xyz cs:x=7) node[above] {$x$};
							\draw[->] (xyz cs:y=-7) -- (xyz cs:y=7) node[right] {$y$};
							\draw[->] (xyz cs:z=-7) -- (xyz cs:z=7) node[above] {$z$};
							
							\foreach \coo in {-7,-6,...,6}
							{
								\draw (\coo,-1.5pt) -- (\coo,1.5pt);
								\draw (-1.5pt,\coo) -- (1.5pt,\coo);
								\draw (xyz cs:y=-0.15pt,z=\coo) -- (xyz cs:y=0.15pt,z=\coo);
							}
							
							\draw[dashed] (xyz cs:x=-3,y=3,z=2) -- (xyz cs:x=-3,y=0,z=2);
							\draw[dashed] (xyz cs:x=-3,y=0,z=2) -- (xyz cs:x=-3,y=0,z=0);
							\draw[dashed] (xyz cs:x=4,y=-2,z=-3) -- (xyz cs:x=4,y=0,z=-3);
							\draw[dashed] (xyz cs:x=4,y=0,z=-3) -- (xyz cs:x=4,y=0,z=0);
							\draw[solid,teal] (xyz cs:x=-3,y=3,z=2) -- (xyz cs:x=4,y=-2,z=-3);
							\node[fill,circle,inner sep=1.5pt,label={left:$P(-3;3;2)$}, red] at (xyz cs:x=-3,y=3,z=2) {};
							\node[fill,circle,inner sep=1.5pt,label={right:$Q(4;-2;-3)$}, red] at (xyz cs:x=4,y=-2,z=-3) {};
							\end{tikzpicture}
						\end{flushright}
					\end{tabularx}
					\egroup
				\end{center}
				\hrule
				\begin{center}
					\bgroup
					\def\arraystretch{0}
					\def\tabcolsep{0pt}
					\begin{tabularx}{\linewidth}{XX}
						\textbf{Parameterform}\newline\newline
						\index{Parameterform}Bei dieser Notationsform wird an einen Ortsvektor – genannt \textcolor{blue}{Stützvektor}\index{Stützvektor} – ein \textcolor{red}{Richtungsvektoren}\index{Richtungsvektor} angelegt. Dieser \textcolor{red}{Vektor} liegt jetzt auf einer \textcolor{teal}{Geraden}. Da die Länge des \textcolor{red}{Richtungsvektors} egal ist, nimmt man noch den Skalar\index{Skalar} $r$ hinzu.
						\begin{tcolorbox}[boxsep=0pt,top=0cm,left=.5cm,right=.5cm, bottom=.5cm,arc=0pt,auto outer arc,colback=white,colframe=black, enlarge top by=0.5cm]
							\begin{flalign*}
							g:\vec{x}=\overrightarrow{OP}+r\cdot\overrightarrow{PQ}
							\end{flalign*}
						\end{tcolorbox}
						&
						\begin{flushright}
							\begin{tikzpicture}[x=0.5cm,y=0.5cm,z=0.3cm,>=stealth]
							\draw[->] (xyz cs:x=-7) -- (xyz cs:x=7) node[above] {$x$};
							\draw[->] (xyz cs:y=-7) -- (xyz cs:y=7) node[right] {$y$};
							\draw[->] (xyz cs:z=-7) -- (xyz cs:z=7) node[above] {$z$};
							
							\foreach \coo in {-7,-6,...,6}
							{
								\draw (\coo,-1.5pt) -- (\coo,1.5pt);
								\draw (-1.5pt,\coo) -- (1.5pt,\coo);
								\draw (xyz cs:y=-0.15pt,z=\coo) -- (xyz cs:y=0.15pt,z=\coo);
							}
							
							\draw[dashed] (xyz cs:x=4,y=0,z=1) -- (xyz cs:x=4,y=3,z=1);
							\draw[dashed] (xyz cs:x=4,y=0,z=1) -- (xyz cs:x=4,y=0,z=0);
							\draw[-stealth,blue] (xyz cs:x=0,y=0,z=0) -- (xyz cs:x=4,y=3,z=1);
							
							\draw[dashed] (xyz cs:x=-4,y=2,z=1) -- (xyz cs:x=-4,y=0,z=1);
							\draw[dashed] (xyz cs:x=-4,y=0,z=1) -- (xyz cs:x=-4,y=0,z=0);
							\draw[solid,teal] (xyz cs:x=6,y=3.25,z=1) -- (xyz cs:x=-6,y=1.75,z=1);
							\draw[-stealth,red] (xyz cs:x=4,y=3,z=1) -- (xyz cs:x=-4,y=2,z=1);
							
							\node [above right] at (xyz cs:x=4,y=3,z=1) {$P(4;3;1)$};
							\node [above] at (xyz cs:x=-4,y=2,z=1) {$Q(-4;2;1)$};
							\end{tikzpicture}
						\end{flushright}
					\end{tabularx}
					\egroup
				\end{center}
				\hrule
				\begin{center}
					\bgroup
					\def\arraystretch{0}
					\def\tabcolsep{0pt}
					\begin{tabularx}{\linewidth}{XX}
						\textbf{Normalenform}\newline\newline
						\index{Normalenform}Bei dieser Notationsform braucht einen \textcolor{blue}{Stützvektor} und einen \textcolor{red}{Normalenvektor}, der orthogonal\index{Orthogonal} zu einer \textcolor{teal}{Geraden} ist. Die Normalenform für Geraden existiert nur in $\mathbb{R}^2$, denn in drei Dimension gibt es keinen eindeutigen Normalenvektor, der orthogonal zum Stützvektor ist.
						\begin{tcolorbox}[boxsep=0pt,top=0cm,left=.5cm,right=.5cm, bottom=.5cm,arc=0pt,auto outer arc,colback=white,colframe=black,enlarge top by=0.5cm]
							\begin{flalign*}
							g:\vec{n}\cdot[\vec{x}-\overrightarrow{OP}]=0
							\end{flalign*}
						\end{tcolorbox}
						&
						\begin{flushright}
							\begin{tikzpicture}[x=0.5cm,y=0.5cm,z=0.3cm,>=stealth]
							\draw[->] (xyz cs:x=-7) -- (xyz cs:x=7) node[above] {$x$};
							\draw[->] (xyz cs:y=-7) -- (xyz cs:y=7) node[right] {$y$};
							
							\foreach \coo in {-7,-6,...,6}
							{
								\draw (\coo,-1.5pt) -- (\coo,1.5pt);
								\draw (-1.5pt,\coo) -- (1.5pt,\coo);
							}
							
							\node [above right] at (2,1) {$P(2;1)$};
							\draw[solid,teal] (2,6) -- (2,-6);
							\draw[-stealth,blue] (0,0) -- (2,1);
							\draw[-stealth,red] (2,1) -- (5,1);
							\end{tikzpicture}
						\end{flushright}
					\end{tabularx}
					\egroup
				\end{center}
			\subsection{Ebenen}
				Um eine Ebene aufzustellen, braucht man drei Punkte, die paarweise verschieden sind und nicht auf einer Geraden liegen.
				\begin{center}
					\bgroup
					\def\arraystretch{0}
					\def\tabcolsep{0pt}
					\begin{tabularx}{\linewidth}{XX}
						\textbf{Koordinatenform}\newline\newline
						\index{Koordinatenform}Diese Notation ist wahrscheinlich am einfachsten zu verstehen, denn man hat drei \textcolor{red}{Punkte}, die zusammen eine \textcolor{teal}{Ebene} aufspannen. Genauer gesagt, handelt es sich um die \textcolor{red}{Spurpunkte}\index{Spurpunkt}, d.~h. die Schnittpunkte mit den Achsen.
						\begin{tcolorbox}[boxsep=0pt,top=0cm,left=.5cm,right=.5cm, bottom=.5cm,arc=0pt,auto outer arc,colback=white,colframe=black,enlarge top by=0.5cm]
							\begin{flalign*}
							E:ax+by+cz=d
							\end{flalign*}
						\end{tcolorbox}
						&
						\begin{flushright}
							\begin{tikzpicture}[x=0.5cm,y=0.5cm,z=0.3cm,>=stealth]
							\draw[->] (xyz cs:x=-7) -- (xyz cs:x=7) node[above] {$x$};
							\draw[->] (xyz cs:y=-7) -- (xyz cs:y=7) node[right] {$y$};
							\draw[->] (xyz cs:z=-7) -- (xyz cs:z=7) node[above] {$z$};
							
							\foreach \coo in {-7,-6,...,6}
							{
								\draw (\coo,-1.5pt) -- (\coo,1.5pt);
								\draw (-1.5pt,\coo) -- (1.5pt,\coo);
								\draw (xyz cs:y=-0.15pt,z=\coo) -- (xyz cs:y=0.15pt,z=\coo);
							}
							\fill[<->,teal,opacity=.2] (xyz cs:x=0,y=3,z=0) -- (xyz cs:x=4,y=0,z=0) -- (xyz cs:x=0,y=0,z=-3);
							\node[fill,circle,inner sep=1.5pt,label={left:$P(0;3;0)$}, red] at (xyz cs:x=0,y=3,z=0) {};
							\node[fill,circle,inner sep=1.5pt,label={below:$Q(4;0;0)$}, red] at (xyz cs:x=4,y=0,z=0) {};
							\node[fill,circle,inner sep=1.5pt,label={left:$R(0;0;-3)$}, red] at (xyz cs:x=0,y=0,z=-3) {};
							\end{tikzpicture}
						\end{flushright}
					\end{tabularx}
					\egroup
				\end{center}
				\hrule
				\begin{center}
					\bgroup
					\def\arraystretch{0}
					\def\tabcolsep{0pt}
					\begin{tabularx}{\linewidth}{XX}
						\textbf{Parameterform}\newline\newline
						\index{Parameterform}Bei dieser Notationsform werden an einen Ortsvektor – genannt \textcolor{blue}{Stützvektor}\index{Stützvektor} – zwei \textcolor{red}{Richtungsvektoren}\index{Richtungsvektor} oder auch \textcolor{red}{Spannvektoren}\index{Spannvektor} angelegt. Diese beiden \textcolor{red}{Vektoren} liegen jetzt auf einer \textcolor{teal}{Ebene}. Da die Länge der \textcolor{red}{Spannvektoren} egal ist, nimmt man noch die Skalaren\index{Skalar} $r$ und $s$ hinzu.
						\begin{tcolorbox}[boxsep=0pt,top=0cm,left=.5cm,right=.5cm, bottom=.5cm,arc=0pt,auto outer arc,colback=white,colframe=black, enlarge top by=0.5cm]
							\begin{flalign*}
							E:\vec{x}=\overrightarrow{OP}+r\cdot\overrightarrow{PQ}+s\cdot\overrightarrow{PR}
							\end{flalign*}
						\end{tcolorbox}
						&
						\begin{flushright}
							\begin{tikzpicture}[x=0.5cm,y=0.5cm,z=0.3cm,>=stealth]
							\draw[->] (xyz cs:x=-7) -- (xyz cs:x=7) node[above] {$x$};
							\draw[->] (xyz cs:y=-7) -- (xyz cs:y=7) node[right] {$y$};
							\draw[->] (xyz cs:z=-7) -- (xyz cs:z=7) node[above] {$z$};
							
							\foreach \coo in {-7,-6,...,6}
							{
								\draw (\coo,-1.5pt) -- (\coo,1.5pt);
								\draw (-1.5pt,\coo) -- (1.5pt,\coo);
								\draw (xyz cs:y=-0.15pt,z=\coo) -- (xyz cs:y=0.15pt,z=\coo);
							}
							
							\fill[<->,teal,opacity=.2] (xyz cs:x=4,y=3,z=1) -- (xyz cs:x=3,y=-3,z=-2) -- (xyz cs:x=-4,y=2,z=1);
							
							\draw[dashed] (xyz cs:x=4,y=0,z=1) -- (xyz cs:x=4,y=3,z=1);
							\draw[dashed] (xyz cs:x=4,y=0,z=1) -- (xyz cs:x=4,y=0,z=0);
							\draw[-stealth,blue] (xyz cs:x=0,y=0,z=0) -- (xyz cs:x=4,y=3,z=1);
							
							\draw[dashed] (xyz cs:x=3,y=-3,z=-2) -- (xyz cs:x=3,y=0,z=-2);
							\draw[dashed] (xyz cs:x=3,y=0,z=-2) -- (xyz cs:x=3,y=0,z=0);
							\draw[-stealth,red] (xyz cs:x=4,y=3,z=1) -- (xyz cs:x=3,y=-3,z=-2);
							
							\draw[dashed] (xyz cs:x=-4,y=2,z=1) -- (xyz cs:x=-4,y=0,z=1);
							\draw[dashed] (xyz cs:x=-4,y=0,z=1) -- (xyz cs:x=-4,y=0,z=0);
							\draw[-stealth,red] (xyz cs:x=4,y=3,z=1) -- (xyz cs:x=-4,y=2,z=1);
							
							\node [above right] at (xyz cs:x=4,y=3,z=1) {$P(4;3;1)$};
							\node [below right] at (xyz cs:x=3,y=-3,z=-2) {$Q(3;-3;-2)$};
							\node [above] at (xyz cs:x=-4,y=2,z=1) {$R(-4;2;1)$};
							\end{tikzpicture}
						\end{flushright}
					\end{tabularx}
					\egroup
				\end{center}
				\hrule
				\begin{center}
					\bgroup
					\def\arraystretch{0}
					\def\tabcolsep{0pt}
					\begin{tabularx}{\linewidth}{XX}
						\textbf{Normalenform}\newline\newline
						\index{Normalenform}Bei dieser Notationsform braucht man nur zwei Vektoren: Einmal den \textcolor{blue}{Stützvektor} und einmal den \textcolor{red}{Normalenvektor}. Der \textcolor{red}{Normalenvektor}\index{Normalenvektor} ist ein Vektor, der orthogonal\index{Orthogonal} zur \textcolor{teal}{Ebene} liegt. Da durch ihn die Ebene eindeutig identifiziert werden kann, kann er die beiden Spannvektoren ersetzen.
						\begin{tcolorbox}[boxsep=0pt,top=0cm,left=.5cm,right=.5cm, bottom=.5cm,arc=0pt,auto outer arc,colback=white,colframe=black,enlarge top by=0.5cm]
							\begin{flalign*}
							E:\vec{n}\cdot[\vec{x}-\overrightarrow{OP}]=0
							\end{flalign*}
						\end{tcolorbox}
						&
						\begin{flushright}
							\begin{tikzpicture}[x=0.5cm,y=0.5cm,z=0.3cm,>=stealth]
							\draw[->] (xyz cs:x=-7) -- (xyz cs:x=7) node[above] {$x$};
							\draw[->] (xyz cs:y=-7) -- (xyz cs:y=7) node[right] {$y$};
							\draw[->] (xyz cs:z=-7) -- (xyz cs:z=7) node[above] {$z$};
							
							\foreach \coo in {-7,-6,...,6}
							{
								\draw (\coo,-1.5pt) -- (\coo,1.5pt);
								\draw (-1.5pt,\coo) -- (1.5pt,\coo);
								\draw (xyz cs:y=-0.15pt,z=\coo) -- (xyz cs:y=0.15pt,z=\coo);
							}
							\fill[<->,teal,opacity=.2] (xyz cs:x=6,y=3,z=3) -- (xyz cs:x=6,y=3,z=-1) -- (xyz cs:x=2,y=3,z=-1) -- (xyz cs:x=2,y=3,z=3);
							
							\draw[dashed] (xyz cs:x=4,y=0,z=1) -- (xyz cs:x=4,y=3,z=1);
							\draw[dashed] (xyz cs:x=4,y=0,z=1) -- (xyz cs:x=4,y=0,z=0);
							\draw[-stealth,blue] (xyz cs:x=0,y=0,z=0) -- (xyz cs:x=4,y=3,z=1);
							
							\draw[-stealth,red] (xyz cs:x=4,y=3,z=1) -- (xyz cs:x=4,y=5,z=1);
							
							\node [below right] at (xyz cs:x=4,y=3,z=1) {$P(4;3;1)$};
							\end{tikzpicture}
						\end{flushright}
					\end{tabularx}
					\egroup
				\end{center}
		\subsection{Umwandeln von Ebenengleichungen}
		\noindent Manchmal muss man zwischen den verschiedenen Formeln umwandeln. Dazu folgt hier eine Übersicht. Dazu sei gesagt, dass diese Tabelle für Ebenen- und Geradengleichung gleichermaßen gilt.
		\begin{center}
			\bgroup
			\def\arraystretch{1.75}
			\begin{tabularx}{\textwidth}{|l|l|X|}
				\hline
				\textbf{Von} & \textbf{Nach} & \textbf{Wie?} \\ \hline
				Parameterform & Normalenform & Normalenvektor aufstellen mithilfe der Richtungsvektoren: Entweder Kreuzprodukt bilden oder Gleichungssystem aufstellen (Kreuzprodukt geht nicht bei Geradengleichungen) \\ \hline
				Normalenform & Koordinatenform & Den kompletten Term ausmultiplizieren \\ \hline
				Koordinatenform & Parameterform & x durch r und y durch s ersetzen (soweit vorhanden) und dann nach z umstellen und Parameterform aufstellen \\ \hline
				Koordinatenform & Normalenform & Normalenvektor ablesen (Koeffizienten vor xyz) und Stützvektor berechnen, indem man die Gleichung der Koordinatenform löst \\
				\hline
			\end{tabularx}
			\egroup
		\end{center}
		Für die nicht aufgeführten Umwandlungen muss man erst in eine andere Form als Zwischenschritt umwandeln. Zum besseren Verständnis folgt für jede Umwandlung noch ein Beispiel.\newline\newline
		\textbf{Von Parameterform nach Normalenform}
		\begin{flalign*}
			&E:\begin{pmatrix}x\\y\\z\end{pmatrix}=\begin{pmatrix}3\\2\\1\end{pmatrix}+r\cdot\begin{pmatrix}0\\2\\1\end{pmatrix}+s\cdot\begin{pmatrix}-3\\-7\\0\end{pmatrix}\\
			\Rightarrow&E:\begin{pmatrix}0\\2\\1\end{pmatrix}\times\begin{pmatrix}-3\\-7\\0\end{pmatrix}\cdot\left[\begin{pmatrix}x\\y\\z\end{pmatrix}-\begin{pmatrix}2\\3\\1\end{pmatrix}\right]\\
			\Rightarrow&E:\begin{pmatrix}7\\-3\\6\end{pmatrix}\cdot\left[\begin{pmatrix}x\\y\\z\end{pmatrix}-\begin{pmatrix}2\\3\\1\end{pmatrix}\right]&&
		\end{flalign*}\newline
		\textbf{Von Normalenform nach Koordinatenform}
		\begin{flalign*}
		&E:\begin{pmatrix}7\\-3\\6\end{pmatrix}\cdot\left[\begin{pmatrix}x\\y\\z\end{pmatrix}-\begin{pmatrix}2\\3\\1\end{pmatrix}\right]\\
		\Rightarrow&E:\begin{pmatrix}7\\-3\\6\end{pmatrix}\cdot\begin{pmatrix}x\\y\\z\end{pmatrix}-\begin{pmatrix}7\\-3\\6\end{pmatrix}\cdot\begin{pmatrix}2\\3\\1\end{pmatrix}\\
		\Rightarrow&E:7x-3y+6z=11&&
		\end{flalign*}\newline
		\textbf{Von Koordinatenform nach Parameterform}
		\begin{flalign*}
		&E:7x-3y+6z=11\\
		\Rightarrow&E:7r-3s+6z=11\\
		\Rightarrow&E:6z=11-7r+3s\\
		\Rightarrow&E:z=\frac{11}{6}-\frac{7}{6}r+\frac{1}{2}s\\
		\Rightarrow&E:\begin{pmatrix}x\\y\\z\end{pmatrix}=\begin{pmatrix}0&r&0\\0&0&s\\\frac{11}{6}&-\frac{7}{6}r&\frac{1}{2}s\end{pmatrix}=\begin{pmatrix}0\\0\\\frac{11}{6}\end{pmatrix}+r\cdot\begin{pmatrix}1\\0\\-\frac{7}{6}\end{pmatrix}+r\cdot\begin{pmatrix}0\\1\\\frac{1}{2}\end{pmatrix}&&
		\end{flalign*}
		Hinweis: Sollten von $x,y$ und $z$ nicht alle gegeben sein, einfach an der entsprechenden Stelle $0$ einsetzen. Übrigens nicht wundern, dass nicht dasselbe wie bei den vorherigen Beispiel rauskommt. Es gibt schließlich unendliche viele Möglichkeiten eine Ebenengleichung\index{Ebenengleichung} aufzustellen.\newline\newline\newline
		\textbf{Von Koordinatenform nach Normalenform}
		\begin{flalign*}
		&E:7x-3y+6z=11\\
		\Rightarrow&E:\begin{pmatrix}7\\-3\\6\end{pmatrix}\cdot\left[\begin{pmatrix}x\\y\\z\end{pmatrix}-\begin{pmatrix}2\\3\\1\end{pmatrix}\right]&&
		\end{flalign*}
		Hinweis: Für den Stützvektor können auch komplett andere Werte benutzt werden, solange sie die Gleichung in Koordinatenform erfüllen.
	\subsection{Lagebeziehungen}
		In der analytischen Geometrie schaut man sich oft an, wie Punkte, Geraden und Ebenen zueinander stehen. Von Interesse ist u.~a., ob sich Geraden schneiden, ob Punkte auf einer Geraden liegen oder auch, ob Ebenen parallel sind. Für dieses Kapitel solltest du \highlight{sec:gleichungssysteme}{Gleichungssysteme} lösen können. Vorab noch eine kleine Begriffsklärung: \textit{Windschief}\index{Windschief} bedeutet, dass zwei Geraden sich weder schneiden, noch parallel sind.
		\subsubsection{Punkt – Punkt}
			Für die Beziehung zwischen zwei Punkten gibt es nur ein Szenario: Entweder sie sind gleich oder nicht und das ist wirklich einfach zu überprüfen. Haben zwei Punkte dieselben Koordinaten, so sind sie gleich.
		\subsubsection{Punkt – Gerade/Ebene}
			Ein Punkt kann auf einer Geraden bzw. einer Ebene liegen oder nicht. Das überprüft man, indem man den Punkt für $\vec{x}$ in die Parameterform einsetzt und daraus ein Gleichungssystem\index{Gleichungssystem} aufstellt. Ist dieses lösbar, so liegt der Punkt auf der Geraden oder auf der Ebene. Dazu ein Beispiel mit dem Punkt $P(-3;-2;5)$ und einer Geraden:
			\begin{tcolorbox}[boxsep=0pt,top=0cm,left=0cm,right=20cm, bottom=0cm,arc=0pt,auto outer arc,colback=white,colframe=white]
				\begin{flalign*}
					&&g:\vec{x}=&\begin{pmatrix}4\\1\\3\end{pmatrix}+s\cdot\begin{pmatrix}2\\1\\-2\end{pmatrix}\\
					\Rightarrow&&\begin{pmatrix}-3\\-2\\5\end{pmatrix}=&\begin{pmatrix}4\\1\\3\end{pmatrix}+s\cdot\begin{pmatrix}2\\1\\-2\end{pmatrix}&&
				\end{flalign*}
			\end{tcolorbox}
			\begin{tcolorbox}[boxsep=0pt,top=0cm,left=0cm,right=20cm, bottom=0cm,arc=0pt,auto outer arc,colback=white,colframe=white]
				\begin{flalign*}
				(I)&&-3=&4+2s\\
				(II)&&-2=&1+s&&\mid -1\\
				(III)&&5=&3-2s&&\mid (II)\text{ einsetzen}\\
				&&\Downarrow&&\\
				(III)&&5=&3-2(-2-1)\\
				&&5=&9&&
				\end{flalign*}
			\end{tcolorbox}
			\noindent Wir erhalten eine unwahre Aussage. Damit ist das Gleichungssystem nicht lösbar und der Punkte liegt nicht auf der Geraden.
		\subsubsection{Gerade – Gerade/Ebene}
			\textbf{Schnittpunkt}\newline
			Um zu überprüfen, ob sich zwei Geraden schneiden oder eine Gerade mit einer Ebene schneidet, so setzt man diese gleich und stellt daraus ein Gleichungssystem auf. Es gilt ganz einfach, dass wenn das Gleichungssystem lösbar ist, dann existiert auch Schnittpunkt. Vorsicht, man darf für den Skalar $s$ im Gleichungssystem nicht den gleichen Buchstaben verwenden, denn unsere Richtungsvektoren dürfen unterschiedlich skaliert werden. Wenn es kein Ergebnis gibt, dann heißt das für den Fall Gerade/Ebene, dass diese parallel zueinander liegen.
			\begin{tcolorbox}[boxsep=0pt,top=0cm,left=0cm,right=20cm, bottom=0cm,arc=0pt,auto outer arc,colback=white,colframe=white]
				\begin{flalign*}
				&&\begin{pmatrix}4\\1\\3\end{pmatrix}+s\cdot\begin{pmatrix}2\\1\\-2\end{pmatrix}=&\begin{pmatrix}1\\2\\4\end{pmatrix}+r\cdot\begin{pmatrix}-2\\-1\\3\end{pmatrix}\\
				&&\Downarrow&&\\
				(I)&&4+2s=&1-2r\\
				(II)&&1+s=&2-r&&\mid -1\\
				(III)&&3-2s=&4+3r&&\mid (II)\text{ einsetzen}\\
				&&\Downarrow&&\\
				(III)&&3-2(1-r)=&4+3r\\
				\Leftrightarrow&&1+2r=&4+3r\\
				\Leftrightarrow&&-3=&r\\
				&&\Downarrow&&\\
				(I)&&4+2s=&1-2r&&\mid (II)\text{ einsetzen}\\
				\Rightarrow&&4+2(1-r)=&1-2r&&\mid (III)\text{ einsetzen}\\
				\Rightarrow&&4+2(1+3)=&1+6\\
				\Leftrightarrow&&12=&7&&
				\end{flalign*}
			\end{tcolorbox}
			Das Gleichungssystem ist nicht lösbar. Damit schneiden sich die beiden Geraden nicht.\newline\newline
			\textbf{Parallelität (nur Geraden)}\newline
			Sind die Richtungsvektoren\index{Richtungsvektor} von zwei Geraden Vielfache voneinander, so verlaufen sie parallel. Das kann man wieder mit einem Gleichungssystem ganz einfach beweisen.
			\begin{tcolorbox}[boxsep=0pt,top=0cm,left=0cm,right=20cm, bottom=0cm,arc=0pt,auto outer arc,colback=white,colframe=white]
				\begin{flalign*}
				&&\begin{pmatrix}2\\1\\-2\end{pmatrix}=&a\cdot\begin{pmatrix}-2\\-1\\3\end{pmatrix}\\
				&&\Downarrow&&\\
				(I)&&2=&-2a\\
				(II)&&1=&-1a\\
				(III)&&-2=&3a&&
				\end{flalign*}
			\end{tcolorbox}
			\noindent Für dieses Beispiel sind die Geraden nicht parallel, denn für $(I)$ und $(II)$ erhält man die Lösung $1$, diese erfüllt jedoch nicht die dritte Gleichung. Folglich sind diese beiden Vektoren und damit auch mögliche in Verbindung stehende Gerade nicht parallel zueinander.\newline\newline
			Achtung: Windschief\index{Windschief} sind sie erst, wenn man auch bewiesen hat, dass sie keinen Schnittpunkt haben.
		\subsubsection{Ebene – Ebene}
			Der größte Unterschied bei der Betrachtung des Lageverhältnisses zwischen zwei Ebenen, besteht bei der Interpretation des Ergebnisses. Kriegt man am Ende ein unwahre Aussage, liegen die Ebenen logischerweise parallel zu einander aber nicht aufeinander. Bekommen wir eine wahre Aussage als Ergebnis, sind die Ebenen identisch. Sollten am Ende ein oder zwei Parameter übrig bleiben, existiert eine Schnittgerade\index{Schnittgerade} der beiden Ebenen. In diesem Fall rechnet es sich am einfachsten, wenn man eine Gleichung in der Koordinatenform\index{Koordinatenform} und die andere in der Parameterform hat. Dazu ein Beispiel:
			\begin{tcolorbox}[boxsep=0pt,top=0cm,left=0cm,right=20cm, bottom=0cm,arc=0pt,auto outer arc,colback=white,colframe=white]
				\begin{flalign*}
				&&E_1:\begin{pmatrix}x\\y\\z\end{pmatrix}=&\begin{pmatrix}3\\1\\-2\end{pmatrix}+r\cdot\begin{pmatrix}2\\2\\2\end{pmatrix}+s\cdot\begin{pmatrix}3\\0\\-0,5\end{pmatrix}\\
				&&E_2:3x+y+2z=&10\\
				&&\Downarrow&&\\
				&&x=&3+2r+3s\\
				&&y=&1+2r\\
				&&z=&-2+2r-0,5s\\
				\end{flalign*}
			\end{tcolorbox}
			\noindent Diese Werte aus $E_1$ setzen wir jetzt in $E_2$ ein.
			\begin{tcolorbox}[boxsep=0pt,top=0cm,left=0cm,right=20cm, bottom=0cm,arc=0pt,auto outer arc,colback=white,colframe=white]
				\begin{flalign*}
				&&3x+y+2z=&10\\
				\Rightarrow&&3(3+2r+3s)+1+2r+2(-2+2r-0,5s)=&10\\
				\Leftrightarrow&&9+6r+9s+1+2r-4+4r-s=&10\\
				\Leftrightarrow&&12r+8s=&4&&
				\end{flalign*}
			\end{tcolorbox}
			\noindent Mithilfe dieses Ergebnisses kann man jetzt eine Schnittgeradengleichung aufstellen. Dafür muss man zunächst nach $r$ oder nach $s$ umstellen.
			\begin{tcolorbox}[boxsep=0pt,top=0cm,left=0cm,right=20cm, bottom=0cm,arc=0pt,auto outer arc,colback=white,colframe=white]
				\begin{flalign*}
				&&12r+8s=&4\\
				\Leftrightarrow&&8s=4-12r\\
				\Leftrightarrow&&s=\frac{1}{2}-\frac{3}{2}r&&
				\end{flalign*}
			\end{tcolorbox}
			\noindent Dieses $s$ setzt man jetzt in die Ebenengleichung in Parameterform ein und erhält damit die Schnittgerade.
			\begin{tcolorbox}[boxsep=0pt,top=0cm,left=0cm,right=20cm, bottom=0cm,arc=0pt,auto outer arc,colback=white,colframe=white]
				\begin{flalign*}
				&&g_s:\begin{pmatrix}x\\y\\z\end{pmatrix}=&\begin{pmatrix}3\\1\\-2\end{pmatrix}+r\cdot\begin{pmatrix}2\\2\\2\end{pmatrix}+\left(\frac{1}{2}-\frac{3}{2}r\right)\cdot\begin{pmatrix}3\\0\\-0,5\end{pmatrix}\\
				\Leftrightarrow&&g_s:\begin{pmatrix}x\\y\\z\end{pmatrix}=&\begin{pmatrix}3\\1\\-2\end{pmatrix}+r\cdot\begin{pmatrix}2\\2\\2\end{pmatrix}+\begin{pmatrix}3\left(\frac{1}{2}-\frac{3}{2}r\right)\\0\\-0,5\left(\frac{1}{2}-\frac{3}{2}r\right)\end{pmatrix}\\
				\Leftrightarrow&&g_s:\begin{pmatrix}x\\y\\z\end{pmatrix}=&\begin{pmatrix}3\\1\\-2\end{pmatrix}+r\cdot\begin{pmatrix}2\\2\\2\end{pmatrix}+\begin{pmatrix}\frac{3}{2}-\frac{9}{2}r\\0\\-\frac{1}{4}+\frac{3}{4}r\end{pmatrix}\\
				\Leftrightarrow&&g_s:\begin{pmatrix}x\\y\\z\end{pmatrix}=&\begin{pmatrix}4,5\\1\\-2,25\end{pmatrix}+r\cdot\begin{pmatrix}-2,5\\2\\2,75\end{pmatrix}&&
				\end{flalign*}
			\end{tcolorbox}
	\pagebreak
	\section{2D-Koordinatensystem}
		\subsection{Allgemeines}
			\subsubsection{Monotonie}
			\label{subsubsec:monotonie}
				\begin{tcolorbox}[boxsep=0pt,top=1cm,left=1cm,right=1cm, bottom=.75cm,arc=0pt,auto outer arc,colback=white,colframe=black, enlarge top by=.25cm, enlarge bottom by=.25cm]
						Seien $x_1$ und $x_2$ zwei Argumente einer Funktion, so gelten folgende Definitionen:\newline\newline
						\textbf{Monoton wachsend}\index{Monoton}\index{Streng}\index{Wachsen}\index{Fallen}
						\begin{flalign*}
						wenn\;x_1\le x_2\; und\; f(x_1)\le f(x_2)&&
						\end{flalign*}
						\textbf{Streng monoton wachsend}
						\begin{flalign*}
						wenn\;x_1< x_2\; und\; f(x_1)< f(x_2)&&
						\end{flalign*}
						\textbf{Monoton fallend}
						\begin{flalign*}
						wenn\;x_1\le x_2\; und\; f(x_1)\ge f(x_2)&&
						\end{flalign*}
						\textbf{Streng monoton fallend}
						\begin{flalign*}
						wenn\;x_1< x_2\; und\; f(x_1)> f(x_2)&&
						\end{flalign*}
				\end{tcolorbox}
				\noindent Des Weiteren kann man, das Monotonieverhalten\index{Monotonieverhalten} einer Funktion mithilfe ihrer \highlight{subsec:ableitung}{Ableitung}\index{Ableitung} bestimmen. Ist die Ableitung $f^{\prime}$ einer Funktion\index{Funktion} größer oder gleich Null, so ist sie monoton wachsend. Ist sie größer als und ungleich Null, ist sie sogar streng monoton wachsend. Dasselbe gilt umgekehrt für monoton fallende Funktion, wenn ihre Ableitung an der untersuchten Stelle negativ ist.\newline\newline
				\textbf{Monoton wachsend}: $f^{\prime}(x)\ge 0$\newline\newline
				\textbf{Streng monoton wachsend}: $f^{\prime}(x)> 0$\newline\newline
				\textbf{Monoton fallend}: $f^{\prime}(x)\le 0$\newline\newline
				\textbf{Streng monoton fallend}: $f^{\prime}(x)< 0$\newline\newline
			\subsubsection{Umkehrbarkeit}
			\label{subsubsec:umkehrbarkeit}
				Bei der Umkehrfunktion $f^{-1}$ einer Funktion $f$ werden quasi Abszissen\index{Abszisse} und Ordinaten\index{Ordinate} vertauscht. Das heißt quasi, dass die Funktion von der $y$-Achse auf die $x$-Achse gelegt wird. Da es aber für jedes $x$ nur ein $y$ geben darf, gelten besondere Regeln für die Umkehrbarkeit von Funktionen: Eine Funktion ist in einem Intervall umkehrbar, wenn sie in diesem Intervall entweder nur streng monoton fallend oder nur streng monoton wachsend ist. In anderen Worten, darf es für jeden $y$-Wert maximal einen $x$-Wert geben, der auf ersteren verweist. Diese Eigenschaft nennt man auch \textit{Injektivität}\index{Injektivität}. Um das zu überprüfen kann man mithilfe der \highlight{subsec:ableitung}{Ableitung} die Extremstellen der Funktion untersuchen. Die Funktion sollte außerdem \textit{surjektiv}\index{Surjektiv} sein, d.~h. Jeder $y$-Wert hat mindestens einen $x$-Wert zu gewiesen. Wenn beides gegeben ist, braucht jeder $y$ in anderen Worten genau einen $x$-Wert. Das wird auch als \textit{bijektiv}\index{Bijektiv} bezeichnet.\newline
				Wenn man sich also sicher ist, dass die Funktion in jeweiligen Intervall bijektiv ist, darf man auch eine Umkehrfunktion bilden. Das ist je nach Funktionstyp gar nicht so schwer, denn man muss die Funktion nur nach $y$ umstellen und dann $x$ und $y$ vertauschen.
				Hier ein Beispiel für eine Funktion und ihre Umkehrfunktion:
				\makeplot{{e^x-1},{ln(x+1)}}{{$f(x)=e^x-1$},{$f^{-1}(x)=\ln(x+1)$}}{{3,2.5},{9,3}}{-10,16}{-3,10}{-25:30}{200}{17cm,7cm}{smooth}
			\subsubsection{Besondere Stellen}
				Für manche Stellen einer Funktion werden besondere Begriffe benutzt. Hinweis: $D_f$ ist der Definitionsbereich\index{Definitionsbereich} der Funktion\index{Funktion} und $I$ ein beliebig kleiner offener Intervall\index{Intervall}, der $x_{max}$ bzw. $x_{min}$ beinhaltet.
				\begin{tcolorbox}[boxsep=0pt,top=1cm,left=1cm,right=1cm, bottom=.75cm,arc=0pt,auto outer arc,colback=white,colframe=black, enlarge top by=.25cm, enlarge bottom by=.25cm]
					\textbf{Globale Maximalstelle}\index{Global}\index{Lokal}\index{Maximalstelle}\index{Minimalstelle}\index{Strickt}
					\begin{flalign*}
						wenn\;f(x_{max})\ge f(x)\; aller\;x\in D_f&&
					\end{flalign*}
					\textbf{Lokale Maximalstelle}
					\begin{flalign*}
						wenn\;f(x_{max})\ge f(x)\; aller\;x\in D_f\cap I&&
					\end{flalign*}
					\textbf{Globale Minimalstelle}
					\begin{flalign*}
						wenn\;f(x_{max})\le f(x)\; aller\;x\in D_f&&
					\end{flalign*}
					\textbf{Lokale Minimalstelle}
					\begin{flalign*}
						wenn\;f(x_{max})\le f(x)\; aller\;x\in D_f\cap I&&
					\end{flalign*}
					\textbf{Strikte Extrema}\newline\newline
					Ersetzt man bei den obigen Definitionen das $\ge$ bzw. $\le$ durch $>$ bzw. $<$, spricht man von einem strickten Maximum oder Minimum.
				\end{tcolorbox}
				\noindent Hinweis: Maximal- und Minimalstellen werden auch als Extremalstellen bezeichnet.
				\subsubsection{Symmetrie}
				\label{subsubsec:symmetrie}
				Wenn man Funktionen untersucht, schaut man sich auch oft an, wie deren Symmetrie\index{Symmetrie} ist. Ist eine Funktion achsensymmetrisch\index{Achsensymmetrisch} zur $y$-Achse, spricht man von \textit{gerade}\index{Gerade} und wenn sie punktsymmetrisch\index{Punktsymmetrisch} zum Nullpunkt\index{Nullpunkt} ist, von \textit{ungerade}\index{Ungerade}.
				\begin{tcolorbox}[boxsep=0pt,top=1cm,left=1cm,right=1cm, bottom=.75cm,arc=0pt,auto outer arc,colback=white,colframe=black, enlarge top by=.25cm, enlarge bottom by=.25cm]
					\textbf{Gerade}
					\begin{flalign*}
					wenn\;f(-x)=f(x)&&
					\end{flalign*}
					\textbf{Ungerade}
					\begin{flalign*}
					wenn\;f(-x)=-f(x)&&
					\end{flalign*}
				\end{tcolorbox}
				\noindent Achtung: Eine Funktion kann nur gerade oder ungerade sein, wenn ihr Definitionsbereich\index{Definitionsbereich} symmetrisch zur Nullpunkt auf der $x$-Achse ist.
			\subsubsection{Newtonverfahren}
			\label{subsubsec:newtonverfahren}
			Bei komplizierten Funktion kann es passieren, dass – obwohl welche existieren – man keine Nullstellen\index{Nullstelle} findet. Manchmal nicht einmal durch Raten. Eine ganzrationale Funktion zu vereinfachen durch Polynomdivision\index{Polynomdivision} ist z.~B. ebenfalls erst möglich, wenn man eine Nullstelle gefunden hat. Dennoch findet man online viele Rechner, die einem die Nullstellen geben, aber wie machen die das? In diesem Fall hilft das Newtonverfahren\index{Newtonverfahren} sich einem Wert zu nähern. Hier ist ein Beispiel für so einen Fall.
			\makeplot{{x^3+2*x^2-x-1}}{{$f(x)=x^{3}+2x^{2}-x-1$}}{{3.9,3.5}}{-6,6}{-10,10}{-20:20}{300}{17cm,7cm}{smooth}
			Als erstes legt man eine Wertetabelle\index{Wertetabelle} an, um die grobe Positionen der Nullstellen zu finden. Dabei will man wissen, zwischen welchen Stellen das Vorzeichen\index{Vorzeichen} wechselt.
			\begin{center}
				\bgroup
				\def\arraystretch{1.5}
				\begin{tabular}{ | l | c | c | c | c | c | c | c | }
					\hline
					\textbf{x} & -3 & -2 & -1 & 0 & 1 & 2 & 3 \\ \hline
					\textbf{f(x)} & -7 & 1 & 1 & -1 & 1 & 13 & 41 \\
					\hline
				\end{tabular}
				\egroup
			\end{center}
			Wir untersuchen jetzt einmal die Nullstelle zwischen $-1$ und $0$. Um uns der Nullstelle anzunähern\index{Annähern} teilen wir die Funktion\index{Funktion} durch ihre Ableitung\index{Ableitung} an einer der beiden Stellen, zwischen denen unsere gesuchte Nullstelle\index{Nullstelle} liegt. Es gilt folgende Formel:
			\begin{tcolorbox}[boxsep=0pt,top=.3cm,left=1cm,right=1cm, bottom=.75cm,arc=0pt,auto outer arc,colback=white,colframe=black, enlarge top by=.5cm, enlarge bottom by=.45cm]
				\begin{flalign*}
				x_{n+1}=x_{n}-\frac{f(x_{n})}{f^{\prime}(x_{n})}
				\end{flalign*}
			\end{tcolorbox}
			\noindent Um diese Formel\index{Formel} anzuwenden, brauchen wir als erstes die Ableitung und dann wiederholen wir diesen Prozess solange, bis wir genügend Nachkommastellen\index{Nachkommastelle} oder die tatsächliche Nullstelle gefunden haben.
			\begin{flalign*}
			f^{\prime}(x)=3x^2+4x-1&&
			\end{flalign*}
			\begin{flalign*}
			x_1&=(-1)-\frac{(-1)^{3}+2(-1)^{2}-(-1)-1}{3(-1)^2+4(-1)-1}=-\frac{1}{2}\\
			x_2&=(-\frac{1}{2})-\frac{(-\frac{1}{2})^{3}+2(-\frac{1}{2})^{2}-(-\frac{1}{2})-1}{3(-\frac{1}{2})^2+4(-\frac{1}{2})-1}=-\frac{5}{9}\\
			x_3&=(-\frac{5}{9})-\frac{(-\frac{5}{9})^{3}+2(-\frac{5}{9})^{2}-(-\frac{5}{9})-1}{3(-\frac{5}{9})^2+4(-\frac{5}{9})-1}=-\frac{929}{1674}\\
			x_4&=(-\frac{929}{1674})-\frac{(-\frac{929}{1674})^{3}+2(-\frac{929}{1674})^{2}-(-\frac{929}{1674})-1}{3(-\frac{929}{1674})^2+4(-\frac{929}{1674})-1}=-0,5549581321\\
			x_5&=(-0,5549581321)-\frac{(-0,5549581321)^{3}+2(-0,5549581321)^{2}-(-0,5549581321)-1}{3(-0,5549581321)^2+4(-0,5549581321)-1}\\
			&=-0,5549581321&&
			\end{flalign*}
			Wenn man zwei Mal den gleich Wert bekommt, weiß man, dass man den endgültigen Wert erreicht hat. Damit haben wir jetzt eine Nullstelle bestimmt, mit der wir z.B. die Polynomdivision\index{Polynomdivision} anwenden können.
		\subsection{Potenz- und Wurzelfunktionen}
			Potenzfunktionen\index{Potenzfunktion} in der Form $f(x)=x^m$ mit $m\in\mathbb{N}_0$ und $D_f=\mathbb{R}$ heißen \textbf{Monome} (im Gegensatz zu Polynomen). Potenzfunktionen mit der Form $x^{\frac{m}{n}}$ sind \textbf{Wurzelfunktionen}\index{Wurzelfunktion}, wenn $n\ge 2$ gilt und der Bruch\index{Bruch} keine ganze Zahl\index{Ganze Zahl} ist. An der Potenz\index{Potenz} kann man erkennen, ob eine Funktion gerade\index{Gerade} ($x^{2n}$) oder ungerade\index{Ungerade} ($x^{2n-1}$) ist. Hier sind einige Beispiele für Graphen\index{Graph} von Potenz- und Wurzelfunktionen:\newline
			\makeplot{{x^3},{x^2},{x^(3/2)},{x^(2/3)},{x^(-1/2)},{x^(-2)},{x^(-3)}}{{$f(x)=x^3$},{$f(x)=x^2$},{$f(x)=x^{\frac{3}{2}}$},{$f(x)=x^{\frac{2}{3}}$},{$f(x)=x^{-\frac{1}{2}}$},{$f(x)=x^{-2}$},{$f(x)=x^{-3}$}}{{1,-1},{1,-2},{4,-1},{4,-2},{7,-1},{7,-2},{10,-1}}{-6,6}{-6,6}{-10:10}{300}{17cm,7cm}{smooth}
			\subsubsection{Wurzelgleichungen}
			Bei Wurzelgleichungen\index{Wurzelgleichung} wird zuerst der Definitionsbereich\index{Definitionsbereich} bestimmt werden, also die Menge an reellen Zahlen\index{Reelle Zahl}, für die der Radikand\index{Radikand} positiv\index{Positiv} oder gleich Null ist. Zur Lösung von Wurzelgleichungen wird die Wurzel\index{Wurzel} auf einer Seite der Gleichung isoliert. Dann werden beide Seiten der Gleichung mit dem Wurzelexponenten\index{Wurzelexponent} (im Falle der Quadratwurzel also mit 2) so lange potenziert\index{Potenzieren}, bis alle Wurzeln eliminiert sind. Man bekommt also unter Umständen durch das Quadrieren (das Potenzieren mit einer geraden Zahl ist keine Äquivalenzumformung\index{Äquivalenzumformung}) neue Lösungen (Scheinlösungen\index{Scheinlösung}) hinzu, die die ursprüngliche Gleichung nicht hatte. Die Probe\index{Probe} ist folglich für Wurzelgleichungen unverzichtbar!\newline\newline
			Beispiel ($\sqrt{2x+1}=x-17$):
			\begin{flalign*}
			2x+1&\ge 0\\
			x&\ge -\frac{1}{2}&&
			\end{flalign*}
			Damit haben wir den Definitionsbereich\index{Definitionsbereich}. Jetzt kann man nach der Lösung suchen.
			\begin{flalign*}
			\sqrt{2x+1}&=x-17\\
			2x+1&=(x-17)^2\\
			2x+1&=x^2-34x+289\\
			x^2-36x+288&=0\\
			x_1&=12\\
			x_2&=24&&
			\end{flalign*}
			Jetzt MUSS man das Ergebnis\index{Ergebnis} noch überprüfen\index{Überprüfen}, indem man die Werte $x_1$ und $x_2$ in die ursprüngliche Gleichung einsetzt\index{Einsetzen}.
			\begin{flalign*}
			\sqrt{2x_1+1}&=x_1-17\\
			\sqrt{2\cdot 12+1}&=12-17\\
			\sqrt{25}&=-5\\
			5&=-5&&
			\end{flalign*}
			Das Einsetzen von $x_1$ liefert keine wahre Aussage\index{Aussage} und ist somit nicht Teil der Lösungsmenge\index{Lösungsmenge}.
			\begin{flalign*}
			\sqrt{2x_2+1}&=x_2-17\\
			\sqrt{2\cdot 24+1}&=24-17\\
			\sqrt{49}&=7\\
			7&=7&&
			\end{flalign*}
			Da $x_2$ im Definitionsbereich\index{Definitionsbereich} liegt und beim Einsetzen\index{Einsetzen} eine wahre Aussage\index{Aussage} ergibt, ist es in der Lösungsmenge enthalten.
			\begin{flalign*}
			\mathbb{L}=\{24\}&&
			\end{flalign*}
			Übrigens: Wenn man mehrere Wurzeln\index{Wurzel} in der Gleichung hat, muss man den Definitionsbereich\index{Definitionsbereich} für den Radikanden\index{Radikand} jeder Wurzel bestimmen.
			\makeplot{{sqrt(abs(2*x+1))},{x-17}}{{$f(x)=\sqrt{2x+1}$},{$f(x)=x-17$}}{{6.5,3.6},{8.3,1.4}}{0,30}{0,10}{-60:60}{600}{17cm,7cm}{smooth}
			Mithilfe dieser Graphen\index{Graph} kann man das Ergebnis wunderbar visualisieren, denn das Ergebnis\index{Ergebnis} ist der $x$-Wert des Schnittpunkts\index{Schnittpunkt} der beiden Funktionen, die man aus der linken und rechten Seite der Wurzelgleichung\index{Wurzelgleichung} entnehmen kann.
			\subsubsection{Wurzelgleichungen mit mehreren Wurzeln (Beispiel)}
			\begin{flalign*}
			\sqrt{8x-14}+\sqrt{5x-2}&=\sqrt{27x-36}\\
			(\sqrt{8x-14}+\sqrt{5x-2})^2&=27x-36\\
			8x-14+2\sqrt{(8x-14)(5x-2)}+5x-2&=27x-36\\
			2\sqrt{(8x-14)(5x-2)}&=14x-20\\
			\sqrt{(8x-14)(5x-2)}&=7x-10\\
			40x^2-86x+28&=(7x-10)^2\\
			40x^2-86x+28&=49x^2-140x+100\\
			0&=9x^2-54x+72\\
			0&=x^2-6x+8&&
			\end{flalign*}
			Jetzt kann man die \highlight{subsubsec:pqformel}{p-q-Formel}\index{p-q-Formel} anwenden und erhält die Lösungsmenge $\mathbb{L}=\{2;4\}$.
		\subsection{Betragsfunktionen}
			\label{subsec:betragsfunktionen}
			Um mit Betragsgleichungen\index{Betragsgleichung} oder auch Betragsfunktionen\index{Betragsfunktion} rechnen zu können muss man mehrere Fälle betrachten. Nämlich einmal den Fall\index{Fall}, dass im Betrag ein Wert größer oder gleich $0$ entsteht und einmal den Fall, dass das Ergebnis\index{Ergebnis} im Betrag\index{Betrag} kleiner als Null ist. Betrachten wir einmal ein Beispiel, wo man den Schnittpunkt\index{Schnittpunkt} zwischen $f(x)=\vert x+1\vert$ und $f(x)=x+2$ finden soll.\newline
			\makeplot{{x+2},{abs(x+1)}}{{$f(x)=x+2$},{$f(x)=\vert x+1\vert$}}{{5.5,4},{9.5,3.2}}{-5,5}{-4,4}{-10:10}{100}{17cm,7cm}{sharp plot}
			Zunächst setzen wir unsere Funktionen\index{Funktion} gleich\index{Gleichsetzen} und erhalten eine Betragsgleichung\index{Betragsgleichung}. Dann betrachten wir die verschiedenen Fälle für den Betrag.
			\begin{flalign*}
			\vert x+1\vert = \genfrac{\{}{.}{0pt}{}{x+1}{-(x+1)}\genfrac{}{}{0pt}{}{\text{ falls }x\ge -1}{\text{ falls }x<-1}&&
			\end{flalign*}
			Durch die Fallunterscheidung\index{Fallunterscheidung} kann man die Betragsstriche\index{Betragsstrich} weglassen, indem man jeden Fall einzeln betrachtet. Hinterher muss man aber noch überprüfen, ob das Ergebnis der Bedingung für $x$ in dem Fall entspricht.\newline\newline
			Fall $x\ge-1$ ($x+1$ ist positiv):
			\begin{flalign*}
			x+1&=x+2 &\mid-x &&&&&&&&&&\\
			1&=2&&
			\end{flalign*}
			Für den Fall $x\ge-1$ gibt es keine Lösung, also weiter zum nächsten Fall.\newline\newline
			Fall $x<-1$ ($x+1$ ist negativ):
			\begin{flalign*}
			-x-1&=x+2 &\mid+x-2 &&&&&&&&&&\\
			2x&=-3\\
			x&=-\frac{3}{2}&&
			\end{flalign*}
			Damit haben wir unsere Lösungsmenge, denn wir bekommen für den Fall $-(x<-1)$ ein Ergebnis, welches dem Kriterium $x<-1$ entspricht.
			\begin{flalign*}
			\mathbb{L}=\left\{-\frac{3}{2}\right\}&&
			\end{flalign*}
			Durch einsetzen dieser $x$-Koordinate, finden wir auch den dazugehörigen $y$-Wert: $P\left(-\frac{3}{2}\mid\frac{1}{2}\right)$:
			\subsubsection{Betragsgleichungen mit mehreren Beträgen}
			Haben wir mehrere Beträge in unserer Gleichung, haben wir auch mehrere Fälle zu betrachten. Schon wir uns das an einem Beispiel an, indem wir die Schnittpunkte\index{Schnittpunkt} von $f(x)=\vert x+1 \vert + 5$ und $f(x)=\vert 2x-4 \vert$ suchen.
			\makeplot{{abs(x+1)+5},{abs(2*x-4)}}{{$f(x)=\vert x+1 \vert + 5$},{$f(x)=\vert 2x-4 \vert$}}{{6.9,3.6},{10,1.7}}{-4,14}{0,20}{0:20}{300}{17cm,7cm}{sharp plot}
			Zunächst setzen wir die Funktionen wieder gleich.
			\begin{flalign*}
			\vert x+1 \vert + 5=\vert 2x-4 \vert&&
			\end{flalign*}
			Die Fälle\index{Fall} müssen wir alle einzeln betrachten. Das heißt, wir haben insgesamt 4 Fälle. Wir schauen uns zunächst die beiden Fälle eines Betrages an und dann innerhalb dieser Fälle betrachten wir die Fälle für den zweiten Betrag\index{Betrag}.\newline\newline
			1. Fall für $\vert x+1 \vert$: $x\ge-1$ ($x+1$ ist positiv)
			\begin{flalign*}
			x+1+5&=\vert 2x-4\vert\\
			x+6&=\vert 2x-4\vert&&
			\end{flalign*}
			Innerhalb dieses ersten Falles unterscheiden wir jetzt noch einmal für den übrigen Betrag.
			\begin{tcolorbox}[boxsep=0pt, left=2em, top=1em, bottom=1em,right=0cm,arc=0pt,auto outer arc,colback=white,colframe=white]
				1. Fall für $\vert2x-4\vert$: $x\ge2$ ($2x-4$ ist positiv)
				\begin{flalign*}
				x+6&=2x-4\\
				x+10&=2x\\
				10&=x&&
				\end{flalign*}
				Jetzt müssen wir überprüfen, ob $x\ge2$ und $x\ge-1$ für $x=10$ gelten. Das ist der Fall daher haben wir schon mal einen Teil unserer Lösungsmenge\index{Lösungsmenge}. Auf der Grafik kann man auch sehen, dass sich die beiden Graphen\index{Graph} dort schneiden.\newline\newline
				2. Fall für $\vert2x-4\vert$: $x<2$ ($2x-4$ ist negativ)
				\begin{flalign*}
				x+6&=-(2x-4)\\
				x+6&=-2x+4\\
				3x+6&=4\\
				3x&=-2\\
				x&=-\frac{2}{3}&&
				\end{flalign*}
				Wir überprüfen\index{Überprüfen} jetzt wieder, ob $x<2$ und $x\ge-1$ für $x=-\frac{2}{3}$ gelten. Da das der Fall ist, können wir auch dieses $x$ zu unserer Lösungsmenge\index{Lösungsmenge} hinzufügen.
			\end{tcolorbox}
			\noindent 2. Fall für $\vert x+1 \vert$: $x<-1$ ($x+1$ ist negativ\index{Negativ})
			\begin{flalign*}
			-(x+1)+5&=\vert 2x-4\vert\\
			-x+4&=\vert 2x-4\vert&&
			\end{flalign*}
			\begin{tcolorbox}[boxsep=0pt, left=2em, top=1em, bottom=1em,right=0cm,arc=0pt,auto outer arc,colback=white,colframe=white]
				1. Fall für $\vert2x-4\vert$: $x\ge2$ ($2x-4$ ist positiv)\newline\newline
				In diesem Fall müssen wir gar nicht erst versuchen $x$ auszurechnen, denn es gibt keine Zahl, die sowohl $x\ge2$, als auch $x<-1$ erfüllt.\newline\newline
				2. Fall für $\vert2x-4\vert$: $x<2$ ($2x-4$ ist negativ)
				\begin{flalign*}
				-x+4&=-(2x-4)\\
				-x+4&=-2x+4\\
				-x&=-2x\\
				x&=0&&
				\end{flalign*}
				Wir haben jetzt $x=0$ als Lösung, jedoch erfüllt dieses Ergebnis nicht die Bedingung $x<-1$ und ist daher auch nicht in der Lösungsmenge\index{Lösungsmenge} enthalten.
			\end{tcolorbox}
			\noindent Abschließend können wir feststellen, dass unsere Lösungsmenge $\mathbb{L}=\left\{10;-\frac{2}{3}\right\}$ ist. Durch Einsetzen\index{Einsetzen} in eine der beiden Funktionen erhalten wir dann unsere Schnittpunkte\index{Schnittpunkt} $P_1(10\mid 16)$ und $P_2\left(-\frac{2}{3}\mid\frac{16}{3}\right)$.
		\subsection{Ganzrationale Funktionen}
			Polynome\index{Polynom} sind die Summe aus den Vielfachen von Monomen\index{Monom}. Eine ganzrationale Funktion\index{Ganzrationale Funktion} oder auch Polynomfunktion\index{Polynomfunktion} genannt mit dem Koeffizienten\index{Koeffizient} $a_n$ hat folgende Form:
			\begin{flalign*}
			p(x)=a_nx^n+a_{n-1}x^{n-1}+...+a_1x^1+1_0&&
			\end{flalign*}
			Das Verhalten einer Polynomfunktion hängt für $x\to \infty$ vom Summanden\index{Summand} mit der höchsten Potenz\index{Potenz} und für $x\to 0$ vom Summanden mit der niedrigsten Potenz ab.
			\makeplot{{x^4-x^3-2*x^2},{x^4},{-2*x^2}}{{$f(x)=x^4-x^3-2x^2$},{$f(x)=x^4$},{$f(x)=-2x^2$}}{{6.5,3.2},{7.2,5},{7.4,1}}{-7,3}{-4,4}{-10:10}{400}{17cm,7cm}{smooth}
			\begin{tcolorbox}[boxsep=0pt,top=.85cm,left=1cm,right=1cm, bottom=.85cm,arc=0pt,auto outer arc,colback=white,colframe=black, enlarge top by=.25cm, enlarge bottom by=.25cm]
				\textbf{Nullstellen}\newline\newline
				Polynome\index{Polynom} n-ten Grades\index{n-ter Grad} haben maximal n Nullstellen\index{Nullstelle}.
				\begin{flalign*}
				p(x)=a_{2k-1}x^{2k-1}+...+a_1x+a_0,\;wenn\;a_{2k-1}\neq 0&&
				\end{flalign*}
				Polynome ungeraden Grades haben mindestens eine Nullstelle.
				\begin{flalign*}
				p(x)=a_{2k}x^{2k}+...+a_2x^2+a_0,\;wenn\;a_{2k}\ge 0\;und\;a_0>0&&
				\end{flalign*}
				Polynome geraden Grades besitzen keine Nullstellen.
			\end{tcolorbox}
			\begin{tcolorbox}[boxsep=0pt,top=.85cm,left=1cm,right=1cm, bottom=.85cm,arc=0pt,auto outer arc,colback=white,colframe=black, enlarge top by=.25cm, enlarge bottom by=.25cm]
				\textbf{Symmetrie}\newline\newline
				Für die Symmetrie\index{Symmetrie} der Funktion gilt wie bei Monomen\index{Monom} weiterhin, dass bei geraden Potenzen\index{Potenz} eine gerade\index{Gerade} Funktion vorliegt und bei ungeraden Potenzen eine ungerade\index{Ungerade} Funktion. Hat ein Polynom jedoch sowohl gerade, wie auch ungerade Exponenten\index{Exponent}, so kann man beides ausschließen\index{Ausschließen}.
			\end{tcolorbox}
		\subsubsection{Lösen durch Substitution}
		In diesem Beispiel werden die Nullstellen der Funktion mithilfe von \highlight{subsec:substitution}{Substitution}\index{Substitution} und anschließendem Anwenden der \highlight{subsubsec:pqformel}{p-q-Formel}\index{p-q-Formel} ermittelt.\newline
		\makeplot{{x^4-5*x^2+2}}{{$p(x)=x^4-5x^2+2$}}{{12.5,1.3}}{-6,6}{-6,6}{-30:60}{300}{17cm,7cm}{smooth}
		\begin{flalign*}
			p(x)&=x^4-5x^2+2\\
			0&=x^4-5x^2+2\\
			0&=u^2-5u+2\\
			u_{1,2}&=\frac{5}{2}\pm\sqrt{\left(-\frac{5}{2}\right)^2-2}\\
			u_1&=\frac{5+\sqrt{17}}{2}\\
			u_2&=\frac{5-\sqrt{17}}{2}\\\\
			x_{1,2}^2&=\frac{5+\sqrt{17}}{2}\\
			x_{1,2}^2&=\pm 2,135779205\\\\
			x_{3,4}^2&=\frac{5-\sqrt{17}}{2}\\
			x_{3,4}^2&=0,6621534469\\\\
			\mathbb{L}&=\{-2,135779205;-0,6621534469;0,6621534469;2,135779205\}&&
		\end{flalign*}
		\hrule
		\subsubsection{Lösen durch Faktorisierung}
		In diesem Beispiel werden die Nullstellen der Funktion\index{Funktion} mithilfe von \highlight{subsubsec:ausklammern}{Faktorisierung durch Ausklammern}\index{Faktorisieren}\index{Ausklammern} ermittelt.\newline
		\makeplot{{x^5-3*x^3}}{{$p(x)=x^5-3x^3$}}{{11.75,1.3}}{-6,6}{-6,6}{-30:60}{300}{17cm,7cm}{smooth}
		\begin{flalign*}
		p(x)&=x^5-3x^3\\
		0&=x^5-3x^3\\
		0&=x^2(x^2-3)\\
		0&=x^2(x^2-\sqrt{3})(x^2+\sqrt{3})\\
		\mathbb{L}&=\{-\sqrt{3};0;\sqrt{3}\}&&
		\end{flalign*}
		\hrule
		\subsubsection{Lösen mit binomischen Formeln}
		In diesem Beispiel wird Funktion mithilfe der \highlight{sec:binomischeformeln}{binomischen Formeln}\index{Binomische Formel} so vereinfacht, dass man die Nullstellen ganz einfach ablesen kann.\newline
		\makeplot{{x^4-2*x^2+1}}{{$p(x)=x^4-2x^2+1$}}{{11.6,3.2}}{-6,6}{-6,6}{-30:60}{300}{17cm,7cm}{smooth}
		\begin{flalign*}
		p(x)&=x^4-4x^2+1\\
		0&=x^4-4x^2+1\\
		0&=(x^2-1)^2\\
		x&=\pm 1\\
		\mathbb{L}&=\{-1;1\}&&
		\end{flalign*}
		\hrule
		\subsubsection{Lösen durch Polynomdivision}
			Wenn alle anderen Stränge reißen, ist man leider gezwungen die Polynomdivision\index{Polynomdivision} durchzuführen. Um damit beginnen zu können, braucht man aber mindestens eine Nullstelle\index{Nullstelle}, die man durch Raten findet. Für das Beispiel unten finden wir so heraus, dass eine Nullstelle $x_1=1$ ist. Jetzt stellen wir $x=1$ nach $0$ um und erhalten $0=x-1$. Anschließend teilen wir unser Polynom\index{Polynom} durch $x-1$.\newline
			\makeplot{{2*x^3-5*x^2-2*x+5}}{{$p(x)=2x^3-5x^2-2x+5$}}{{3.9,3.2}}{-6,6}{-6,6}{-30:60}{300}{17cm,7cm}{smooth}
			\begin{flalign*}
			(2x^3-5x^2-2x+5):(x-1)&&
			\end{flalign*}
			Zunächst teilt man den Term mit der höchsten Potenz\index{Potenz} $2x^3$ durch $x$ und erhält $2x^2$. Das ist der erste Teil unseres Ergebnisses\index{Ergebnis}.
			\begin{flalign*}
			(2x^3-5x^2-2x+5):(x-1)=2x^2...&&
			\end{flalign*}
			Jetzt muss man zurück multiplizieren, indem man den Term $2x^2$, den wir gerade bekommen haben, mit unserem ursprünglichen Divisor\index{Divisor} $x-1$ multiplizieren. Das Ergebnis ziehen wir von unserem Polynom\index{Polynom} ab und holen anschließend den nächsten Ausdruck runter. Diesen Prozess wiederholen wir jetzt so oft, wie möglich.\newline\newline
			\polylongdiv[style=C,div=:]{2x^3-5x^2-2x+5}{x-1}\newline\newline
			Mit der Funktion, die wir jetzt haben, können wir ganz einfach die restlichen Nullstellen\index{Nullstelle} errechnen.\newline
			\makeplot{{2*x^2-3*x-5}}{{$f(x)=2x^2-3x-5$}}{{3.9,3.2}}{-6,6}{-10,10}{-30:60}{300}{17cm,7cm}{smooth}
			\begin{flalign*}
				f(x)&=2x^2-3x-5\\
				0&=2x^2-3x-5\\
				0&=x^2-1,5x-2,5\\
				x_{1,2}&=\frac{1,5}{2}\pm\sqrt{\left(-\frac{1,5}{2}\right)^2+2,5}\\
				x_1&=2,5\\
				x_2&=-1\\
				\mathbb{L}&=\{-1;1;2,5\}&&
			\end{flalign*}
			\subsubsection{Grenzverhalten von ganzrationalen Funktionen}
				Hat man eine Funktion wie z.~B. $f(x)=-x^5+2x^3+3x^2+x+2$ und untersucht, wie sie sich gegen (minus) Unendlich\index{Unendlich} verhält, würde man intuitiv sagen, sie nähert sich (minus) Unendlich an. Hier soll es darum gehen, wie man das auch rechnerisch herausfinden kann und sicher unterscheidet, ob nun plus oder minus Unendlich richtig ist. Der Trick bei Funktionen dieser Form ist es, das $x$ mit dem höchsten Exponenten\index{Exponent} auszuklammern\index{Ausklammern}.
				\makeplot{{-x^5+2*x^3+3*x^2+x+2}}{{$f(x)=-x^5+2x^3+3x^2+x+2$}}{{12.2,4.6}}{-10,10}{-5,15}{-10:20}{200}{17cm,7cm}{smooth}
				\begin{flalign*}
				&\lim_{x\to\infty}-x^5+2x^3+3x^2+x+2\\
				&\lim_{x\to\infty}x^5(\frac{-x^5}{x^5}+\frac{2x^3}{x^5}+\frac{3x^2}{x^5}+\frac{x}{x^5}+\frac{2}{x^5})\\
				&\lim_{x\to\infty}x^5(-1+\frac{2}{x^2}+\frac{3}{x^3}+\frac{1}{x^4}+\frac{2}{x^5})&&
				\end{flalign*}
				Wir sehen, dass sich die Brüche in der Klammer alle Null annähern\index{Annähern}, somit bleibt dort nur noch $-1$. Währenddessen nähert sich $x^5$ Unendlich\index{Unendlich} an. Multipliziert mit $-1$ ergibt das dann minus Unendlich.
				\begin{flalign*}
				&\infty^5(-1+\frac{2}{\infty^2}+\frac{3}{\infty^3}+\frac{1}{\infty^4}+\frac{2}{\infty^5})\\
				=&\infty^5(-1+0+0+0+0)\\
				=&-\infty&&
				\end{flalign*}
				Dasselbe kann man jetzt natürlich auch für $x\to -\infty$ testen.
				\begin{flalign*}
				&-\infty^5(-1-\frac{2}{\infty^2}-\frac{3}{\infty^3}-\frac{1}{\infty^4}-\frac{2}{\infty^5})\\
				=&-\infty^5(-1-0-0-0-0)\\
				=&\infty&&
				\end{flalign*}
				Hinweis: Wenn man mit $x\to(-)\infty$ arbeitet, setzt man $\infty$ normalerweise nicht in die Funktion ein. Hier habe ich es einmal gemacht, damit man das Ergebnis besser nachvollziehen kann. Wenn man ausführlicher zu arbeiten will/muss, kann man den Limes\index{Limes} von jedem Term einzeln aufstellen, um das Endergebnis\index{Ergebnis} zu begründen.
		\subsection{(Gebrochen)rationale Funktionen}
			Wenn wir von (gebrochen)rationalen Funktionen\index{Rationale Funktion}\index{Gebrochenrationale Funktion} reden, meinem wir eine Funktion mit einem Polynom\index{Polynom} im Nenner\index{Nenner} eines Bruches\index{Bruch}. $f(x)=\frac{2}{x^3}$ ist z.~B. eine rationale Funktion, $f(x)=\frac{x^3}{2}$ jedoch nicht.
			\makeplot{{((2*x^4-10)/(x^3-3))}}{{$f(x)=\dfrac{2x^4-10}{x^3-3}$}}{{4.5,2.8}}{-6,6}{-5,10}{-22:20}{600}{17cm,7cm}{smooth}
			Das Besondere an rationalen Funktionen der Form $f(x)=\frac{g(x)}{h(x)}$ ist, dass wir zum Bestimmen von Nullstellen\index{Nullstelle} und Definitionslücken\index{Definitionslücke} den Zähler\index{Zähler} und Nenner\index{Nenner} einzeln betrachten können. Mithilfe des Zählers bestimmen wir ganz einfach Nullstellen der Funktion\index{Funktion}.
			\begin{flalign*}
				g(x)&=0\\
				0&=2x^4-10\\
				10&=2x^4\\
				5&=x^4\\
				x&=\pm\sqrt[4]{5}\\
				\mathbb{L}&=\{-\sqrt[4]{5};\sqrt[4]{5}\}&&
			\end{flalign*}
			Mithilfe des Nenners bestimmen wir Definitionslücken.
			\begin{flalign*}
				h(x)&=0\\
				0&=x^3-3\\
				3&=x^3\\
				x&=\sqrt[3]{3}\\
				\mathbb{L}&=\{\sqrt[3]{3}\}&&
			\end{flalign*}
		\subsection{Exponentialfunktionen}
			Eine Funktion der Form $f(x)=a^x$ wird als Exponentialfunktion bezeichnet, denn die Variable $x$ steht im Exponenten. Speziell wird die Funktion $f(x)=e^x$ als \textit{natürliche Exponentialfunktion} bezeichnet.
			\makeplot{{e^x},{e^(-x)}}{{$f(x)=e^x$},{$f(x)=e^{-x}$}}{{9.5,3.3},{5.6,3.3}}{-6,6}{-6,6}{-10:10}{300}{17cm,7cm}{smooth}
			\subsubsection{Lösen von Exponentialgleichungen}
				Zum Lösen von Exponentialgleichungen\index{Exponentialgleichung} brauchen wir in der Regel den \highlight{subsec:logarithmusgesetze}{Logarithmus}\index{Logarithmus}. Wie das funktioniert, sehen wir an dem Beispiel hier drunter. Dabei ist die Nullstelle\index{Nullstelle} der Funktion zu bestimmen. In diesem Beispiel sollte man sich außerdem nochmal daran erinnern, dass $\sqrt[n]{x}$ dasselbe ist, wie $x^{\frac{1}{n}}$.
				\makeplot{{(7^x)^(1/4)-4}}{{$f(x)=\sqrt[4]{7^x}-4$}}{{10,3.2}}{-6,6}{-10,10}{-30:30}{300}{17cm,7cm}{smooth}
				\begin{flalign*}
					f(x)&=\sqrt[4]{7^x}-4\\
					0&=\sqrt[4]{7^x}-4 && \mid +4 &&&&&&&&&&&&& \\
					4&=7^{\frac{x}{4}} && \mid log_7() &&&&&&&&&&&&& \\
					0,7124143742&=\frac{x}{4} && \mid \cdot 4 &&&&&&&&&&&&& \\
					x&=2,849657497\\
					\mathbb{L}=\{2,849657497\}&&
				\end{flalign*}
		\subsection{Logarithmusfunktionen}
			Funktionen wie $f(x)=log_3(x^2)$ werden als Logarithmusfunktionen bezeichnet, da sie einen oder mehrere Logarithmen beinhalten. Speziell bezeichnet man $f(x)=ln(x)$ als \textit{natürliche Logarithmusfunktion} und $f(x)=lg(x)$ als \textit{dekadische Logarithmusfunktion}.
			\makeplot{{ln(x)},{log10(x)},{-ln(abs(x))}, {log2(x)}}{{$f(x)=ln(x)$},{$f(x)=lg(x)$},{$f(x)=-ln(\vert x\vert)$},{$f(x)=log_2(x)$}}{{9.5,4.3},{12.5,4.8},{3.5,1},{9.5,1}}{-6,6}{-4,4}{-200:200}{500}{17cm,7cm}{smooth}
			\subsubsection{Lösen von Logarithmusgleichungen}
				Gesucht wird hier die Nullstelle einer Logarithmusfunktion\index{Logarithmusfunktion} gesucht. Hinweis: Um einen Logarithmus\index{Logarithmus} aufzulösen musst du beide Seiten der Gleichung als Exponent\index{Exponent} zur Basis\index{Basis} des jeweiligen Logarithmus setzen. Um dort hinzu kommen hilft es enorm, die Gleichung zunächst umzuformen\index{Umformen}. Wenn du Schwierigkeiten mit den Umformungen in diesem Beispiel hast, schaue dir noch einmal die \highlight{subsec:potenzgesetze}{Potenzgesetze}\index{Potenzgesetz} und \highlight{subsec:logarithmusgesetze}{Logarithmusgesetze}\index{Logarithmusgesetz} an.
				\makeplot{{3*log10(x^3)-2*log10(x^2)-4}}{{$f(x)=3\cdot lg(x^3)-2\cdot lg(x^2)-4$}}{{11,3.3}}{-10,10}{-10,10}{-20:20}{300}{17cm,7cm}{smooth}
				\begin{flalign*}
					f(x)&=3\cdot lg(x^3)-2\cdot lg(x^2)-4\\
					0&=3\cdot lg(x^3)-2\cdot lg(x^2)-4\\
					0&=lg((x^3)^3)-lg((x^2)^2)-4&&\mid +4&&&&&&&&&&&\\
					lg\left(\frac{x^9}{x^4} \right)&=4\\
					lg(x^5)&=4&&\mid 10^{()}\\
					10^{lg(x^5)}&=10^4\\
					x^5&=10^4&&\mid \sqrt[5]{\;}\\
					x&=6,309573445\\
					\mathbb{L}&=\{6,309573445\}&&
				\end{flalign*}
		\subsection{Trigonometrische Funktionen}
		\label{subsec:trigonometrisch}
			Trigonometrische Funktionen\index{Trigonometrische Funktion} oder auch Winkelfunktionen\index{Winkelfunktion} genannt, beinhalten die aus der \highlight{sec:geometrie}{Geometrie}\index{Geometrie} bekannten winkelabhängigen Funktionen, wie Sinus\index{Sinus}, Kosinus\index{Kosinus} und Tangens\index{Tangens}. Dabei sind diese Funktionen hier allerdings abhängig von der Variable\index{Variable} $x$ und damit im Bogenmaß\index{Bogenmaß}, nicht im Gradmaß\index{Gradmaß}. Beim Taschenrechner\index{Taschenrechner} muss man darauf achten, dass der richtige Modus eingestellt ist, ansonsten kann es sein, dass man versehentlich im falschen Maß rechnet. Auf dem CASIO fx-86DE PLUS, drückt man Shift, dann Setup und wählt dort die 3:Deg (engl. degree) für Gradmaß oder 4:Rad (engl. radian) fürs Bogenmaß.
			\makeplot{{sin(deg(x))},{cos(deg(x))},{tan(deg(x))}}{{$f(x)=\sin(x)$},{$f(x)=\cos(x)$},{$f(x)=\tan(x)$}}{{1.5,-1},{4.5,-1},{7.5,-1}}{-10,10}{-2,2}{-10:10}{200}{17cm,7cm}{smooth}
			Anmerkung: Die Nullstellen des Sinus sind die Extremstellen\index{Extremstelle} des Kosinus und umgekehrt. Ebenso haben Sinus und Tanges dieselben Nullstellen\index{Nullstelle}.
		\subsection{Verkettete Funktionen}
			Verkettungen\index{Verkettung} sind eigentliche keine eigene Funktionsart, sondern eine Möglichkeit Funktionen durch Zusammensetzung zu transformieren\index{Transformieren}. Man schreibt das als $f\circ g$ ("f nach g"). Man spricht hier bei $g$ auch von der \textit{inneren Funktion}\index{Innere Funktion}, da sie als Argument in die \textit{äußere Funktion}\index{Äußere Funktion} $f$ eingesetzt wird.
			\makeplot{{sin(deg(x))},{(x^2)/2},{sin(deg((x^2)/2))}}{{$f(x)=\sin(x)$},{$g(x)=\frac{x^2}{3}$},{$f\circ g(x)=\sin(\frac{x^2}{3})$}}{{1.5,-1},{4.5,-1},{8,-1}}{-10,10}{-2,2}{-10:10}{200}{17cm,7cm}{smooth}
		\subsection{Kreisgleichungen}
			Bei Funktionen ist es so, dass jeder Abszisse\index{Abszisse} ($x$-Achsenwert) genau eine Ordinate\index{Ordinate} ($y$-Achsenwert) zugeordnet werden kann. Das ist eine Besonderheit von Funktionen und nicht des Koordinatensystems an sich. Im Koordinatensystem\index{Koordinatensystem} können wir noch viele andere Gebilde darstellen, so z.~B. auch einen Kreis\index{Kreis}.
			\begin{tcolorbox}[boxsep=0pt,top=.75cm,left=1cm,right=1cm, bottom=.75cm,arc=0pt,auto outer arc,colback=white,colframe=black, enlarge top by=.5cm, enlarge bottom by=.45cm]
				Für die Koordinatenform\index{Koordinatenform} eines Kreises mit dem Radius $r$ und dem Mittelpunkt $M(x_m\mid y_m)$ gilt folgende Notation:
				\begin{flalign*}
				(x-x_m)^2+(y-y_m)^2=r^2&&
				\end{flalign*}
			\end{tcolorbox}
			\begin{center}
				\begin{tikzpicture}
				\begin{axis}[
				domain=0:10,
				width=17cm,
				height=7cm,
				restrict y to domain=0:10,
				xmin=0, xmax=28,
				ymin=0, ymax=10,
				axis equal image,
				samples=100,
				axis y line=center,
				axis x line=middle,
				ticklabel style={fill=white},
				minor tick num=2,
				grid=both,
				grid style={line width=.1pt, draw=gridgray!10},
				major grid style={line width=.2pt,draw=gridgray!50}
				]
				\draw[color=blue] (axis cs:14,5) circle [radius=3];
				\end{axis}
				\node [color=blue] at (3,1.6) {$(x-14)^2+(y-5)^2=3$};
				\end{tikzpicture}
			\end{center}
	\pagebreak
	\section{Differenzialrechnung}
		\subsection{Die Ableitung}
		\label{subsec:ableitung}
				Die Ableitung\index{Ableitung} einer Funktion\index{Funktion} gibt Aufschluss über ihr \highlight{subsubsec:monotonie}{Monotonieverhalten}\index{Monotonieverhalten} und die Veränderung ihres Anstiegs\index{Anstieg}, denn die Werte der Ableitung $f^{\prime} (x)$ einer Funktion $f(x)$ entsprechen dem Anstieg einer Tangente\index{Tangente} an derselben Stelle von $f(x)$. Das kann man auch an dem unten stehenden Beispiel erkennen. Um eine Potenzfunktion\index{Potenzfunktion} abzuleiten, nehmen wir den Exponenten\index{Exponent} von jedem $x$ und holen ihn hinunter, um ihn vor das jeweilige $x$ zu schreiben. Anschließend reduzieren wir den Exponenten um $1$. Dabei ist zu beachten, dass die Ableitung einer reinen Zahl ohne $x$ immer $0$ ist und, dass die Ableitung von $x=1$ ist, denn $x^0$ entspricht $1$.
			\makeplot{{x^2},{2*x}}{{$f(x)=x^2$},{$f^{\prime}(x)=2x$}}{{4,3.5},{3,1}}{-10,10}{-10,10}{-20:20}{200}{17cm,7cm}{smooth}
			Hinweis: Um Fehler zu vermeiden, sollte man zunächst alle Terme so umformen\index{Umformen}, dass man einfach ableiten kann. Dafür solltest du die wichtigsten \highlight{sec:rechengesetze}{Rechengesetze}\index{Rechengesetz} beherrschen.
			\begin{tcolorbox}[boxsep=0pt,top=.75cm,left=1cm,right=1cm, bottom=.65cm,arc=0pt,auto outer arc,colback=white,colframe=black, enlarge top by=.45cm, enlarge bottom by=.25cm]
				\textbf{Ableitungsregeln}\newline\newline
				Neben der oben genannten Ableitungsregel\index{Ableitungsregel} von Funktionen, gibt es noch einige andere, die einem das Leben erleichtern:
				\begin{multicols}{2}
					\noindent\begin{flalign*}
					&c&\rightarrow&\;\;0\\\\
					&x^n&\rightarrow&\;\;nx^{n-1}\\\\
					&\sqrt{x}&\rightarrow&\;\;\frac{1}{2\sqrt{x}}\\\\
					&e^x&\rightarrow&\;\;e^x\\\\
					&ln(x)&\rightarrow&\;\;\frac{1}{x}\\\\
					&a^x(a>0)&\rightarrow&\;\;ln(a\cdot a^x)\\\\
					&\frac{1}{x^n}&\rightarrow&\;\;-\frac{n}{x^{n+1}}&&&&&&&&&&&
					\end{flalign*}
					\begin{flalign*}
					&\sin(x)&\rightarrow&\;\; \cos(x)\\\\
					&\cos(x)&\rightarrow&\;\;-\sin(x)\\\\
					&\tan(x)&\rightarrow&\;\;\frac{1}{\cos^2x}\\\\
					&u(x)\cdot v(x)&\rightarrow&\;\;u^{\prime}(x)\cdot v(x)+u(x)\cdot v^{\prime}(x)\\\\
					&\frac{u(x)}{v(x)}&\rightarrow&\;\;\frac{u^{\prime}(x)\cdot v(x)-u(x)\cdot v^{\prime}(x)}{(v(x))^2}\\\\
					&(u\circ v)(x)&\rightarrow&\;\;u^{\prime}(v(x))\cdot v^{\prime}(x)&&&&&&&&&&&
					\end{flalign*}
				\end{multicols}
			\end{tcolorbox}
			Tipp: Solltest du es einmal nicht schaffen, eine Funktion mit den Ableitungsregeln abzuleiten oder diese vergessen haben, kannst du die Ableitung immer noch mithilfe des \highlight{subsubsec:differentialquotient}{Differentialquotienten!}\index{Differentialquotient} bestimmen.
			\subsubsection{Differenzenquotient}
			\label{subsubsec:differenzenquotient}
				Um die Steigung\index{Steigung} einer Sekante\index{Sekante} zwischen zwei Punkten\index{Punkt} zu berechnen, benutzen wir den Differenzenquotient\index{Differenzenquotient}.
				Dieser Quotient\index{Quotient} berechnet sich indem man $x$ und $y$ der beiden Punkte jeweils voneinander abzieht und dann $y$ durch $x$ teilt.
				\begin{flalign*}
				m=\frac{y_2-y_1}{x_2-x_1}&&
				\end{flalign*}
				\begin{center}
					\begin{tikzpicture}
					\begin{axis}[
					domain=-10:10,
					width=17cm,
					height=7cm,
					restrict y to domain=-10:20,
					xmin=-10, xmax=10,
					ymin=-5, ymax=10,
					samples=100,
					axis y line=center,
					axis x line=middle,
					ticklabel style={fill=white},
					minor tick num=2,
					grid=both,
					grid style={line width=.1pt, draw=gridgray!10},
					major grid style={line width=.2pt,draw=gridgray!50}
					]
					\addplot+[mark=none, color=blue, solid, name path=A, smooth] {x^3+2};
					\draw [color=red, name path=B] (axis cs: .2,0.75) -- (axis cs: 2.2,9);
					\fill [name intersections={of=A and B, by={a,b}}]
					(a) circle(2.5pt) node[below right] {$A(x_1\mid y_1)$}
					(b) circle (2.5pt) node[right] {$B(x_2\mid y_2)$};
					\end{axis}
					\node [color=blue] at (4.75,3) {$f(x)=x^3+2$};
					\end{tikzpicture}
				\end{center}
				Allgemeiner ausgedrückt, gilt die Formel:
				\begin{flalign*}
				m=\frac{f(x+h)-f(x)}{x+h-x}=\frac{f(x+h)-f(x)}{h}&&
				\end{flalign*}
				Dabei ist $h$ eine beliebige Zahl, mit deren Hilfe wir uns jetzt an die genaue Steigung\index{Steigung} in einem Punkt annähern können. Je kleiner wir den Abstand $h$ wählen, desto genauer kommen wir an die Tangente\index{Tangente} oder auch Steigung der Stelle $x$.
			\subsubsection{Differentialquotient}
			\label{subsubsec:differentialquotient}
				Der Differenzenquotient\index{Differenzenquotient} erlaubt es uns die Steigung einer Funktion an einer bestimmten Stelle zu bestimmen. Im Abschnitt über den \highlight{subsubsec:differenzenquotient}{Differenzenquotient} haben wir schon geklärt, das wir näher an den Anstieg an der Stelle $x$ kommen, wenn wir den Abstand $h$ verringern. Der kleinstmögliche Abstand wäre theoretisch $0$. Das geht allerdings nicht, da wir nicht $h=0$ in die Formel für den Differenzenquotient einsetzen und den Divisor\index{Divisor} somit gleich $0$ setzen dürfen. Um ums zu überlegen, was also für ein minimal kleines $h$ passieren würde, brauchen wir den Limes\index{Limes}.
				\begin{flalign*}
				\lim_{h \to 0}	m=\lim_{h \to 0}\frac{f(x+h)-f(x)}{h}=f^{\prime}(x)&&
				\end{flalign*}
				Da setzen wir jetzt unsere Funktion $f(x)=x^3+2$ ein und formen solange um, bis wir das $h$ aus dem Divisor\index{Divisor} kriegen, damit wir für $h$ die Zahl $0$ einsetzen können. Hinweis: Nachdem du für $h$ die Zahl $0$ eingesetzt hast, darfst du nicht mehr Limes\index{Limes} davor schreiben!
				\begin{flalign*}
				&\lim_{h \to 0}\frac{(x+h)^3+2-x^3-2}{h}\\
				=&\lim_{h \to 0}\frac{x^3+3hx^2+2xh^2+h^3+2-x^3-2}{h}\\
				=&\lim_{h \to 0}\frac{3hx^2+2xh^2+h^3}{h}\\
				=&\lim_{h \to 0}(3x^2+2xh+h^2)\\
				=&3x^2&&
				\end{flalign*}
				Das, was wir jetzt haben ist die erste Ableitung\index{Ableitung} $f^{\prime}(x)$. Sofern nicht anders gewünscht, kann man diese oft auch wesentlich leichter bestimmen, indem man die \highlight{subsec:ableitung}{Ableitungsregeln}\index{Ableitungsregel} kennt.
			\subsubsection{Extrem- und Wendepunkte}
				\label{subsubsec:extrema}
				Leitet man die Ableitung einer Funktion noch mal ab, erhält man die zweite Ableitung\index{Zweite Ableitung} $f^{\prime\prime}(x)$. Ebenso verhält es sich mit der dritten Ableitung\index{Dritte Ableitung} und allen weiteren. In der folgenden Abbildung sieht man eine Funktion\index{Funktion} und ihre erste, zweite sowie dritte Ableitung.
				\makeplot{{x^4+2*x^3-2*x^2+x-3},{4*x^3+6*x^2-4*x+1},{12*x^2+12*x-4},{24*x+12}}{{$f(x)=x^4+2x^3-2x^2+x-3$},{$f^{\prime}(x)=4x^3+6x^2-4x+1$},{$f^{\prime\prime}(x)=12x^2+12x-4$},{$f^{\prime\prime\prime}(x)=24x+12$}}{{12,2.4},{11.7,1.75},{11.4,1.1},{10.9,0.45}}{-6,6}{-20,10}{-30:20}{200}{17cm,7cm}{smooth}
				Man spricht bei den Bedingungen zur Bestimmung besonderer Punkte von notwendigen Bedingungen\index{Notwendige Bedingung}. Es existieren zudem weitere Bedingungen, mit deren Hilfe man diese Punkt genauer untersuchen kann, die sogenannten hinreichenden Bedingungen\index{Hinreichende Bedingung}. Die obige Abbildung soll helfen, diese Bedingungen nachzuvollziehen.
				\begin{tcolorbox}[boxsep=0pt,top=.75cm,left=1cm,right=1cm, bottom=.65cm,arc=0pt,auto outer arc,colback=white,colframe=black, enlarge top by=.45cm, enlarge bottom by=.25cm]
					\textbf{Notwendige Bedingungen}\newline\newline
					Wenn \colorbox{blue!30}{$f^{\prime}(x)=0$} gilt, dann ist bei $x$ ein \colorbox{blue!30}{Extremal- o. Sattelpunkt} von $f$.\newline\newline
					Wenn \colorbox{red!30}{$f^{\prime\prime}(x)=0$} gilt, dann ist bei $x$ ein \colorbox{red!30}{Wendepunkt} von $f(x)$.
				\end{tcolorbox}
				\begin{tcolorbox}[boxsep=0pt,top=.75cm,left=1cm,right=1cm, bottom=.65cm,arc=0pt,auto outer arc,colback=white,colframe=black, enlarge top by=.45cm, enlarge bottom by=.25cm]
					\textbf{Hinreichende Bedingungen}\newline\newline
					Wenn \colorbox{teal!30}{$f^{\prime}(x)=0\land f^{\prime\prime}(x)\neq 0$} gilt, besitzt $f$ eine \colorbox{teal!30}{Extremalpunkt} bei $x$.\newline\newline
					Wenn \colorbox{yellow!30}{$f^{\prime}(x)=0\land f^{\prime\prime}(x)>0$} gilt, dann ist bei $x$ ein \colorbox{yellow!30}{Minimum} von $f$.\newline\newline
					Wenn \colorbox{green!30}{$f^{\prime}(x)=0\land f^{\prime\prime}(x)<0$} gilt, dann ist bei $x$ ein \colorbox{green!30}{Maximum} von $f$.\newline\newline\newline
					Wenn \colorbox{violet!30}{$f^{\prime\prime}(x)=0\land f^{\prime}(x)=0$} gilt, besitzt $f$ einen \colorbox{violet!30}{Sattelpunkt} bei $x$.\newline\newline
					Wenn \colorbox{purple!30}{$f^{\prime\prime}(x)=0\land f^{\prime\prime\prime}(x)<0$} gilt, dann ist bei $x$ eine \colorbox{purple!30}{Links-Rechts-Krümmung.}\newline\newline
					Wenn \colorbox{orange!30}{$f^{\prime\prime}(x)=0\land f^{\prime\prime\prime}(x)>0$} gilt, dann ist bei $x$ eine \colorbox{orange!30}{Rechts-Links-Krümmung.}
				\end{tcolorbox}
				\noindent Hinweis: Das Zeichen $\land$ bedeutet »und«, während $\lor$ »oder« bedeutet.
	\subsection{Limes: Der Grenzwert}
	\label{subsec:limes}
	Für manche Funktionen ist es nicht möglich den Werte einer bestimmten Stelle zu errechnen. Manchmal will man auch das Verhalten einer Funktion wissen, wenn $x$ gegen Unendlich\index{Unendlich} geht. In u.a. diesen Fällen braucht man dem Limes\index{Limes}. Um den Grenzwert\index{Grenzwert} einer Funktion für $x\to a$ zu ermitteln muss man den links- und den rechtsseitigen Grenzwert\index{Linksseitiger Grenzwert}\index{Rechtsseitiger Grenzwert} betrachten. Es gilt Folgendes: 
	\begin{tcolorbox}[boxsep=0pt,top=.75cm,left=1cm,right=1cm, bottom=.65cm,arc=0pt,auto outer arc,colback=white,colframe=black, enlarge top by=.45cm, enlarge bottom by=.25cm]
		Um zu überprüfen, ob für eine Funktion mit $x$ gegen $a$ ein Grenzwert existiert, schaut man sich jeweils den linksseitigen und rechtsseitigen Grenzwert an. Sind diese gleich, so existiert ein Grenzwert. Sind sie unterschiedlich, existiert kein Grenzwert.
	\end{tcolorbox}
	\noindent Beim Grenzwert setzt man Zahlen ein, die sich in die Richtung von $a$ bewegen. Nehmen wir beispielsweise $f(x)=\frac{1}{x}$. $0$ können wir ja nicht für $x$ einsetzen, da wir nicht durch Null teilen dürfen. Was wir aber machen können ist sehr kleine Zahlen einsetzen, um zu schauen, ob wir eine Tendenz feststellen können.
	\begin{flalign*}
		f(1)&=\frac{1}{1}=1\\
		f(0,1)&=\frac{1}{0,1}=10\\
		f(00,1)&=\frac{1}{00,1}=100\\
		f(000,1)&=\frac{1}{000,1}=1000\\
		\lim_{x\searrow 0}\frac{1}{x}&=\infty&&
	\end{flalign*}
	Wir sehen also, dass sich unsere Funktion für $x$ gegen Null Unendlich\index{Unendlich} nähert. Das, was wir uns jetzt angeguckt haben, ist aber nur der \text{rechtsseitige Grenzwert}, da wir Werte größer als $0$ eingesetzt haben und uns somit von rechts angenähert haben. Jetzt machen wir das Ganze noch einmal von links.
	\begin{flalign*}
		f(-1)&=\frac{1}{-1}=-1\\
		f(-0,1)&=\frac{1}{-0,1}=-10\\
		f(-00,1)&=\frac{1}{-00,1}=-100\\
		f(-000,1)&=\frac{1}{-000,1}=-1000\\
		\lim_{x\nearrow 0}\frac{1}{x}&=-\infty&&
	\end{flalign*}
	Wir sehen, dass $\infty \neq -\infty$ gilt. Somit haben wir keinen Grenzwert für $x\to 0$. Übrigens gibt es für den links- und rechtsseitigen Grenzwert\index{Linksseitiger Grenzwert}\index{Rechtsseitger Grenzwert} unterschiedliche Notationen\index{Notation}. Ich benutze hier, die Variante mit den diagonalen Pfeilen, da ich finde, dass sie gut darstellt, was man bei der Annäherung mit dem $x$ anstellt.
	\begin{tcolorbox}[boxsep=0pt,top=.75cm,left=1cm,right=1cm, bottom=.65cm,arc=0pt,auto outer arc,colback=white,colframe=black, enlarge top by=.45cm, enlarge bottom by=.25cm]
		\begin{multicols}{2}
			\begin{flushleft}
				\textbf{Linksseitiger Grenzwert}
				\begin{flalign*}
				\lim_{x\to a^-}\;oder\;\lim_{x\uparrow a}\;oder\;\lim_{x\nearrow a}\;oder\;\lim_{\substack{x\to a\\ x<a}}&&
				\end{flalign*}
			\end{flushleft}
			\begin{flushleft}
				\textbf{Rechtsseitiger Grenzwert}
				\begin{flalign*}
				\lim_{x\to a^+}\;oder\;\lim_{x\downarrow a}\;oder\;\lim_{x\searrow a}\;oder\;\lim_{\substack{x\to a\\ x>a}}&&
				\end{flalign*}
			\end{flushleft}
		\end{multicols}
	\end{tcolorbox}
	\subsection{Differenzierbarkeit}
	\begin{center}
		\begin{tikzpicture}
		\begin{axis}[
		domain=-10:10,
		width=17cm,
		height=7cm,
		restrict y to domain=-20:20,
		xmin=-10, xmax=10,
		ymin=-5, ymax=15,
		samples=200,
		axis y line=center,
		axis x line=middle,
		ticklabel style={fill=white},
		minor tick num=2,
		grid=both,
		grid style={line width=.1pt, draw=gridgray!10},
		major grid style={line width=.2pt,draw=gridgray!50}
		]
		\addplot+[mark=none, color=blue, solid, smooth, domain=-10:2] {x^2};
		\addplot+[mark=none, color=blue, solid, smooth, domain=2:10] {0.5*x^2};
		
		
		\end{axis}
		\node [color=blue] at (12.1,2.1) {$f(x)=\left\{x^2;\;\;\;\;\;\;\;x<2\atop\frac{1}{2}x^2;\;\;\;\;x\ge2\right.$};
		
		\end{tikzpicture}
	\end{center}
	Was uns an dem obigen Beispiel jetzt speziell interessiert, ist die Stelle $x=2$. Wir wollen wissen, ob $f$ in dieser Stelle stetig\index{Stetig} bzw. differenzierbar\index{Differenzierbar} ist. Das heißt quasi, dass wir wissen wollen, ob man die Funktion zeichnen kann, ohne den Stift abzusetzen. Aber Achtung: Das ist keine sehr akkurate Definition, denn es gibt auch Funktionen, die stetig sind, obwohl man sie nicht durchzeichnen kann. Deshalb hier die mathematischen Bedingungen, die erfüllt sein müssen.
	\begin{tcolorbox}[boxsep=0pt,top=.75cm,left=1cm,right=1cm, bottom=.65cm,arc=0pt,auto outer arc,colback=white,colframe=black, enlarge top by=.45cm, enlarge bottom by=.25cm]
		Wenn eine Funktion $f(x)$ an der Stelle $x_0$ folgende Bedingungen erfüllt, so ist sie in dieser Stelle \textit{differenzierbar} bzw. \textit{stetig}.
		\begin{flalign*}
			&1.\;x_0 \in \mathbb{D}\\\\
			&2.\;\lim_{x\to x_0}f(x)\;existiert\\\\
			&3.\;\lim_{x\to x_0}f(x) = f(x_0)&&
		\end{flalign*}
		Sind die obigen Bedingungen für alle $x$ der Definitionsmenge erfüllt, so spricht man von einer \textit{stetigen Funktion}.
	\end{tcolorbox}
	\noindent Schauen wir uns das einmal für unser Beispiel an. Die erste Bedingung ist erfüllt, denn für $x=2$ ist $x$ definiert und es gilt $f(2)=\frac{1}{2}(2)^2=2$. Als nächsten prüfen wir die zweite Bedingung.
	\begin{multicols}{2}
		\noindent\begin{flalign*}
		\lim_{x\nearrow 2}f(x)&=\lim_{x\nearrow 2}x^2=2^2=4&&
		\end{flalign*}
		\begin{flalign*}
		\lim_{x\searrow 2}f(x)&=\lim_{x\searrow 2}\frac{1}{2}x^2=\frac{1}{2}\cdot 2^2=2&&
		\end{flalign*}
	\end{multicols}
	\noindent Unsere zweite Bedingung ist somit nicht erfüllt. Der linksseitige\index{Linksseitiger Grenzwert} und der rechtsseitige Grenzwert\index{Rechtsseitiger Grenzwert} sind unterschiedlich und somit existiert an dieser Stelle auch kein Grenzwert. Daher brauchen wir die letzte Bedingung gar nicht erst überprüfen und können es sogar nicht, da uns der Grenzwert\index{Grenzwert} fehlt.
	\subsubsection{Stetige Erweiterung}
	Eine Funktion wie z.~B. $f(x)=\frac{1}{x}$ hat zwar keinen Grenzwert für $x\to 0$, allerdings ist sie trotzdem stetig, da ihr Definitionsbereich $\mathbb{D}=\mathbb{R}\setminus\{0\}$ die $0$ ausschließt. Das ist wichtig zu wissen bei der Bestimmung der Differenzierbarkeit.
	\makeplot{{1/x}}{{$f(x)=\frac{1}{x}$}}{{5,1.4}}{-10,10}{-10,10}{-25:25}{300}{17cm,7cm}{smooth}
	Jetzt kann es aber sein, dass wir gerne die Definitionslücken\index{Definitionslücke} unserer Funktion definieren wollen. Das geht sogar mithilfe der stetigen Erweiterung\index{Stetige Erweiterung}, allerdings nicht für jede Lücke. Eine Lücke, die man bestimmen kann, nennt man (be)hebbar\index{Behebbar}\index{Hebbar}. Damit eine Lücke behebbar ist, müssen die Bedingungen für die Differenzierbarkeit gegeben sein, außer der, dass $x_0$ nicht im Definitionsbereich\index{Definitionsbereich} liegt. Würde $x$ im Definitionsbereich liegen hätten wir natürlich auch keine Lücke. Logisch. Das Kriterium, dass der Grenzwert dem Funktionswert an der Stelle $x_0$ entspricht entfällt dementsprechend auch.
	\begin{tcolorbox}[boxsep=0pt,top=.75cm,left=1cm,right=1cm, bottom=.65cm,arc=0pt,auto outer arc,colback=white,colframe=black, enlarge top by=.45cm, enlarge bottom by=.25cm]
		Ist eine Lücke einer Funktion \textit{(be)hebbar}, so sind folgende Bedingungen erfüllt:
		\begin{flalign*}
		&1.\;x_0 \notin \mathbb{D}\\\\
		&2.\;\lim_{x\to x_0}f(x)\;existiert&&
		\end{flalign*}
	\end{tcolorbox}
	\noindent Für $x\to 0$ bei $f(x)=\frac{1}{x}$ haben wir keinen Grenzwert, somit ist die Funktion an der Stelle $x=0$ nicht stetig erweiterbar. Es folgt noch ein Beispiel einer stetig erweiterbaren\index{Erweiterbar} Lücke.
	\makeplot{{((x^3)/x)-5}}{{$f(x)=\frac{x^3}{x}-5$}}{{12.2,4.6}}{-10,10}{-10,10}{-25:25}{300}{17cm,7cm}{smooth}
	Diese Funktion verhält sich jetzt wie eine Normalparabel. Leider ist sie für $x=0$ nicht definiert, da wir nicht durch $0$ teilen dürfen, also schauen wir, ob wir sie an dieser Stelle stetig erweitern können. In diesem Fall können wir uns den Test für links- und rechtsseitigen Grenzwert\index{Grenzwert} sparen, denn ich denke, man erkennt hier, dass wir einen Grenzwert haben.
	\begin{flalign*}
		\lim_{x\to 0}\frac{x^3}{x}-5&=\lim_{x\to 0}(x^2-5)\\
		&=-5&&
	\end{flalign*}
	Mit dem Grenzwert können wir jetzt eine zusammengesetzte Funktion\index{Zusammengesetzte Funktion} aufstellen.
	\begin{flalign*}
		f(x)=\genfrac{\{}{.}{0pt}{}{\frac{x^3}{x}}{-5}\genfrac{}{}{0pt}{}{\text{ falls } x\neq 0}{\text{ falls } x=0}&&
	\end{flalign*}
	\subsection{Tangentengleichung}
	\label{subsec:tangentengleichung}
		Wenn man Extremal- und Wendepunkte\index{Extremalpunkt}\index{Wendepunkt} untersucht, kann es vorkommen, dass man eine Tangenten\index{Tangente} für diese Punkte\index{Punkt} aufstellen soll. Das ist nicht schwer, denn man muss nur zwei Konstanten bestimmen. Da eine Tangente eine lineare Funktion\index{Lineare Funktion} ist, hat sie die grundlegende Form $T(x)=mx+n$. Sagen wir, wir wollen die Tangentengleichung\index{Tangentengleichung} an der Stelle $x=2$ der Funktion $f(x)=x^2+4$ aufstellen. Um $m$ zu bestimmen brauchen wir den Anstieg\index{Anstieg} an der Stelle $x$. Diesen können wir mit der ersten Ableitung\index{Ableitung} an der Stelle $x$ ermitteln.
		\begin{flalign*}
			f(x)&=x^2+4\\
			f^{\prime}(x)&=2x\\
			f^{\prime}(2)&=4&&
		\end{flalign*}
		Das können wir schon mal in unsere Gleichung einsetzen und erhalten $T(x)=4x+n$. Jetzt fehlt uns noch $n$, welches wir durch Umstellen\index{Umstellen} bestimmen können, nachdem wir $x$ und $T(x)$ eingesetzt haben. Gesucht ist ja die Tangente an der Stelle $x=2$. Den $x$-Wert haben wir also schon mal und $f(x)$ bzw. $T(x)$ (Punkt\index{Punkt} existiert auf beiden Funktionen\index{Funktion}) können wir ganz einfach durch Einsetzen\index{Einsetzen} in die Funktion berechnen.
		\begin{flalign*}
			f(2)&=8=T(2)\\
			T(2)&=4\cdot 2+n\\
			8&=4\cdot 2+n\\
			n&=0&&
		\end{flalign*}
		Da $n=0$ ist, können wir es weglassen und haben bereits unsere vollständige Tangentengleichung\index{Tangentengleichung}. Hier nochmal eine Abbildung, um zu zeigen, dass das Ergebnis\index{Ergebnis} auch wirklich richtig ist.
		\makeplot{{x^2+4},{4*x}}{{$f(x)=x^2+4$},{$T(x)=4x$}}{{4,3},{10.5,2}}{-10,10}{0,20}{-5:25}{200}{17cm,7cm}{smooth}
	\subsection{Kurvendiskussion}
		In einer Kurvendiskussion untersucht man verschiedene Eigenschaften einer Funktion. Wie man diese Eigenschaften jeweils untersucht wird an anderen Stellen erklärt, die auch noch mal genannt werden. Im Folgenden wird einmal eine komplette Kurvendiskussion\index{Kurvendiskussion} beispielhaft durchgeführt. Da man normalerweise keine Abbildung zur Verfügung hat, weil man die Funktion selber skizzieren soll, gibt es den Graphen\index{Graph} der Funktion erst am Ende. Die zu untersuchende Funktion ist:\newline\newline
		$f(x)=x^{5}-3x^{3}+2x$
		\subsubsection{Symmetrie}
			Für diesen Teil der Kurvendiskussion solltest du wissen, wie man die \highlight{subsubsec:symmetrie}{Symmetrie} einer Funktion bestimmt.\newline\newline
			\textbf{Aufgabe:}\newline Bestimme begründet, ob die Funktion gerade\index{Gerade}, ungerade\index{Ungerade} oder weder noch ist.\newline\newline
			\textbf{Lösung:}\newline Dazu müssen wir $f(-x)$ betrachten. Ist es gleich $f(x)$ haben wir eine gerade Funktion, ist es gleich $-f(x)$ haben wir eine ungerade Funktion, ansonsten haben wir weder noch.
			\begin{flalign*}
				f(-x)&=(-x)^5-3\cdot (-x)^3+2\cdot (-x)\\
				f(-x)&=-x^5+3x^3-2x\\
				-f(x)&=-(x^5-3x^3+2x)\\
				-f(x)&=-x^5+3x^3-2x&&
			\end{flalign*}
			Da $f(-x)=-f(x)$ gilt, liegt hier eine ungerade Funktion vor.
		\subsubsection{Nullstellen}
			Für diesen Teil der Kurvendiskussion solltest du wissen, wie man \highlight{sec:gleichungenvereinfachen}{Gleichungen}\index{Gleichung} lösen kann.\newline\newline
			\textbf{Aufgabe:}\newline Bestimme alle Nullstellen\index{Nullstelle} der Funktion.\newline\newline
			\textbf{Lösung:}
			\begin{tcolorbox}[boxsep=0pt,top=0cm,left=0cm,right=20cm, bottom=0cm,arc=0pt,auto outer arc,colback=white,colframe=white]
				\begin{flalign*}
					&&x^5-3x^3+2x&=0&&\mid \text{Faktorisierung durch Ausklammern von }x\\
					\Leftrightarrow &&x(x^4-3x^2+2)&=0\\
					\Rightarrow &&x^4-3x^2+2&=0&&\mid \text{Substitution: }u=x^2\\
					\Rightarrow &&u^2-3u+2&=0&&\mid \text{p-q-Formel}\\
					\Rightarrow &&u_{1,2}&=\frac{3}{2}\pm\sqrt{\left(-\frac{3}{2}\right)^2-2}\\
					\Rightarrow &&u_1&=2&&\mid \text{Resubstitution: }u=x^2\\
					\Rightarrow &&x^2&=2\\
					\Rightarrow &&x_{1,2}&=\pm\sqrt{2}\\
					\Rightarrow &&u_2&=1&&\mid \text{Resubstitution: }u=x^2\\
					\Rightarrow &&x^2&=1\\
					\Rightarrow &&x_{3,4}&=\pm 1&&
				\end{flalign*}
			\end{tcolorbox}
			\begin{flalign*}
				\mathbb{L}=\left\{-\sqrt{2};-1;0;1;\sqrt{2}\right\}&&
			\end{flalign*}
		\subsubsection{Schnittpunkt mit y-Achse}
			Den $y$-Achsenschnittpunkt\index{Achsenschnittpunkt} zu bestimmen, ist ganz einfach. Man muss nur $x=0$ einsetzen\index{Einsetzen} und das Ergebnis\index{Ergebnis} ausrechnen.\newline\newline
			\textbf{Aufgabe:}\newline Bestimme den Schnittpunkt mit der $y$-Achse der Funktion.\newline\newline
			\textbf{Lösung:}\newline
			In diesem Fall müssen wir eigentlich gar nicht rechnen, da wir bereits wissen, dass bei $x=0$ eine Nullstelle\index{Nullstelle} vorliegt. Trotzdem ist hier noch einmal der rechnerische Nachweis.
			\begin{flalign*}
				f(x)&=x^{5}-3x^{3}+2x\\
				\Rightarrow\qquad f(0)&=0^{5}-3\cdot 0^{3}+2\cdot 0\\
				\Leftrightarrow\qquad f(0)&=0&&
			\end{flalign*}
			Damit ist der Schnittpunkt mit der $y$-Achse der Punkt $S(0\mid 0)$.
		\subsubsection{Grenzverhalten}
			Für diesen Teil der Kurvendiskussion solltest du wissen, wie man den \highlight{subsec:limes}{Limes}\index{Limes} benutzt.\newline\newline
			\textbf{Aufgabe:}\newline Bestimme das Verhalten der Funktion für $x\to\pm\infty$.\newline\newline
			\textbf{Lösung:}\newline
			Zunächst stellen wir die Funktion so um\index{Umstellen}, dass wir den Grenzwert\index{Grenzwert} jedes Terms\index{Term} leicht einzeln betrachten können.
			\begin{tcolorbox}[boxsep=0pt,top=0cm,left=0cm,right=20cm, bottom=0cm,arc=0pt,auto outer arc,colback=white,colframe=white]
				\begin{flalign*}
				&&f(x)=&x^5-3x^3+2x&&\mid x^5\text{ ausklammern}\\
				\Leftrightarrow &&=&x^5\left(\frac{x^5}{x^5}-\frac{3x^3}{x^5}+\frac{2x}{x^5}\right)\\
				\Leftrightarrow &&=&x^5\left(1-\frac{3}{x^2}+\frac{2}{x^4}\right)&&
				\end{flalign*}
			\end{tcolorbox}
			\noindent Durch das Umstellen ist der Grenzwert der Funktion wesentlich einfacher zu zeigen.
			\begin{flalign*}
			&\lim_{x\to -\infty}x^5\left(1-\frac{3}{x^2}+\frac{2}{x^4}\right)=-\infty\\
			&\lim_{x\to \infty}x^5\left(1-\frac{3}{x^2}+\frac{2}{x^4}\right)=\infty&&
			\end{flalign*}
		\subsubsection{Extrema}
			Für diesen Teil der Kurvendiskussion solltest du wissen, wie man \highlight{subsubsec:extrema}{Extrema} bestimmt.\newline\newline
			\textbf{Aufgabe:}\newline Bestimme alle Extrempunkte\index{Extrempunkt} der Funktion und gib an, ob es sich jeweils um Maxima oder Minima handelt.\newline\newline
			\textbf{Lösung:}\newline
			In den folgenden Abschnitten werden wir die Ableitungen\index{Ableitung} benötigen, daher hier einmal alle Funktionen gesammelt:
			\begin{tcolorbox}[boxsep=0pt,top=0cm,left=0cm,right=20cm, bottom=0cm,arc=0pt,auto outer arc,colback=white,colframe=white]
				\begin{flalign*}
				&&f(x)=&x^{5}-3x^{3}+2x\\
				\Rightarrow &&f^{\prime}(x)=&5x^{4}-9x^{2}+2\\
				\Rightarrow &&f^{\prime\prime}(x)=&20x^{3}-18x\\
				\Rightarrow &&f^{\prime\prime\prime}(x)=&60x^{2}-18&&
				\end{flalign*}
			\end{tcolorbox}
			\noindent Notwendige Bedingung\index{Notwendige Bedingung} für Extrempunkte: $f^{\prime}(x)=0$
			\begin{tcolorbox}[boxsep=0pt,top=0cm,left=0cm,right=20cm, bottom=0cm,arc=0pt,auto outer arc,colback=white,colframe=white]
				\begin{flalign*}
				&&f^{\prime}(x)=&5x^{4}-9x^{2}+2\\
				\Rightarrow &&0=&5x^{4}-9x^{2}+2&&\mid \text{Substitution: }u=x^2\\
				\Rightarrow &&0=&5u^{2}-9u+2&&\mid :5\\
				\Leftrightarrow &&0=&u^{2}-1,8u+0,4&&\mid \text{p-q-Formel}\\
				\Rightarrow &&u_{1,2}=&\frac{1,8}{2}\pm\sqrt{\left(-\frac{1,8}{2}\right)^2-0,4}\\
				\Rightarrow &&u_1=&\frac{9+\sqrt{41}}{10}&&\mid \text{Resubstitution: }u=x^2\\
				\Rightarrow &&x^2=&\frac{9+\sqrt{41}}{10}\\
				\Rightarrow &&x_{1,2}=&\pm 1,241093237\\
				\Rightarrow &&u_2=&\frac{9-\sqrt{41}}{10}&&\mid \text{Resubstitution: }u=x^2\\
				\Rightarrow &&x^2=&\frac{9-\sqrt{41}}{10}\\
				\Rightarrow &&x_{3,4}=&\pm 0,5095955026&&
				\end{flalign*}
			\end{tcolorbox}
			\noindent Durch Einsetzen\index{Einsetzen} der ermittelten $x$-Werte, bekommen wir auch die $y$-Werte für die Punkte\index{Punkt}.
			\begin{flalign*}
				\mathbb{L}=\{&(-1,241093237;0,3082564193),(-0,5095955026;-0,656550059),\\
				&(0,5095955026;-0,656550059),(1,241093237;0,3082564193)\}&&
			\end{flalign*}
			Hinreichende Bedingung\index{Hinreichende Bedingung} für Maxima: $f^{\prime}(x)=0\land f^{\prime\prime}(x)<0$\newline
			Hinreichende Bedingung für Minima: $f^{\prime}(x)=0\land f^{\prime\prime}(x)>0$
			\begin{flalign*}
				&f^{\prime\prime}(-1,241093237)=20\cdot (-1,241093237)^{3}-18\cdot (-1,241093237)=-15,89374835<0\\
				&f^{\prime\prime}(-0,5095955026)=20\cdot (-0,5095955026)^{3}-18\cdot (-0,5095955026)=6,526006628>0&&
			\end{flalign*}
			Da unsere Funktion ungerade\index{Ungerade} ist, können wir darauf schließen, dass die Extrema auf der anderen Seite der $y$-Achse genau gegenteilig sind. Daher gilt also:
			\begin{tcolorbox}[boxsep=0pt,top=0cm,left=0cm,right=20cm, bottom=0cm,arc=0pt,auto outer arc,colback=white,colframe=white]
				\begin{flalign*}
					&x=-1,241093237\Rightarrow&\text{Maximalstelle}\\
					&x=-0,5095955026\Rightarrow&\text{Minimalstelle}\\
					&x=0,5095955026\Rightarrow&\text{Maximalstelle}\\
					&x=1,241093237\Rightarrow&\text{Minimalstelle}&&
				\end{flalign*}
			\end{tcolorbox}
		\subsubsection{Wendepunkte}
			Für diesen Teil der Kurvendiskussion solltest du wissen, wie man \highlight{subsubsec:extrema}{Wendepunkte} bestimmt.\newline\newline
			\textbf{Aufgabe:}\newline Bestimme alle Wendepunkte\index{Wendepunkt} der Funktion und gib an, ob gib ihr Krümmungsverhalten\index{Krümmungsverhalten} an.\newline\newline
			\textbf{Lösung:}\newline
			Notwendige Bedingung\index{Notwendige Bedingung} für Wendepunkte: $f^{\prime\prime}(x)=0$
			\begin{tcolorbox}[boxsep=0pt,top=0cm,left=0cm,right=20cm, bottom=0cm,arc=0pt,auto outer arc,colback=white,colframe=white]
				\begin{flalign*}
				&&f^{\prime\prime}(x)=&20x^{3}-18x\\
				\Rightarrow &&0=&20x^{3}-18x&&\mid \text{Faktorisierung durch Ausklammern von }x\\
				\Rightarrow &&0=&x(20x^{2}-18)&&\mid \text{Ein Wendepunkt ist }x_1=0\\
				\Rightarrow &&0=&20x^{2}-18&&\mid +18\\
				\Leftrightarrow &&18=&20x^{2}&&\mid :20\\
				\Leftrightarrow &&\frac{9}{10}=&x^{2}&&\mid \sqrt{\ }\\
				\Rightarrow &&x_{2,3}=&\pm\sqrt{\frac{9}{10}}&&
				\end{flalign*}
			\end{tcolorbox}
			\begin{flalign*}
				\mathbb{L}=\left\{\left(-\sqrt{\frac{9}{10}};-0,1043551628\right),(0;0),\left(\sqrt{\frac{9}{10}};0,1043551628\right)\right\}&&
			\end{flalign*}
			\noindent Hinreichende Bedingung\index{Hinreichende Bedingung} für Links-Rechts-Krümmung\index{Links-Rechts-Krümmung}: $f^{\prime}(x)=0\land f^{\prime\prime\prime}(x)<0$\newline
			Hinreichende Bedingung für Rechts-Links-Krümmung\index{Rechts-Links-Krümmung}: $f^{\prime}(x)=0\land f^{\prime\prime\prime}(x)>0$
			\begin{flalign*}
				&f^{\prime\prime\prime}\left(-\sqrt{\frac{9}{10}}\right)=60\cdot\left(-\sqrt{\frac{9}{10}}\right)^2-18=36>0\\
				&f^{\prime\prime\prime}(0)=60\cdot(0)^2-18=-18<0\\
				&f^{\prime\prime\prime}\left(\sqrt{\frac{9}{10}}\right)=60\cdot\left(\sqrt{\frac{9}{10}}\right)^2-18=36>0&&
			\end{flalign*}
			\begin{tcolorbox}[boxsep=0pt,top=0cm,left=0cm,right=20cm, bottom=0cm,arc=0pt,auto outer arc,colback=white,colframe=white]
				\begin{flalign*}
					&x=-\sqrt{\frac{9}{10}}\Rightarrow&\text{Rechts-Links-Krümmung}\\
					&x=0\Rightarrow&\text{Links-Rechts-Krümmung}\\
					&x=\sqrt{\frac{9}{10}}\Rightarrow&\text{Rechts-Links-Krümmung}&&
				\end{flalign*}
			\end{tcolorbox}
		\subsubsection{Tangentengleichungen der Wendepunkte}
			Für diesen Teil der Kurvendiskussion solltest du wissen, wie man eine \highlight{subsec:tangentengleichung}{Tangentengleichung} aufstellt.\newline\newline
			\textbf{Aufgabe:}\newline Stelle eine Tangentengleichung\index{Tangentengleichung} für die Wendestelle\index{Wendestelle} mit dem kleinsten $x$-Wert auf.\newline\newline
			\textbf{Lösung:}\newline
			Die Grundgleichung für eine lineare Funktion\index{Lineare Funktion} ist $T(x)=mx+n$. Wir müssen dafür $m$ und $n$ bestimmen. Den Anstieg $m$ erhalten wir aus der ersten Ableitung\index{Ableitung} an der Wendestelle\index{Wendestelle}.
			\begin{tcolorbox}[boxsep=0pt,top=0cm,left=0cm,right=20cm, bottom=0cm,arc=0pt,auto outer arc,colback=white,colframe=white]
				\begin{flalign*}
					&&f^{\prime}(x)=&5x^{4}-9x^{2}+2\\
					\Rightarrow &&f^{\prime}\left(-\sqrt{\frac{9}{10}}\right)=&5\left(-\sqrt{\frac{9}{10}}\right)^{4}-9\left(-\sqrt{\frac{9}{10}}\right)^{2}+2\\
					\Leftrightarrow &&=&-\frac{41}{20}&&
				\end{flalign*}
			\end{tcolorbox}
			Jetzt, wo wir $m$ bestimmt haben, können wir die Koordinaten\index{Koordinate} von unserem Wendepunkt\index{Wendepunkt} einsetzen\index{Einsetzen} und nach $n$ umstellen\index{Umstellen}.
			\begin{tcolorbox}[boxsep=0pt,top=0cm,left=0cm,right=20cm, bottom=.3cm,arc=0pt,auto outer arc,colback=white,colframe=white]
				\begin{flalign*}
				&&-0,1043551628=&-\frac{41}{20}\cdot \left(-\sqrt{\frac{9}{10}}\right)+n\\
				\Leftrightarrow&&-0,1043551628=&1,944800761+n&&\mid -1,944800761\\
				\Leftrightarrow&&n=&-2,049155924&&
				\end{flalign*}
			\end{tcolorbox}
			\noindent Damit ist unsere Tangentengleichung $T(x)=-\frac{41}{20}x-2.049155924$. Mit den ermittelten Informationen können wir folgende Abbildung zeichnen:
			\makeplot{{x^5-3*x^3+2*x},{-(41/20)*x-2.049155924}}{{$f(x)=x^{5}-3x^{3}+2x$},{$T(x)=-\frac{41}{20}x-2.049155924$}}{{2.1,3.5},{10.5,0.6}}{-3,3}{-1,1}{-5:5}{300}{17cm,7cm}{smooth}
		\subsubsection{Flächenberechnung}
			\label{subsubsec:flaechenberechnung}
			Für diesen Teil der Kurvendiskussion solltest du wissen, wie man \highlight{sec:integralrechnung}{Integralrechnung} durchführt.\newline\newline
			\textbf{Aufgabe:}\newline Berechne die Fläche\index{Fläche}, die zwischen den äußersten Nullstellen von Funktion und $x$-Achse eingeschlossen wird. \newline\newline
			\textbf{Lösung:}\newline
			Zunächst müssen wir die Stammfunktion bilden. Diese lautet $F(x)=\frac{1}{6}x^6-\frac{3}{4}x^4+x^2+k$. Anschließend bilden wir das Integral mit den Grenzen $-\sqrt{2}$ und $-1$, sowie mit den Grenzen $0$ und $1$. Welche Flächen wir damit berechnen, kann man an der obigen Skizze gut erkennen. Da durch die Punktsymmetrie der Funktion zum Nullpunkt diese Flächen jeweils zwei Mal vorliegen, müssen wir das Ergebnis nur verdoppeln und haben die Gesamtfläche.
			\begin{flalign*}
				\int^{-1}_{-\sqrt{2}}(x^5-3x^3+2x)dx&=\left[\frac{1}{6}x^6-\frac{3}{4}x^4+x^2+k\right]^{-1}_{-\sqrt{2}}\\
				&=\frac{1}{6}(-1)^6-\frac{3}{4}(-1)^4+(-1)^2+k-\left(\frac{1}{6}(-\sqrt{2})^6-\frac{3}{4}(-\sqrt{2})^4+(-\sqrt{2})^2+k\right)\\
				&=\frac{5}{12}+k-(0,\overline{3}+k)\\
				&=\frac{5}{12}+k-0,\overline{3}-k)\\
				&=\frac{1}{12}&&
			\end{flalign*}
			\begin{flalign*}
				\int^{1}_{0}(x^5-3x^3+2x)dx&=\left[\frac{1}{6}x^6-\frac{3}{4}x^4+x^2+k\right]^{1}_{0}\\
				&=\frac{1}{6}(1)^6-\frac{3}{4}(1)^4+(1)^2+k-\left(\frac{1}{6}(0)^6-\frac{3}{4}(0)^4+(0)^2+k\right)\\
				&=\frac{5}{12}+k-(0+k)\\
				&=\frac{5}{12}+k-0-k)\\
				&=\frac{5}{12}&&
			\end{flalign*}
			\begin{flalign*}
				F=2\cdot \left(\frac{1}{12}+\frac{5}{12}\right)=1\text{ Flächeneinheiten}&&
			\end{flalign*}
	\pagebreak
	\section{Integralrechnung}
	\label{sec:integralrechnung}
		Ein Integral gibt den Flächeninhalt zwischen einer Funktion und der $x$-Achse in einem bestimmten Intervall wieder. Dabei ist zu beachten, dass die Flächen unterhalb der $x$-Achse negativ sind, weshalb man jeden Abschnitt einzeln berechnen und anschließend addieren sollte.
		\begin{center}
			\begin{tikzpicture} 
			\begin{axis}[
			domain=-10:10,
			width=17cm,
			height=7cm,
			xmin=-4, xmax=4,
			ymin=-4, ymax=4,
			samples=100,
			axis y line=center,
			axis x line=middle,
			minor tick num=0,
			grid=both,
			grid style={line width=.1pt, draw=gridgray!10},
			major grid style={line width=.2pt,draw=gridgray!50},
			yticklabels={,,},
			xticklabels={,,}]
			\addplot [name path = A,
			-latex,
			samples = 200,
			domain=-2.2:2.2,
			color=blue] {(x^4)-(4*x^2)+2};
			
			% Plot 2
			\addplot [name path = B,
			-latex] {0};
			
			% Fill area between paths
			\addplot [fill=blue, fill opacity=0.2] fill between [of = A and B, soft clip={domain=-1.848:1.848}, opacity=.2];
			\end{axis}
			\node [color=blue] at (5.1,2.1) {\colorbox{transparentblue}{\Huge $-$}};
			\node [color=blue] at (10.35,2.1) {\colorbox{transparentblue}{\Huge $-$}};
			\node [color=blue] at (7.725,3.3) {\colorbox{transparentblue}{\Huge $+$}};
			\end{tikzpicture}
		\end{center}
		\begin{tcolorbox}[boxsep=0pt,top=.75cm,left=1cm,right=1cm, bottom=.65cm,arc=0pt,auto outer arc,colback=white,colframe=black, enlarge top by=.25cm, enlarge bottom by=.25cm]
			Das unbestimmte Integral\index{Unbestimmtes Integral} ist die Menge der Stammfunktionen einer Funktion $f(x)$:
			\begin{flalign*}
				\int f(x)dx=F(x)+k&&
			\end{flalign*}
		\end{tcolorbox}
		\begin{tcolorbox}[boxsep=0pt,top=.75cm,left=1cm,right=1cm, bottom=.65cm,arc=0pt,auto outer arc,colback=white,colframe=black, enlarge top by=.25cm, enlarge bottom by=.25cm]
			Für das bestimmte Integral\index{Bestimmtes Integral} mit den Integrationsgrenzen $a$ und $b$, sowie der Integrationsvariable $x$ und dem Differential $dx$ gilt folgende Notation:
			\begin{flalign*}
				\int^b_a f(x)dx=F(b)-F(a)=[F(x)]^b_a&&
			\end{flalign*}
		\end{tcolorbox}
		\noindent Um das zu verstehen, müssen wir uns erst einmal angucken, was das überhaupt bedeutet. Wenn wir die Fläche unter einer linearen Gleichung\index{Lineare Gleichung} berechnen wollten, wäre das ziemlich einfach, da wir schnell ein Drei- oder Viereck konstruieren können, dessen Fläche\index{Fläche} wir berechnen. Mit anderen Funktionstypen ist das aber nicht so einfach. Wir können die Fläche annähernd berechnen, wenn wir sie mit Formen füllen, deren Flächen wir einfach berechnen können. Dafür nehmen wir Rechtecke. Die breite der Rechtecke nennen wir $\Delta x$. Das bedeutet quasi nur $x_2-x_1$.
		\begin{center}
			\begin{tikzpicture} 
			\begin{axis}[
			domain=-10:10,
			width=17cm,
			height=7cm,
			xmin=-4, xmax=4,
			ymin=-4, ymax=4,
			samples=100,
			axis y line=center,
			axis x line=middle,
			minor tick num=0,
			grid=both,
			grid style={line width=.1pt, draw=gridgray!10},
			major grid style={line width=.2pt,draw=gridgray!50},
			yticklabels={,,},
			xticklabels={,,}]
			\addplot [name path = A,
			-latex,
			samples = 200,
			domain=-2.2:2.2,
			color=blue] {(x^4)-(4*x^2)+2};
			
			% Plot 2
			\addplot [name path = B,
			-latex] {0};
			
			% Fill area between paths
			\addplot [
			red,
			integral segments=16,
			integral=-1.848:1.848
			] {(x^4)-(4*x^2)+2};
			\end{axis}
			\end{tikzpicture}
		\end{center}
		Wenn wir unsere Rechtecke schmaler und schmaler machen, dann wird unser Ergebnis immer akkurater. Wenn wir unsere Rechtecke minimal klein machen schreiben wir anstatt $\Delta x$ jedoch $dx$.
		\begin{center}
			\begin{tikzpicture} 
			\begin{axis}[
			domain=-10:10,
			width=17cm,
			height=7cm,
			xmin=-4, xmax=4,
			ymin=-4, ymax=4,
			samples=100,
			axis y line=center,
			axis x line=middle,
			minor tick num=0,
			grid=both,
			grid style={line width=.1pt, draw=gridgray!10},
			major grid style={line width=.2pt,draw=gridgray!50},
			yticklabels={,,},
			xticklabels={,,}]
			\addplot [name path = A,
			-latex,
			samples = 200,
			domain=-2.2:2.2,
			color=blue] {(x^4)-(4*x^2)+2};
			
			% Plot 2
			\addplot [name path = B,
			-latex] {0};
			
			% Fill area between paths
			\addplot [
			red,
			integral segments=32,
			integral=-1.848:1.848
			] {(x^4)-(4*x^2)+2};
			\end{axis}
			\end{tikzpicture}
		\end{center}
		Genau diese immer kleiner werdenden Rechteck\index{Rechteck} werden durch das Integral dargestellt. Man rechnet die Stammfunktion an der obereren Grenze minus die Stammfunktion an der unteren Grenze. In der Kurvendiskussion gibt es ein rechnerisches \highlight{subsubsec:flaechenberechnung}{Beispiel} für die Anwendung des Integrals.
		\subsection{Die Stammfunktion}
			Das Gegenteil vom Differenzieren\index{Differenzieren} ist das Integrieren\index{Integrieren}. Damit kann man von einer Ableitung auf die nächst niedrigere Ableitung schließen, z.~B. von $f^{\prime\prime}(x)$ auf $f^{\prime}(x)$. Die Funktion, die man daraus erhält, nennt man Stammfunktion\index{Stammfunktion}. Auch eine nicht abgeleitete Funktion besitzt eine Stammfunktion. Diese wird mit einem großen $F(x)$ bezeichnet. Da das Integrieren das Gegenteil des Differenzieren ist, gelten dieselben \highlight{subsec:ableitung}{Regeln}, jedoch umgekehrt. Anstatt z.~B. den Exponenten um $1$ zu reduzieren und mit ihm zu multiplizieren, wird er um $1$ erhöht und durch ihn geteilt. Beim Integrieren fügt man jedoch noch $k$ für eine beliebige Zahl hinzu, denn beim Ableiten geht diese verloren. Daher sind Stammfunktionen nicht eindeutig, wie es die Ableitungen sind. Die Stammfunktion von $f(x)=2x$ ist beispielsweise $F(x)=x^2+k$.
			\begin{tcolorbox}[boxsep=0pt,top=.75cm,left=1cm,right=1cm, bottom=0cm,arc=0pt,auto outer arc,colback=white,colframe=black, enlarge top by=.45cm, enlarge bottom by=.25cm]
				\textbf{Integrationsregeln}\newline\newline
				Da es manchmal schwierig sein kann, umgekehrt zu denken, hier noch mal die wichtigsten Regeln zur Bildung der Stammfunktion:
				\begin{multicols}{2}
					\noindent\begin{flalign*}
					&c&\rightarrow&\;\;c\cdot x+k\\\\
					&x^n&\rightarrow&\;\;\frac{1}{n+1}x^{n+1}+k\\\\
					&\frac{1}{x}&\rightarrow&\;\;\ln{\vert x\vert}+k\\\\
					&\sqrt[n]{x}&\rightarrow&\;\;\frac{1}{\frac{1}{n}+1}x^{\frac{1}{n}+1}+k\\\\
					&e^x&\rightarrow&\;\;e^x\\\\&&&&&&&&&&&
					\end{flalign*}
					\begin{flalign*}
					&ln(x)&\rightarrow&\;\;-x+x\cdot \ln(x)+k\\\\
					&\sin(x)&\rightarrow&\;\; -\cos(x)+k\\\\
					&\cos(x)&\rightarrow&\;\;\sin(x)+k\\\\
					&\tan(x)&\rightarrow&\;\;-\ln\vert\cos(x)\vert + k\\\\
					&\frac{f^{\prime}(x)}{f(x)}&\rightarrow&\;\;\ln f(x)+k&&&&&&&&&&&
					\end{flalign*}
				\end{multicols}
			\end{tcolorbox}
			\begin{tcolorbox}[boxsep=0pt,top=.75cm,left=1cm,right=1cm, bottom=.65cm,arc=0pt,auto outer arc,colback=white,colframe=black, enlarge top by=.45cm, enlarge bottom by=.25cm]
				\textbf{Die lineare Substitution}\label{Lineare Substitution}\newline\newline
				Haben wir eine verkettete Funktion deren innere Funktion eine lineare Funktion ist, so gilt folgende Integrationsregel:
				\begin{flalign*}
					\int f(mx+b)dx=\frac{1}{m}F(mx+b)+k&&
				\end{flalign*}
			\end{tcolorbox}
		\subsection{Integration durch Substitution}
			Möchte man eine verkettete Funktion integrieren, wird es mit den bisherigen Regeln schwierig. Wir müssen hier substituieren, um eine Stammfunktion zu erzeugen. Schauen wir uns das mal an einem Beispiel an.
			\begin{flalign*}
				\int^0_{-1}\left(\sqrt{1-2x}\right)dx&&
			\end{flalign*}
			Diesen Term können wir so erst einmal nicht integrieren. Wenn wir jetzt jedoch die innere Funktion mit $u$ substituieren, geht das schon. Wir sagen also, dass $u=1-2x$ gilt.
			\begin{flalign*}
				\int\sqrt{u}&&
			\end{flalign*}
			Das war jedoch nicht alles, denn jetzt, wo wir $u$ in unserer Funktion haben, müssen wir unsere Grenzen und das Differenzial anpassen. Das funktioniert, in dem wir die Ableitung\index{Ableitung} von $u$ bilden.
			\begin{flalign*}
				u^{\prime}=-2=\frac{du}{dx}&&
			\end{flalign*}
			Wichtig ist sich zu merken, dass wir sagen können, dass $\frac{du}{dx}$ das Gleiche ist, wie die Ableitung. Jetzt stellen wir nach $dx$ um, damit wir auch das Differenzial ersetzen können.
			\begin{flalign*}
				\int\sqrt{u}\cdot\left(-\frac{1}{2}\right)du&&
			\end{flalign*}
			Letztlich müssen wir nur noch unsere Grenzen anpassen, indem wir $x$ in $u$ einsetzen.
			\begin{flalign*}
			&\int^{u(0)}_{u(-1)}\sqrt{u}\cdot\left(-\frac{1}{2}\right)du\\
			=&\int^1_3\sqrt{u}\cdot\left(-\frac{1}{2}\right)du\\
			=&\left[-\frac{1}{3}u^{\frac{3}{2}}\right]\\
			\approx&1,399&&
			\end{flalign*}
			Hinweis: Hat man ein unbestimmtes Integral, muss man resubstituieren. Also in dem Fall:
			\begin{flalign*}
			\int\left(\sqrt{1-2x}\right)dx=\left[-\frac{1}{3}(1-2x)^{\frac{3}{2}}+k\right]&&
			\end{flalign*}
		\subsection{Partielle Integration}
			Wenn man ein Produkt integrieren möchte, braucht man die partielle Integration. Abgeleitet aus der Produktregel\index{Produktregel} der Differentialrechnung, ergibt sich folgende Formel:
			\begin{tcolorbox}[boxsep=0pt,top=.25cm,left=1cm,right=1cm, bottom=.65cm,arc=0pt,auto outer arc,colback=white,colframe=black, enlarge top by=.45cm, enlarge bottom by=.25cm]
				\begin{flalign*}
					\int f^{\prime}(x)\cdot g(x) dx=f(x)\cdot g(x)-\int f(x)\cdot g^{\prime}(x) dx
				\end{flalign*}
			\end{tcolorbox}
			\noindent Zum besseren Verständnis noch mal ein Beispiel: $\int x\cdot e^xdx$.
			\begin{flalign*}
				&\int (x\cdot e^x)dx\\
				=&e^x\cdot x - \int(e^x\cdot 1)dx\\
				=&e^x\cdot x - e^x+k\\
				=&e^x\cdot (x-1)+k&&
			\end{flalign*}
		\subsection{Die Fläche zwischen zwei Graphen}
			Wollen wir die Fläche zwischen zwei Funktionen berechnen, bilden wir einfach das Integral der Differenzfunktion\index{Differenzfunktion} dieser zwei Funktionen ermitteln.
			\begin{tcolorbox}[boxsep=0pt,top=.25cm,left=1cm,right=1cm, bottom=.65cm,arc=0pt,auto outer arc,colback=white,colframe=black, enlarge top by=.45cm, enlarge bottom by=.25cm]
				\begin{flalign*}
				\int^b_a \vert f(x)-g(x) \vert dx
				\end{flalign*}
			\end{tcolorbox}
			\noindent Achtung: Wenn die Differenzfunktion Nullstellen hat, bzw. die Funktionen sich schneiden, muss die Fläche schrittweise ermittelt werden. Dazu ein Beispiel:\newline\newline
			\begin{center}
				\begin{tikzpicture} 
				\begin{axis}[
				domain=-3:3,
				width=17cm,
				height=7cm,
				xmin=-3, xmax=3,
				ymin=-2, ymax=2,
				samples=100,
				axis y line=center,
				axis x line=middle,
				minor tick num=0,
				grid=both,
				grid style={line width=.1pt, draw=gridgray!10},
				major grid style={line width=.2pt,draw=gridgray!50}]
				\addplot [name path = A,
				-latex,
				samples = 200,
				color=blue] {-x^3+2*x^2};
				

				\addplot [name path = B,
				-latex,
				color=red] {-0.5*x+1};
				
				\addplot [color=teal,
				-latex,
				samples = 200] {-x^3+2*x^2+0.5*x-1};
				
				\addplot [fill=blue, fill opacity=0.2] fill between [of = A and B, soft clip={domain=0:2}, opacity=.2];
				\end{axis}
				\node [color=blue] at (2,0.75) {$f(x)=-x^{3}+2x^{2}$};
				\node [color=red] at (1.8,1.75) {$g(x)=-\frac{1}{2}x+1$};
				\node [color=teal] at (10.5,0.75) {$d(x)=-x^{3}+2x^{2}+\frac{1}{2}x-1$};
				\end{tikzpicture}
			\end{center}
			Gesucht ist die blau markierte Fläche im Intervall $[0,2]$. Wenn wir den gezeichneten Graphen nicht vorliegen hätten, dann müssten wir erst testen, ob eine Nullstelle\index{Nullstelle} in diesem Intervall\index{Intervall} vorliegt. Dazu können wir z.~B. eine Wertetabelle\index{Wertetabelle} anlegen.
			\begin{center}
				\bgroup
				\def\arraystretch{1.5}
				\begin{tabular}{ | l | c | c | c | c | c | }
					\hline
					\textbf{x} & 0 & 0,5 & 1 & 1,5 & 2 \\ \hline
					\textbf{f(x)} & -1 & -0,375 & 0,5 & 0,875 & 0 \\
					\hline
				\end{tabular}
				\egroup
			\end{center}
			Jetzt können wir das \highlight{subsubsec:newtonverfahren}{Newtonverfahren}\label{Newtonverfahren} anwenden, um die Nullstelle zu finden. Dafür brauchen wir die Ableitung\index{Ableitung} der Differenzfunktion\index{Differenzfunktion} $d^{\prime}(x)=-3x^2+4x+\frac{1}{2}$.
			\begin{flalign*}
			x_1&=(0,5)-\frac{-(0,5)^{3}+2(0,5)^{2}+\frac{1}{2}(0,5)-1}{-3(0,5)^2+4(0,5)+\frac{1}{2}}=\frac{5}{7}\\
			x_2&=\left(\frac{5}{7}\right)-\frac{-(\frac{5}{7})^{3}+2(\frac{5}{7})^{2}+\frac{1}{2}(\frac{5}{7})-1}{-3(\frac{5}{7})^2+4(\frac{5}{7})+\frac{1}{2}}=\frac{886}{1253}\\
			x_3&=\left(\frac{886}{1253}\right)-\frac{-(\frac{886}{1253})^{3}+2(\frac{886}{1253})^{2}+\frac{1}{2}(\frac{886}{1253})-1}{-3(\frac{886}{1253})^2+4(\frac{886}{1253})+\frac{1}{2}}=0,7071067812\\
			x_4&=(0,7071067812)-\frac{-(0,7071067812)^{3}+2(0,7071067812)^{2}+\frac{1}{2}(0,7071067812)-1}{-3(0,7071067812)^2+4(0,7071067812)+\frac{1}{2}}\\
			&=0,7071067812&&
			\end{flalign*}
			Der Einfachheit halber kürzen wir die Nullstelle auf $x_0=0,707$, bevor die Teilintegrale berechnen.
			\begin{flalign*}
				\int_0^{0,707}\left(-x^{3}+2x^{2}+\frac{1}{2}x-1\right)dx=\left[-\frac{1}{4}x^{4}+\frac{2}{3}x^{3}+\frac{1}{4}x^2-x+k\right]_0^{0,707}\approx-0,408&&
			\end{flalign*}
			\begin{flalign*}
			\int_{0,707}^2\left(-x^{3}+2x^{2}+\frac{1}{2}x-1\right)dx=\left[-\frac{1}{4}x^{4}+\frac{2}{3}x^{3}+\frac{1}{4}x^2-x+k\right]_{0,707}^2\approx0,742&&
			\end{flalign*}
			Damit ist die gesuchte Fläche $\vert-0,408\vert + 0,742$ bzw. $1,15$ Flächeneinheiten groß.
		\subsection{Rotationsvolumen}
			Neben der Fläche kann man mit dem Integral auch das Volumen berechnen. Dafür muss man nur eine ganz einfache Formel anwenden:
			\begin{tcolorbox}[boxsep=0pt,top=.25cm,left=1cm,right=1cm, bottom=.65cm,arc=0pt,auto outer arc,colback=white,colframe=black, enlarge top by=.45cm, enlarge bottom by=.25cm]
				\begin{flalign*}
				V=\pi\cdot\int_a^b(f(x))^2dx
				\end{flalign*}
			\end{tcolorbox}
			\noindent Dazu ein Beispiel: Gesucht ist das Volumen der abgebildeten Vase, im Bereich $-1$ bis $3$.
			\begin{center}
				\begin{tikzpicture} 
				\begin{axis}[
				domain=-2:5,
				width=17cm,
				height=7cm,
				xmin=-2, xmax=5,
				ymin=-4, ymax=4,
				samples=100,
				axis y line=center,
				axis x line=middle,
				minor tick num=0,
				grid=both,
				grid style={line width=.1pt, draw=gridgray!10},
				major grid style={line width=.2pt,draw=gridgray!50}]
				\addplot [red] coordinates { (3,4) (3,-4)};
				\draw[red] (axis cs:-1,0) ellipse [ x radius=0.15, y radius=2.71];
				\addplot+[mark=none, color=blue, solid, smooth] {1/(e^x)};
				\addplot+[mark=none, color=blue, solid, smooth, dashed] {-1/(e^x)};
				\end{axis}
				\node [color=blue] at (9,3.5) {$f(x)=\frac{1}{e^x}$};
				\end{tikzpicture}
			\end{center}
			\begin{tcolorbox}[boxsep=0pt,top=0cm,left=0cm,right=20cm, bottom=0cm,arc=0pt,auto outer arc,colback=white,colframe=white]
				\begin{flalign*}
					&&f(x)=&\frac{1}{e^x}\\
					&&\Downarrow&&\\
					&&V=&\pi\cdot\int_{-1}^3\left(\frac{1}{e^x}\right)^2dx\\
					\Leftrightarrow&&=&\pi\cdot\left[-e^{-x}+k\right]_{-1}^3\\
					\Leftrightarrow&&=&\pi\cdot(-e^{-3}+k+e^{1}-k)\\
					\Leftrightarrow&&\approx&2,67\cdot\pi\\
					\Leftrightarrow&&\approx&8,38&&
				\end{flalign*}
			\end{tcolorbox}
			\noindent Hinweis: Wenn um die $y$-Achse rotiert werden soll, muss man diese Formel auf die \highlight{subsubsec:umkehrbarkeit}{Umkehrfunktion}\index{Umkehrfunktion} anwenden!
	\pagebreak
	\section{Logik}
		Die Grundlagen der Logik ähneln stark der Logik in Programmiersprachen, jedoch mit andern Symbolen. Die wichtigsten Begriffe sind zunächst einmal Negation\index{Negation} ($\lnot$, „nicht“), Konjunktion\index{Konjunktion} ($\land$,  „und“), Disjunktion\index{Disjunktion} ($\lor$, „oder“), Kontravalenz\index{Kontravalenz} ($\nLeftrightarrow$, „entweder oder“), Äquivalenz\index{Äquivalenz} ($\Leftrightarrow$) und Implikation\index{Implikation} ($\Rightarrow$). Außerdem redet man bei $W$ (wahr) und $F$ (falsch) von Wahrheitswerten\index{Wahrheitswert}. Ob Aussagen\index{Aussage} wahr oder falsch sind, wird oft in Wahrheitstafeln\index{Wahrheitstafel} dargestellt. Dazu eine Übersicht:
		\begin{center}
			\bgroup
			\def\arraystretch{1.5}
			\begin{tabularx}{\linewidth}{|l|l|@{\hspace{0.5cm}}|X|X|X|X|X|X|}
				\hline
				$A$ & $B$ & $\lnot A$ & $A\land B$ & $A\lor B$ &  $A\dot\lor B$ & $A\Leftrightarrow B$ & $A\Rightarrow B$ \\ \hline
				$W$ & $W$ & $F$ & $W$ & $W$ & $F$ & $W$ & $W$ \\ \hline
				$W$ & $F$ & $F$ & $F$ & $W$ & $W$ & $F$ & $F$ \\ \hline
				$F$ & $W$ & $W$ & $F$ & $W$ & $W$ & $F$ & $W$ \\ \hline
				$F$ & $F$ & $W$ & $F$ & $F$ & $F$ & $W$ & $W$ \\ \hline
			\end{tabularx}
			\egroup
		\end{center}
		Übrigens eine Eselsbrücke, um die Zeichen für Konjunktion und Disjunktion zu unterscheiden, sind die drei O: „Oder Oben Offen“. Und noch ein Hinweis: Die Implikation hat kein genaues Äquivalent in der Programmierung. Am besten merkt man sicher, dass eine Implikation immer wahr ist, außer für den Fall $W\Rightarrow F$.\newline\newline
		Wenn eine Formel in jedem Szenario (für alle Bewertungen) wahr ist, dann spricht man von einer Tautologie\index{Tautologie}. Beispielsweise ist $A\lor\lnot A$ eine solche Tautologie. Anders herum nennt man $A\land\lnot A$ einen Widerspruch\index{Widerspruch}.\newline\newline
		Man kann des weiteren Aussagen auch von Elementen einer Menge abhängig machen. Dabei unterscheidet man, für wie viele Elemente die Aussage gelten soll. Dazu noch mal eine Übersicht:
		\begin{center}
			\bgroup
			\def\arraystretch{1.5}
			\begin{tabularx}{\linewidth}{|X|X|X|X|}
				\hline
				Alle $x$ & Mindestens ein $x$ & Genau ein $x$ & Kein $x$ \\ \hline
				$\forall x\in M:A(x)$ & $\exists x\in M:A(x)$ & $\exists! x\in M:A(x)$ & $\nexists x\in M:A(x)$ \\ \hline
			\end{tabularx}
			\egroup
		\end{center}
		Ein Beispiel: $A(x) = x$ ist eine natürliche Zahl\newline
		Die Aussage $\exists x\in \mathbb{Z}:A(x)$ ist wahr, da die ganzen Zahlen die natürlichen Zahlen beinhalten. Wenn die Aussage jedoch $\forall x\in \mathbb{Z}:A(x)$ wäre, wäre sie falsch, da z.~B. $A(-1)$ dem widerspricht. Wenn man mehrere Quantoren\index{Quantor} benutzt, kann man gleiche auch zusammen fassen. So kann man anstatt
		\begin{flalign*}
			\forall a\in \mathbb{N}:\forall b\in \mathbb{N}:a>b&&
		\end{flalign*}
		kann man auch Folgendes schreiben:
		\begin{flalign*}
			\forall a,b\in \mathbb{N}:a>b.&&
		\end{flalign*}
		\subsection{Beweise}
			Oft reicht es in der Mathematik nicht aus nur ein Ergebnis zu berechnen, sondern man muss auch beweisen, dass das was man da gemacht hat stimmt. Beweise zu finden ist nicht so trivial, wie Zahlen in eine Formel einzusetzen. Man muss ein bisschen kreativ sein und viel üben. Dafür gibt es verschiedene Herangehensweisen, die man kennen sollte.
			\subsubsection{Direkter Beweis}
				Beim direkten Beweis\index{Direkter Beweis} versucht man über Zwischenschritte zu zeigen, dass aus $A$ auch wirklich $B$ folgt. Die Idee ist simpel, aber einen Beweis zu finden ist nicht immer einfach, deshalb ein Beispiel:\newline\newline
				$\text{Zu beweisen ist, dass für ein ungerades }n\in\mathbb{N} \text{ gilt, dass }n^2-1\text{ durch }4\text{ teilbar ist.}$
				\begin{flalign*}
					n\text{ ungerade}\Rightarrow\;&\text{Es gibt ein }k\in\mathbb{N}\text{, dass }n=2k+1\text{ erfüllt.}\\
					\Leftrightarrow\;&\exists k\in\mathbb{N}: n^2-1=(2k+1)^2-1\\
					\Leftrightarrow\;&\exists k\in\mathbb{N}: n^2-1=4k^2+4k\\
					\Leftrightarrow\;&\exists k\in\mathbb{N}: n^2-1=4(k^2+k)\\
					\Rightarrow\;&\exists m\in\mathbb{N}: n^2-1=4m\\
					\Rightarrow\;&n^2-1\text{ ist durch 4 teilbar, da es das Vierfache einer natürlich Zahl }m\text{ ist.}&&
				\end{flalign*}
			\subsubsection{Indirekter Beweis}
				Anders als beim direkten Beweis zeigt man beim indirekten Beweis\index{Indirekter Beweis} nicht, dass $A\Rightarrow B$ gilt, sondern $\lnot B\Rightarrow\lnot A$ oder auch $A\land\lnot B\Rightarrow F$. Dass diese Ausdrücke äquivalent sind, kann man anhand einer Wahrheitstafel überprüfen, denn sind alle Bewertungen gleich, sind die Ausdrücke äquivalent.
				\begin{center}
					\bgroup
					\def\arraystretch{1.5}
					\begin{tabularx}{\linewidth}{|l|l|@{\hspace{0.5cm}}|X|X|X|}
						\hline
						$A$ & $B$ & $A\Rightarrow B$ & $\lnot B\Rightarrow \lnot A$ & $A\land\lnot B\Rightarrow F$ \\ \hline
						$W$ & $W$ & $W$ & $W$ & $W$ \\ \hline
						$W$ & $F$ & $F$ & $F$ & $F$ \\ \hline
						$F$ & $W$ & $W$ & $W$ & $W$ \\ \hline
						$F$ & $F$ & $W$ & $W$ & $W$ \\ \hline
					\end{tabularx}
					\egroup
				\end{center}
				Dazu nochmal ein Beispiel: $\sqrt{2}\text{ ist irrational.}$
				\begin{flalign*}
					\sqrt{2}\text{ ist irrational.}\Rightarrow&\sqrt{2}\neq\frac{p}{q} \mid p, q \in \mathbb{N}, q \neq 0 \text{, p und q teilerfremd.}\\
					\Updownarrow&\\
					\sqrt{2}=\frac{p}{q} \mid p, q \in \mathbb{N}, q &\neq 0 \text{, p und q teilerfremd.}\Rightarrow\sqrt{2}\text{ ist rational.}\\
					\Rightarrow&\;\sqrt{2}=\frac{p}{q}\\
					\Leftrightarrow&\;2=\frac{p^2}{q^2}\\
					\Leftrightarrow&\;2q^2=p^2\\
					\Rightarrow&\;p^2\text{ ist gerade.}
					\Rightarrow p\text{ ist gerade.}\\
					\Rightarrow&\;p=2k\\
					\Rightarrow&\;2q^2=4k^2\\
					\Leftrightarrow&\;q^2=2k^2\\
					\Rightarrow&\;q^2\text{ ist gerade.}
					\Rightarrow q\text{ ist gerade.}\\
					\Rightarrow&\;\sqrt{2}\text{ ist irrational, da p und q beide gerade und somit}\\
					&\;\text{beide durch zwei teilbar, also nicht teilerfremd sind.}&&
				\end{flalign*}
			\subsubsection{Vollständige Induktion}
				Die vollständige Induktion\index{Vollständige Induktion} wird genutzt um Aussagen der Form $\forall n\in\mathbb{N}:A(n)$ zu beweisen. Das wird in zwei Schritten getan. Zunächst beweist man durch Einsetzen (Induktionsanfang), dass die Aussage für das kleinstmögliche $n$ wahr ist. Wenn man dann im Indikationsschritt zeigen kann, dass die Aussage, wenn sie für $n$ gilt auch für $n+1$ gilt, hat man einen Beweis. Das kommt daher, dass man für das kleinste $n$ eindeutig durch einsetzen bewiesen hat. Dieses $n$ beweist dann, dass $n+1$ die Aussage erfüllt. Dieses wiederum beweist, dass $n+2$ die Aussage erfüllt und so weiter. Auch das erklärt sich am besten wieder mal mit einem Beispiel:\newline\newline
				Zu beweisen ist diese Aussage:
				\begin{flalign*}
					A(n)=\forall n\in\mathbb{N}:\sum^n_{k=1}k=\frac{n(n+1)}{2}&&
				\end{flalign*}
				\textbf{1. Indikationsanfang}
				\begin{flalign*}
				\sum^1_{k=1}k=1\\
				\frac{1(1+1)}{2}=1&&
				\end{flalign*}
				Damit haben schon einmal bewiesen, dass es ein $n$ gibt, dass diese Aussage erfüllt.\newline\newline
				\textbf{2. Indikationsschritt}
				\begin{flalign*}
					&\text{Indikationsvoraussetzung: }\exists n\in\mathbb{N}:\sum^n_{k=1}k=\frac{n(n+1)}{2}\\
					&\text{Zu zeigen: }\sum^{n+1}_{k=1}k=\frac{(n+1)(n+2)}{2}&&
				\end{flalign*}
				Um jetzt zu zeigen, dass die Aussage für $n+1$ ebenfalls stimmt, wenn sie für $n$ stimmt, setzen wir sie gleich mit der Aussage für $n$ plus den Teil, der den Unterschied zwischen den beiden bildet ($n+1$). Das ist logisch, da wir nur noch den fehlenden Summanden drauf rechnen.
				\begin{flalign*}
					\sum^{n+1}_{k=1}k=\sum^n_{k=1}k+(n+1)&&
				\end{flalign*}
				Ich glaube, wir können uns darauf einigen, dass die obige Gleichung allgemein-mathematisch wahr ist. Wenn wir jetzt die Summenzeichen durch unsere zu zeigenden Formeln ersetzen, müsste die Gleichung – sofern die Aussage vom Anfang wahr ist – immer noch wahr sein. Dann hätten wir gezeigt, dass die Aussage für $n+1$ dasselbe ist, wie die Aussage für $n$ plus den fehlenden Teil ($n+1$).
				\begin{flalign*}
					&\frac{(n+1)(n+2)}{2}=\frac{n(n+1)}{2}+n+1\\
					\Leftrightarrow\;&(n+1)(n+2)=n(n+1)+2(n+1)\\
					\Leftrightarrow\;&n+2=n+2&&
				\end{flalign*}
				Wir sehen jetzt ganz klar, dass wir eine wahre Aussage haben. Jetzt haben wir gezeigt, dass unter der Voraussetzung, dass es ein $n$ gibt, dass die Aussage erfüllt, auch $n+1$ die Aussage erfüllt. Da wir am Anfang bewiesen haben, dass $n=1$ die Gleichung erfüllt, ist sie somit auch für $n=2$ erfüllt und wenn wir wissen, dass sie für $n=2$ erfüllt ist, ist sie auch für $n=3$ erfüllt und so weiter!
	\section{Stochastik}
		Einem stochastischen Modell liegt zunächst die Grundmenge\index{Grundmenge} oder auch das Universum\index{Universum} $\Omega$ zu Grunde. Das ist eine Menge, die alle möglichen Ausgänge für das jeweilige Zufallsexperiment\index{Zufallsexperiment} beinhaltet. Für einen Würfelwurf wäre die Grundmenge bspw. $\Omega=\{1;2;3;4;5;6\}$. Die Elemente in $\Omega$ (groß Omega) werden mit $\omega$ (klein Omega) bezeichnet und heißt Ergebnis\index{Ergebnis}. Jede Teilmenge von $\Omega$ wird als Ereignis\index{Ereignis} bezeichnet. Die Wahrscheinlichkeit eines Ereignisses wird mit einem $P$ für Probabilität angegeben: $P(\{\omega\})$. Wenn die Wahrscheinlichkeit aller Ergebnisse gleich ist und nur endliche viele mögliche Ergebnisse existieren, spricht man von einem Laplace-Experiment\index{Laplace-Experiment}.\newline\newline
		Von großer Bedeutung, wenn es um Kombinatorik\index{Kombinatorik} geht, sind die Fakultät\index{Fakultät} $n!$ und der Binomialkoeffizient\index{Binomialkoeffizient} $\begin{pmatrix}n\\k\end{pmatrix}$. Diese sind wie folgt definiert:
		\begin{center}
			\bgroup
			\def\arraystretch{0}
			\def\tabcolsep{0pt}
			\begin{tabularx}{\linewidth}{X@{\hspace{0.4cm}}X}
				\adjustbox{valign=t}{\begin{tcolorbox}[boxsep=0pt,top=.5cm,left=.5cm,right=.5cm, bottom=.5cm,arc=0pt,auto outer arc,colback=white,colframe=black]
						\textbf{Fakultät}
						\begin{flalign*}
						n!:=\prod^n_{k=1}k,&& 0!=1&&
						\end{flalign*}
				\end{tcolorbox}}
				&
				\adjustbox{valign=t}{\begin{tcolorbox}[boxsep=0pt,top=.5cm,left=.5cm,right=.5cm, bottom=.74cm,arc=0pt,auto outer arc,colback=white]
						\textbf{Binomialkoeffizient}
						\begin{flalign*}
						\genfrac{(}{)}{0pt}{}{n}{k}:=\frac{n!}{(n-k)!\cdot k!},&&\genfrac{(}{)}{0pt}{}{n}{0}=1&&
						\end{flalign*}
				\end{tcolorbox}}
			\end{tabularx}
			\egroup
		\end{center}
		Die Kombinatorik hilft uns herauszufinden, wie viele Möglichkeiten es für Anordnungen gibt. Dabei unterscheidet man verschiedene Arten von Modellen. Im Großen und Ganzen gibt es drei Arten. Die Erste ist die Permutation\index{Permutation}. Dabei untersucht man die komplette Grundmenge. Wenn man nur eine Teilmenge untersucht, gibt es noch mal eine Unterscheidung. Wenn die Reihenfolge in der Teilmenge wichtig ist, spricht man von Variation\index{Variation}, ansonsten von Kombination\index{Kombination}. Für diese drei Fälle gibt es jeweils zwei Möglichkeiten zur Berechnung der Anzahl der Möglichkeiten, je nachdem ob Elemente mehrfach auftreten oder nicht.
		\begin{center}
			\bgroup
			\tikzstyle{decision} = [rectangle, draw=red, fill=red!20, 
			text width=6em, text badly centered, node distance=3cm, inner sep=0pt, inner sep=6pt, rounded corners]
			\tikzstyle{block} = [rectangle, draw=teal, fill=teal!20, 
			text width=6em, text centered, inner sep=6pt]
			\tikzstyle{result} = [rectangle, draw=violet, fill=violet!20, 
			text centered, inner sep=6pt, minimum height=3em]
			\tikzstyle{line} = [draw, -stealth]
			\begin{tikzpicture}[align=center]
			% Place nodes
			\node [block] (init) {Grundmenge};
			\node [decision, below=.5cm of init] (decide) {alle Elemente?};
			\node [decision, below=1cm of decide] (reihenfolge) {Reihenfolge relevant?};
			\node [block, below=1cm of reihenfolge] (kombination) {Kombination};
			\node [block, left=2cm of kombination] (variation) {Variation};
			\node [block, left=2cm of variation, anchor=east] (permutation) {Permutation};
			\node [decision, below=.5cm of permutation] (permutationW) {Mehrfache Elemente?};
			\node [decision, below=.5cm of variation] (variationW) {Mehrfache Elemente?};
			\node [decision, below=.5cm of kombination] (kombinationW) {Mehrfache Elemente?};
			\node [result, below left=1cm and -0.5cm of permutationW] (permutationYes) {$\frac{k!}{m_1!\cdot m_2!\cdot ...\cdot m_n!}$};
			\node [result, below=1cm of permutationW] (permutationNo) {$n!$};
			\node [result, below left=1cm and 0cm of variationW] (variationYes) {$n^k$};
			\node [result, below=1cm of variationW] (variationNo) {$\frac{n!}{(n-k)!}$};
			\node [result, below left=1cm and 0cm of kombinationW] (kombinationYes) {$\genfrac{(}{)}{0pt}{}{n+k-1}{k}$};
			\node [result, below=1cm of kombinationW] (kombinationNo) {$\genfrac{(}{)}{0pt}{}{n}{k}$};
			% Draw edges
			\path [line] (init) -- (decide);
			\path [line] (decide) -| node [near end, left] {Ja} (permutation);
			\path [line] (decide) -- node [right] {Nein} (reihenfolge);
			\path [line] (reihenfolge) -- node [right] {Nein} (kombination);
			\path [line] (reihenfolge) -| node [near end, left] {Ja} (variation);
			\path [line] (permutation) -- (permutationW);
			\path [line] (variation) -- (variationW);
			\path [line] (kombination) -- (kombinationW);
			\path [line] (permutationW) -| node [near end, left] {Ja} (permutationYes);
			\path [line] (permutationW) -- node [left] {Nein} (permutationNo);
			\path [line] (variationW) -| node [near end, left] {Ja} (variationYes);
			\path [line] (variationW) -- node [left] {Nein} (variationNo);
			\path [line] (kombinationW) -| node [near end, left] {Ja} (kombinationYes);
			\path [line] (kombinationW) -- node [left] {Nein} (kombinationNo);
			\end{tikzpicture}
			\egroup
		\end{center}
		Die Kunst in der Kombinatorik ist es jetzt einer Aufgabenstellung die richtige Formel zuordnen zu können, daher folgt jetzt zu jedem Fall ein Beispiel.
		\subsection{Permutation}
			\subsubsection{Permutation mit Wiederholung}
			\subsubsection{Permutation ohne Wiederholung}
		\subsection{Variation}
			\subsubsection{Variation mit Wiederholung}
			\subsubsection{Variation ohne Wiederholung}
		\subsection{Kombination}
			\subsubsection{Kombination mit Wiederholung}
			\subsubsection{Kombination ohne Wiederholung}
	\pagebreak
	\printindex
\end{document}